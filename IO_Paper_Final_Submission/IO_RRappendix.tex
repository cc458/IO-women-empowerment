\section{Main Tables for Figures 3-7}
\appendixpagenumbering
\vspace*{.2in}
\setlength{\parskip}{-2em}
%\setlength{\parindent}{1cm}
\doublespacing
\setcounter{table}{0}
\setcounter{figure}{0}
\renewcommand{\thetable}{A\arabic{table}}	
\renewcommand{\thefigure}{A\arabic{figure}}	

Tables \ref{tab2}- \ref{intermpolempowerment} present coefficient estimates for all models across Figures 3-7 in our main text.  \\

% Fig 3(a) based on Models 1-8
%the main table for forward effect
{\renewcommand\normalsize{\tiny}% adjust table size
	\normalsize
\begin{table}[htbp]\centering
\def\sym#1{\ifmmode^{#1}\else\(^{#1}\)\fi}
\caption{Fixed-effects models of the effect of war on future changes in women's political empowerment\label{tab2}}
\begin{tabular}{l*{8}{c}}
\hline\hline
                    &\multicolumn{1}{c}{\shortstack{Model 1\\(current)}}&\multicolumn{1}{c}{\shortstack{Model 2\\(1-year)}}&\multicolumn{1}{c}{\shortstack{Model 3\\(2-year)}}&\multicolumn{1}{c}{\shortstack{Model 4\\(3-year)}}&\multicolumn{1}{c}{\shortstack{Model 5\\(4-year)}}&\multicolumn{1}{c}{\shortstack{Model 6\\(5-year)}}&\multicolumn{1}{c}{\shortstack{Model 7\\(10-year)}}&\multicolumn{1}{c}{\shortstack{Model 8\\(15-year)}}\\
\hline
War            &    -0.00155         &    0.000628         &     0.00346         &     0.00697\sym{*}  &     0.00930\sym{**} &      0.0101\sym{*}  &      0.0101         &     0.00506         \\
                    &   (0.00124)         &   (0.00209)         &   (0.00293)         &   (0.00360)         &   (0.00443)         &   (0.00523)         &   (0.00801)         &   (0.00916)         \\
[1em]
Political empowerment, lag   &     -0.0519\sym{***}&     -0.0949\sym{***}&      -0.140\sym{***}&      -0.189\sym{***}&      -0.236\sym{***}&      -0.286\sym{***}&      -0.508\sym{***}&      -0.675\sym{***}\\
                    &   (0.00484)         &   (0.00847)         &    (0.0118)         &    (0.0155)         &    (0.0189)         &    (0.0227)         &    (0.0381)         &    (0.0447)         \\
[1em]
$\Delta$Polity scores           &     0.00318\sym{***}&     0.00572\sym{***}&     0.00595\sym{***}&     0.00539\sym{***}&     0.00509\sym{***}&     0.00482\sym{***}&     0.00357\sym{***}&     0.00356\sym{***}\\
                    &  (0.000434)         &  (0.000759)         &  (0.000870)         &  (0.000869)         &  (0.000928)         &  (0.000895)         &  (0.000787)         &  (0.000835)         \\
[1em]
Polity scores, lag           &    0.000708\sym{***}&    0.000928\sym{***}&     0.00103\sym{***}&     0.00117\sym{***}&     0.00128\sym{***}&     0.00149\sym{***}&     0.00193\sym{**} &     0.00129         \\
                    & (0.0000937)         &  (0.000164)         &  (0.000231)         &  (0.000299)         &  (0.000362)         &  (0.000434)         &  (0.000783)         &  (0.000956)         \\
[1em]
$\Delta$Energy.consump,log           &    0.000444         &  -0.0000561         &   -0.000180         &   -0.000828         &    -0.00130         &    -0.00311         &    -0.00467         &    -0.00476         \\
                    &  (0.000834)         &  (0.000891)         &   (0.00109)         &   (0.00133)         &   (0.00158)         &   (0.00219)         &   (0.00309)         &   (0.00374)         \\
[1em]
Energy.consump, log, lag           &    0.000286         &    0.000555         &    0.000946         &     0.00138\sym{*}  &     0.00184\sym{*}  &     0.00231\sym{**} &     0.00478\sym{**} &     0.00799\sym{***}\\
                    &  (0.000205)         &  (0.000388)         &  (0.000586)         &  (0.000796)         &  (0.000984)         &   (0.00116)         &   (0.00204)         &   (0.00284)         \\
[1em]
Year               &    0.000253\sym{***}&    0.000507\sym{***}&    0.000777\sym{***}&     0.00107\sym{***}&     0.00134\sym{***}&     0.00162\sym{***}&     0.00290\sym{***}&     0.00384\sym{***}\\
                    & (0.0000295)         & (0.0000552)         & (0.0000819)         &  (0.000110)         &  (0.000135)         &  (0.000160)         &  (0.000283)         &  (0.000380)         \\
[1em]
$\Delta$Neighboring empowerment, lag&     0.00606\sym{*}  &      0.0189\sym{***}&      0.0227\sym{***}&      0.0238\sym{***}&      0.0190\sym{**} &      0.0224\sym{**} &      0.0144         &      0.0250\sym{*}  \\
                    &   (0.00361)         &   (0.00574)         &   (0.00853)         &   (0.00861)         &   (0.00834)         &   (0.00872)         &    (0.0139)         &    (0.0132)         \\
[1em]
Constant            &      -0.471\sym{***}&      -0.949\sym{***}&      -1.456\sym{***}&      -1.997\sym{***}&      -2.510\sym{***}&      -3.044\sym{***}&      -5.437\sym{***}&      -7.197\sym{***}\\
                    &    (0.0552)         &     (0.104)         &     (0.154)         &     (0.207)         &     (0.253)         &     (0.301)         &     (0.534)         &     (0.716)         \\
\hline
Observations        &        8090         &        7889         &        7708         &        7527         &        7362         &        7209         &        6538         &        6009         \\
\hline\hline
\multicolumn{9}{l}{\footnotesize Standard errors in parentheses}\\
\multicolumn{9}{l}{\footnotesize \sym{*} \(p<0.10\), \sym{**} \(p<0.05\), \sym{***} \(p<0.01\)}\\
\end{tabular}
\end{table}
 %label {tab2}  %index Models 1-8
}

% Fig 3(b-d) based on Models 9 -16
% the types of war
{\renewcommand\normalsize{\tiny}% adjust table size
	\normalsize
\begin{table}[htbp]\centering
\def\sym#1{\ifmmode^{#1}\else\(^{#1}\)\fi}
\caption{Fixed-effects models of the effect of war types on future changes in women's political empowerment\label{tab3}}
\begin{tabular}{l*{8}{c}}
\hline\hline
                    &\multicolumn{1}{c}{\shortstack{Model 9\\(current)}}&\multicolumn{1}{c}{\shortstack{Model 10\\(1-year)}}&\multicolumn{1}{c}{\shortstack{Model 11\\(2-year)}}&\multicolumn{1}{c}{\shortstack{Model 12\\(3-year)}}&\multicolumn{1}{c}{\shortstack{Model 13\\(4-year)}}&\multicolumn{1}{c}{\shortstack{Model 14\\(5-year)}}&\multicolumn{1}{c}{\shortstack{Model 15\\(10-year)}}&\multicolumn{1}{c}{\shortstack{Model 16\\(15-year)}}\\
\hline
New war              &    -0.00186         &   -0.000833         &    0.000581         &   -0.000171         &    0.000784         &     0.00338         &     0.00492         &     0.00203         \\
                    &   (0.00166)         &   (0.00277)         &   (0.00377)         &   (0.00460)         &   (0.00467)         &   (0.00490)         &   (0.00752)         &   (0.00887)         \\
[1em]
Ongoing war          &   -0.000645         &     0.00229         &     0.00588\sym{*}  &      0.0109\sym{**} &      0.0138\sym{**} &      0.0139\sym{**} &      0.0128         &     0.00639         \\
                    &   (0.00149)         &   (0.00244)         &   (0.00340)         &   (0.00443)         &   (0.00557)         &   (0.00662)         &   (0.00979)         &    (0.0110)         \\
[1em]
Recent war           &     0.00693\sym{***}&     0.00912\sym{***}&      0.0106\sym{**} &     0.00751\sym{*}  &     0.00752         &     0.00828         &     0.00389         &    0.000842         \\
                    &   (0.00228)         &   (0.00340)         &   (0.00411)         &   (0.00450)         &   (0.00481)         &   (0.00510)         &   (0.00723)         &   (0.00829)         \\
[1em]
Political empowerment, lag   &     -0.0514\sym{***}&     -0.0940\sym{***}&      -0.139\sym{***}&      -0.188\sym{***}&      -0.234\sym{***}&      -0.284\sym{***}&      -0.506\sym{***}&      -0.674\sym{***}\\
                    &   (0.00487)         &   (0.00842)         &    (0.0118)         &    (0.0154)         &    (0.0188)         &    (0.0225)         &    (0.0381)         &    (0.0450)         \\
[1em]
$\Delta$Polity scores            &     0.00319\sym{***}&     0.00572\sym{***}&     0.00596\sym{***}&     0.00538\sym{***}&     0.00508\sym{***}&     0.00482\sym{***}&     0.00356\sym{***}&     0.00356\sym{***}\\
                    &  (0.000433)         &  (0.000755)         &  (0.000867)         &  (0.000865)         &  (0.000926)         &  (0.000893)         &  (0.000787)         &  (0.000833)         \\
[1em]
Polity scores, lag              &    0.000700\sym{***}&    0.000915\sym{***}&     0.00102\sym{***}&     0.00115\sym{***}&     0.00126\sym{***}&     0.00147\sym{***}&     0.00191\sym{**} &     0.00128         \\
                    & (0.0000946)         &  (0.000164)         &  (0.000230)         &  (0.000298)         &  (0.000360)         &  (0.000432)         &  (0.000784)         &  (0.000958)         \\
[1em]
$\Delta$Energy.consump,log              &    0.000407         &   -0.000127         &   -0.000265         &    -0.00100         &    -0.00151         &    -0.00331         &    -0.00477         &    -0.00480         \\
                    &  (0.000835)         &  (0.000888)         &   (0.00108)         &   (0.00132)         &   (0.00158)         &   (0.00220)         &   (0.00310)         &   (0.00376)         \\
[1em]
Energy.consump, log, lag               &    0.000278         &    0.000548         &    0.000924         &     0.00135\sym{*}  &     0.00181\sym{*}  &     0.00227\sym{*}  &     0.00475\sym{**} &     0.00798\sym{***}\\
                    &  (0.000204)         &  (0.000390)         &  (0.000588)         &  (0.000796)         &  (0.000984)         &   (0.00116)         &   (0.00204)         &   (0.00284)         \\
[1em]
Year                &    0.000253\sym{***}&    0.000505\sym{***}&    0.000774\sym{***}&     0.00106\sym{***}&     0.00133\sym{***}&     0.00162\sym{***}&     0.00290\sym{***}&     0.00384\sym{***}\\
                    & (0.0000294)         & (0.0000551)         & (0.0000818)         &  (0.000109)         &  (0.000134)         &  (0.000160)         &  (0.000283)         &  (0.000380)         \\
[1em]
$\Delta$Neighboring empowerment, lag&     0.00611\sym{*}  &      0.0192\sym{***}&      0.0228\sym{***}&      0.0239\sym{***}&      0.0192\sym{**} &      0.0225\sym{**} &      0.0146         &      0.0250\sym{*}  \\
                    &   (0.00359)         &   (0.00575)         &   (0.00853)         &   (0.00864)         &   (0.00837)         &   (0.00872)         &    (0.0139)         &    (0.0132)         \\
[1em]
Constant            &      -0.470\sym{***}&      -0.946\sym{***}&      -1.451\sym{***}&      -1.987\sym{***}&      -2.499\sym{***}&      -3.035\sym{***}&      -5.431\sym{***}&      -7.194\sym{***}\\
                    &    (0.0550)         &     (0.103)         &     (0.154)         &     (0.206)         &     (0.253)         &     (0.301)         &     (0.535)         &     (0.717)         \\
\hline
Observations        &        8090         &        7889         &        7708         &        7527         &        7362         &        7209         &        6538         &        6009         \\
\hline\hline
\multicolumn{9}{l}{\footnotesize Standard errors in parentheses}\\
\multicolumn{9}{l}{\footnotesize \sym{*} \(p<0.10\), \sym{**} \(p<0.05\), \sym{***} \(p<0.01\)}\\
\end{tabular}
\end{table}
 %label {tab3}  % index models 9 -16
}


% Fig 4 based on Models 17-24
% IV model
\begin{landscape}
{\renewcommand\normalsize{\tiny}% adjust table size
	\normalsize
\begin{table}[htbp]\centering
\def\sym#1{\ifmmode^{#1}\else\(^{#1}\)\fi}
\caption{Endogenous treatment-regression models of the effect of war on future changes in women’s political empowerment\label{ivpolempowerment}}
\begin{tabular}{l*{8}{c}}
\hline\hline
                    &\multicolumn{1}{c}{\shortstack{Model 17\\(current)}}&\multicolumn{1}{c}{\shortstack{Model 18\\(1-year)}}&\multicolumn{1}{c}{\shortstack{Model 19\\(2-year)}}&\multicolumn{1}{c}{\shortstack{Model 20\\(3-year)}}&\multicolumn{1}{c}{\shortstack{Model 21\\(4-year)}}&\multicolumn{1}{c}{\shortstack{Model 22\\(5-year)}}&\multicolumn{1}{c}{\shortstack{Model 23\\(10-year)}}&\multicolumn{1}{c}{\shortstack{Model 24\\(15-year)}}\\
\hline
$\Delta$Polity scores        &     0.00311\sym{***}&     0.00556\sym{***}&     0.00586\sym{***}&     0.00535\sym{***}&     0.00521\sym{***}&     0.00501\sym{***}&     0.00322\sym{***}&     0.00345\sym{***}\\
                    &  (0.000434)         &  (0.000753)         &  (0.000883)         &  (0.000917)         &   (0.00103)         &   (0.00102)         &  (0.000854)         &  (0.000855)         \\
[1em]
Polity scores, lag           &    0.000135\sym{***}&   0.0000566         &  -0.0000718         &   -0.000197         &   -0.000333\sym{*}  &   -0.000453\sym{**} &    -0.00124\sym{***}&    -0.00209\sym{***}\\
                    & (0.0000465)         & (0.0000744)         &  (0.000108)         &  (0.000143)         &  (0.000179)         &  (0.000216)         &  (0.000425)         &  (0.000635)         \\
[1em]
$\Delta$Energy.consump,log             &    0.000828         &    0.000435         &    0.000310         &   -0.000400         &    -0.00102         &    -0.00250         &    -0.00579\sym{**} &    -0.00518         \\
                    &  (0.000923)         &   (0.00107)         &   (0.00126)         &   (0.00138)         &   (0.00164)         &   (0.00217)         &   (0.00268)         &   (0.00319)         \\
[1em]
Energy.consump,log, lag             &   -0.000209\sym{**} &  -0.0000758         &   0.0000934         &    0.000229         &    0.000344         &    0.000472         &     0.00232\sym{***}&     0.00445\sym{***}\\
                    &  (0.000105)         &  (0.000177)         &  (0.000261)         &  (0.000327)         &  (0.000384)         &  (0.000447)         &  (0.000804)         &   (0.00124)         \\
[1em]
Year                &   0.0000579\sym{***}&    0.000108\sym{***}&    0.000163\sym{***}&    0.000223\sym{***}&    0.000287\sym{***}&    0.000359\sym{***}&    0.000606\sym{***}&    0.000830\sym{***}\\
                    & (0.0000113)         & (0.0000195)         & (0.0000284)         & (0.0000366)         & (0.0000443)         & (0.0000546)         & (0.0000912)         &  (0.000144)         \\
[1em]
War         &      0.0272\sym{***}&      0.0346\sym{***}&      0.0417\sym{**} &      0.0526\sym{**} &      0.0651\sym{***}&      0.0735\sym{***}&      0.0282         &      0.0144         \\
                    &   (0.00384)         &    (0.0108)         &    (0.0190)         &    (0.0223)         &    (0.0217)         &    (0.0248)         &    (0.0172)         &    (0.0136)         \\
[1em]
Constant            &      -0.111\sym{***}&      -0.206\sym{***}&      -0.312\sym{***}&      -0.428\sym{***}&      -0.549\sym{***}&      -0.688\sym{***}&      -1.154\sym{***}&      -1.576\sym{***}\\
                    &    (0.0220)         &    (0.0382)         &    (0.0559)         &    (0.0718)         &    (0.0868)         &     (0.107)         &     (0.178)         &     (0.281)         \\
\hline
Treatment = War                &                     &                     &                     &                     &                     &                     &                     &                     \\
Neigh.civil.war, count, lag&       0.114\sym{***}&      0.0869\sym{*}  &      0.0635         &      0.0760         &      0.0653         &      0.0727         &     0.00943         &      0.0553         \\
                    &    (0.0432)         &    (0.0521)         &    (0.0542)         &    (0.0561)         &    (0.0536)         &    (0.0534)         &    (0.0639)         &    (0.0664)         \\
[1em]
Neigh.civil.war, count, 2 lags&      0.0580         &      0.0869         &       0.104\sym{*}  &      0.0968\sym{*}  &      0.0925\sym{*}  &      0.0745         &       0.136\sym{*}  &       0.109         \\
                    &    (0.0445)         &    (0.0531)         &    (0.0553)         &    (0.0571)         &    (0.0563)         &    (0.0585)         &    (0.0738)         &    (0.0736)         \\
[1em]
Neigh.interstate.wars, count, lag&       0.309\sym{***}&       0.319\sym{***}&       0.369\sym{***}&       0.373\sym{***}&       0.371\sym{***}&       0.391\sym{***}&       0.408\sym{***}&       0.435\sym{***}\\
                    &    (0.0657)         &    (0.0965)         &    (0.0939)         &    (0.0920)         &    (0.0890)         &    (0.0885)         &    (0.0725)         &    (0.0804)         \\
[1em]
Neigh.interstate.wars, count, 2 lags&      0.0268         &      0.0655         &      0.0564         &      0.0604         &      0.0612         &      0.0345         &      0.0149         &     -0.0631         \\
                    &    (0.0499)         &    (0.0588)         &    (0.0490)         &    (0.0496)         &    (0.0505)         &    (0.0446)         &    (0.0557)         &    (0.0619)         \\
[1em]
$\Delta$Polity scores            &     -0.0387\sym{**} &     -0.0446\sym{**} &     -0.0403         &     -0.0431\sym{**} &     -0.0509\sym{***}&     -0.0461\sym{**} &     -0.0364\sym{***}&     -0.0236\sym{*}  \\
                    &    (0.0169)         &    (0.0227)         &    (0.0255)         &    (0.0207)         &    (0.0195)         &    (0.0186)         &    (0.0140)         &    (0.0139)         \\
[1em]
Polity scores, lag             &    -0.00943         &    -0.00918         &    -0.00904         &    -0.00864         &    -0.00793         &    -0.00754         &    -0.00991         &    -0.00917         \\
                    &   (0.00613)         &   (0.00659)         &   (0.00693)         &   (0.00719)         &   (0.00730)         &   (0.00749)         &   (0.00766)         &   (0.00769)         \\
[1em]
$\Delta$Energy.consump,log             &     -0.0464         &     -0.0636         &     -0.0779         &     -0.0833         &     -0.0867         &      -0.115         &      -0.120         &      -0.129         \\
                    &    (0.0654)         &    (0.0663)         &    (0.0725)         &    (0.0918)         &    (0.0915)         &     (0.110)         &     (0.107)         &     (0.106)         \\
[1em]
Energy.consump,log, lag              &      0.0526\sym{***}&      0.0524\sym{***}&      0.0538\sym{***}&      0.0527\sym{***}&      0.0515\sym{***}&      0.0527\sym{***}&      0.0753\sym{***}&      0.0657\sym{***}\\
                    &    (0.0139)         &    (0.0152)         &    (0.0164)         &    (0.0168)         &    (0.0171)         &    (0.0174)         &    (0.0163)         &    (0.0150)         \\
[1em]
Year                &    -0.00376\sym{***}&    -0.00327\sym{**} &    -0.00305\sym{*}  &    -0.00290\sym{*}  &    -0.00257         &    -0.00291\sym{*}  &    -0.00243         &    -0.00207         \\
                    &   (0.00139)         &   (0.00157)         &   (0.00163)         &   (0.00162)         &   (0.00164)         &   (0.00168)         &   (0.00198)         &   (0.00210)         \\
[1em]
Constant            &       5.702\sym{**} &       4.677         &       4.212         &       3.914         &       3.300         &       3.969         &       2.800         &       2.175         \\
                    &     (2.709)         &     (3.061)         &     (3.198)         &     (3.170)         &     (3.199)         &     (3.268)         &     (3.844)         &     (4.065)         \\
\hline
ath$\rho$               &      -0.758\sym{***}&      -0.572\sym{**} &      -0.505\sym{*}  &      -0.512\sym{*}  &      -0.559\sym{**} &      -0.567\sym{**} &     -0.0713         &    -0.00461         \\
                    &     (0.126)         &     (0.223)         &     (0.302)         &     (0.303)         &     (0.267)         &     (0.280)         &     (0.104)         &    (0.0524)         \\
[1em]
lnsigma             &      -3.750\sym{***}&      -3.393\sym{***}&      -3.174\sym{***}&      -3.013\sym{***}&      -2.885\sym{***}&      -2.785\sym{***}&      -2.506\sym{***}&      -2.329\sym{***}\\
                    &    (0.0575)         &    (0.0681)         &    (0.0759)         &    (0.0786)         &    (0.0782)         &    (0.0816)         &    (0.0620)         &    (0.0633)         \\
\hline
Observations        &        8090         &        7889         &        7708         &        7527         &        7362         &        7209         &        6538         &        6009         \\
\hline\hline
\multicolumn{9}{l}{\footnotesize Standard errors in parentheses}\\
\multicolumn{9}{l}{\footnotesize \sym{*} \(p<0.10\), \sym{**} \(p<0.05\), \sym{***} \(p<0.01\)}\\
\end{tabular}
\end{table}
 %label {ivpolempowerment} % index models 17-24
}
\end{landscape}

% Fig 5(a) based Models 25-32
% war duration and women's empowerment
{\renewcommand\normalsize{\tiny}% adjust table size
	\normalsize
\begin{table}[htbp]\centering
\def\sym#1{\ifmmode^{#1}\else\(^{#1}\)\fi}
\caption{Fixed-effects models of the effect of war duration on future changes in women's political empowerment\label{polwardur}}
\begin{tabular}{l*{8}{c}}
\hline\hline
                    &\multicolumn{1}{c}{\shortstack{Model 25\\(current)}}&\multicolumn{1}{c}{\shortstack{Model 26\\(1-year)}}&\multicolumn{1}{c}{\shortstack{Model 27\\(2-year)}}&\multicolumn{1}{c}{\shortstack{Model 28\\(3-year)}}&\multicolumn{1}{c}{\shortstack{Model 29\\(4-year)}}&\multicolumn{1}{c}{\shortstack{Model 30\\(5-year)}}&\multicolumn{1}{c}{\shortstack{Model 31\\(10-year)}}&\multicolumn{1}{c}{\shortstack{Model 32\\(15-year)}}\\
\hline
War duration             &    0.000258\sym{*}  &    0.000782\sym{***}&     0.00127\sym{***}&     0.00173\sym{***}&     0.00198\sym{***}&     0.00205\sym{***}&     0.00172         &     0.00139         \\
                    &  (0.000152)         &  (0.000235)         &  (0.000347)         &  (0.000508)         &  (0.000649)         &  (0.000751)         &   (0.00116)         &   (0.00121)         \\
[1em]
Political empowerment, lag  &     -0.0509\sym{***}&     -0.0940\sym{***}&      -0.139\sym{***}&      -0.189\sym{***}&      -0.236\sym{***}&      -0.286\sym{***}&      -0.506\sym{***}&      -0.673\sym{***}\\
                    &   (0.00481)         &   (0.00841)         &    (0.0118)         &    (0.0155)         &    (0.0189)         &    (0.0226)         &    (0.0381)         &    (0.0440)         \\
[1em]
$\Delta$Polity scores            &     0.00319\sym{***}&     0.00574\sym{***}&     0.00598\sym{***}&     0.00540\sym{***}&     0.00509\sym{***}&     0.00484\sym{***}&     0.00355\sym{***}&     0.00354\sym{***}\\
                    &  (0.000435)         &  (0.000760)         &  (0.000871)         &  (0.000870)         &  (0.000929)         &  (0.000897)         &  (0.000793)         &  (0.000838)         \\
[1em]
Polity scores, lag           &    0.000700\sym{***}&    0.000924\sym{***}&     0.00104\sym{***}&     0.00119\sym{***}&     0.00130\sym{***}&     0.00151\sym{***}&     0.00190\sym{**} &     0.00125         \\
                    & (0.0000939)         &  (0.000163)         &  (0.000228)         &  (0.000295)         &  (0.000356)         &  (0.000428)         &  (0.000781)         &  (0.000948)         \\
[1em]
$\Delta$Energy.consump,log           &    0.000443         &   -0.000129         &   -0.000315         &    -0.00110         &    -0.00168         &    -0.00352         &    -0.00494         &    -0.00489         \\
                    &  (0.000849)         &  (0.000922)         &   (0.00110)         &   (0.00134)         &   (0.00158)         &   (0.00218)         &   (0.00310)         &   (0.00378)         \\
[1em]
Energy.consump,log, lag             &    0.000255         &    0.000511         &    0.000887         &     0.00131\sym{*}  &     0.00177\sym{*}  &     0.00223\sym{*}  &     0.00477\sym{**} &     0.00796\sym{***}\\
                    &  (0.000207)         &  (0.000386)         &  (0.000581)         &  (0.000787)         &  (0.000973)         &   (0.00115)         &   (0.00204)         &   (0.00285)         \\
[1em]
Year                &    0.000251\sym{***}&    0.000504\sym{***}&    0.000772\sym{***}&     0.00106\sym{***}&     0.00133\sym{***}&     0.00162\sym{***}&     0.00289\sym{***}&     0.00382\sym{***}\\
                    & (0.0000293)         & (0.0000550)         & (0.0000819)         &  (0.000110)         &  (0.000135)         &  (0.000160)         &  (0.000284)         &  (0.000378)         \\
[1em]
$\Delta$Neighboring empowerment, lag&     0.00573         &      0.0185\sym{***}&      0.0227\sym{***}&      0.0237\sym{***}&      0.0195\sym{**} &      0.0230\sym{***}&      0.0154         &      0.0256\sym{*}  \\
                    &   (0.00360)         &   (0.00574)         &   (0.00861)         &   (0.00870)         &   (0.00845)         &   (0.00877)         &    (0.0140)         &    (0.0132)         \\
[1em]
Constant            &      -0.467\sym{***}&      -0.942\sym{***}&      -1.446\sym{***}&      -1.985\sym{***}&      -2.498\sym{***}&      -3.030\sym{***}&      -5.411\sym{***}&      -7.168\sym{***}\\
                    &    (0.0550)         &     (0.103)         &     (0.154)         &     (0.207)         &     (0.254)         &     (0.302)         &     (0.535)         &     (0.714)         \\
\hline
Observations        &        8062         &        7861         &        7680         &        7498         &        7333         &        7180         &        6509         &        5980         \\
\hline\hline
\multicolumn{9}{l}{\footnotesize Standard errors in parentheses}\\
\multicolumn{9}{l}{\footnotesize \sym{*} \(p<0.10\), \sym{**} \(p<0.05\), \sym{***} \(p<0.01\)}\\
\end{tabular}
\end{table}
 %label {polwardur} % index models  25-32
}

%Fig 5(b) based on Models 33-40
% battle deaths
{\renewcommand\normalsize{\tiny}% adjust table size
	\normalsize
\begin{table}[htbp]\centering
\def\sym#1{\ifmmode^{#1}\else\(^{#1}\)\fi}
\caption{Fixed-effects models of the effect of battle deaths on future changes in women's political empowerment\label{polbdeath}}
\begin{tabular}{l*{8}{c}}
\hline\hline
                    &\multicolumn{1}{c}{\shortstack{Model 33\\(current)}}&\multicolumn{1}{c}{\shortstack{Model 34\\(1-year)}}&\multicolumn{1}{c}{\shortstack{Model 35\\(2-year)}}&\multicolumn{1}{c}{\shortstack{Model 36\\(3-year)}}&\multicolumn{1}{c}{\shortstack{Model 37\\(4-year)}}&\multicolumn{1}{c}{\shortstack{Model 38\\(5-year)}}&\multicolumn{1}{c}{\shortstack{Model 39\\(10-year)}}&\multicolumn{1}{c}{\shortstack{Model 40\\(15-year)}}\\
\hline
Battle deaths, log        &   -0.000532\sym{*}  &   -0.000508         &   -0.000514         &   -0.000356         &  -0.0000202         &   -0.000159         &    0.000192         &     0.00156         \\
                    &  (0.000275)         &  (0.000444)         &  (0.000583)         &  (0.000747)         &  (0.000935)         &   (0.00113)         &   (0.00125)         &   (0.00145)         \\
[1em]
Political empowerment, lag  &     -0.0739\sym{***}&      -0.134\sym{***}&      -0.192\sym{***}&      -0.249\sym{***}&      -0.301\sym{***}&      -0.360\sym{***}&      -0.617\sym{***}&      -0.810\sym{***}\\
                    &   (0.00693)         &    (0.0112)         &    (0.0157)         &    (0.0198)         &    (0.0233)         &    (0.0272)         &    (0.0414)         &    (0.0450)         \\
[1em]
$\Delta$Polity scores          &     0.00370\sym{***}&     0.00641\sym{***}&     0.00649\sym{***}&     0.00590\sym{***}&     0.00541\sym{***}&     0.00483\sym{***}&     0.00373\sym{***}&     0.00355\sym{***}\\
                    &  (0.000595)         &  (0.000937)         &   (0.00104)         &  (0.000984)         &  (0.000968)         &  (0.000922)         &  (0.000774)         &  (0.000685)         \\
[1em]
Polity scores, lag          &     0.00109\sym{***}&     0.00151\sym{***}&     0.00173\sym{***}&     0.00191\sym{***}&     0.00204\sym{***}&     0.00228\sym{***}&     0.00331\sym{***}&     0.00293\sym{***}\\
                    &  (0.000135)         &  (0.000222)         &  (0.000296)         &  (0.000362)         &  (0.000419)         &  (0.000486)         &  (0.000830)         &  (0.000949)         \\
[1em]
$\Delta$Energy.consump,log             &    0.000109         &   -0.000193         &   -0.000539         &   -0.000828         &   -0.000851         &    -0.00116         &    -0.00171         &    -0.00269         \\
                    &  (0.000746)         &  (0.000895)         &   (0.00122)         &   (0.00147)         &   (0.00171)         &   (0.00244)         &   (0.00295)         &   (0.00368)         \\
[1em]
Energy.consump,log, lag             &    0.000280         &    0.000415         &    0.000749         &     0.00131         &     0.00206         &     0.00293\sym{*}  &     0.00530\sym{**} &     0.00737\sym{**} \\
                    &  (0.000295)         &  (0.000536)         &  (0.000806)         &   (0.00108)         &   (0.00132)         &   (0.00157)         &   (0.00266)         &   (0.00356)         \\
[1em]
Year                &    0.000358\sym{***}&    0.000735\sym{***}&     0.00111\sym{***}&     0.00146\sym{***}&     0.00177\sym{***}&     0.00211\sym{***}&     0.00377\sym{***}&     0.00515\sym{***}\\
                    & (0.0000502)         & (0.0000905)         &  (0.000137)         &  (0.000181)         &  (0.000224)         &  (0.000272)         &  (0.000478)         &  (0.000689)         \\
[1em]
$\Delta$Neighboring  empowerment, lag &     0.00892\sym{**} &      0.0204\sym{***}&      0.0249\sym{***}&      0.0264\sym{***}&      0.0212\sym{**} &      0.0290\sym{***}&      0.0239\sym{*}  &      0.0363\sym{**} \\
                    &   (0.00424)         &   (0.00658)         &   (0.00951)         &   (0.00994)         &   (0.01000)         &    (0.0107)         &    (0.0137)         &    (0.0141)         \\
[1em]
Constant            &      -0.666\sym{***}&      -1.378\sym{***}&      -2.083\sym{***}&      -2.744\sym{***}&      -3.333\sym{***}&      -3.967\sym{***}&      -7.115\sym{***}&      -9.730\sym{***}\\
                    &    (0.0954)         &     (0.172)         &     (0.262)         &     (0.347)         &     (0.428)         &     (0.521)         &     (0.918)         &     (1.326)         \\
\hline
Observations        &        5938         &        5853         &        5798         &        5752         &        5728         &        5651         &        5101         &        4569         \\
\hline\hline
\multicolumn{9}{l}{\footnotesize Standard errors in parentheses}\\
\multicolumn{9}{l}{\footnotesize \sym{*} \(p<0.10\), \sym{**} \(p<0.05\), \sym{***} \(p<0.01\)}\\
\end{tabular}
\end{table}
 %label {polbdeath} % index models  33-40
}

% Fig 6(a) based on Models 41-48
% fertility rates
{\renewcommand\normalsize{\tiny}% adjust table size
	\normalsize
\begin{table}[htbp]\centering
\def\sym#1{\ifmmode^{#1}\else\(^{#1}\)\fi}
\caption{Fixed-effects models of the effect of war on future changes in fertility rates\label{fefertility}}
\begin{tabular}{l*{8}{c}}
\hline\hline
                    &\multicolumn{1}{c}{\shortstack{Model 41\\(current)}}&\multicolumn{1}{c}{\shortstack{Model 42\\(1-year)}}&\multicolumn{1}{c}{\shortstack{Model 43\\(2-year)}}&\multicolumn{1}{c}{\shortstack{Model 44\\(3-year)}}&\multicolumn{1}{c}{\shortstack{Model 45\\(4-year)}}&\multicolumn{1}{c}{\shortstack{Model 46\\(5-year)}}&\multicolumn{1}{c}{\shortstack{Model 47\\(10-year)}}&\multicolumn{1}{c}{\shortstack{Model 48\\(15-year)}}\\
\hline
War            &    -0.00172         &    -0.00335         &    -0.00600         &    -0.00919         &     -0.0110         &     -0.0146         &     -0.0347         &     -0.0719         \\
                    &   (0.00533)         &    (0.0104)         &    (0.0156)         &    (0.0208)         &    (0.0253)         &    (0.0305)         &    (0.0545)         &    (0.0734)         \\
[1em]
Fertility rate, lag      &     -0.0124\sym{***}&     -0.0328\sym{***}&     -0.0593\sym{***}&     -0.0910\sym{***}&      -0.128\sym{***}&      -0.168\sym{***}&      -0.410\sym{***}&      -0.671\sym{***}\\
                    &   (0.00243)         &   (0.00490)         &   (0.00744)         &    (0.0100)         &    (0.0127)         &    (0.0154)         &    (0.0278)         &    (0.0354)         \\
[1em]
$\Delta$Polity scores             &    -0.00164\sym{***}&    -0.00346\sym{***}&    -0.00477\sym{***}&    -0.00624\sym{***}&    -0.00749\sym{***}&    -0.00913\sym{***}&     -0.0115\sym{***}&     -0.0108\sym{***}\\
                    &  (0.000519)         &  (0.000936)         &   (0.00132)         &   (0.00168)         &   (0.00208)         &   (0.00231)         &   (0.00329)         &   (0.00364)         \\
[1em]
Polity scores, lag        &    -0.00103\sym{**} &    -0.00212\sym{**} &    -0.00330\sym{**} &    -0.00442\sym{**} &    -0.00551\sym{**} &    -0.00643\sym{**} &    -0.00824\sym{*}  &    -0.00733         \\
                    &  (0.000479)         &  (0.000943)         &   (0.00140)         &   (0.00184)         &   (0.00226)         &   (0.00266)         &   (0.00444)         &   (0.00568)         \\
[1em]
$\Delta$Energy.consump,log           &    -0.00132         &    -0.00223         &    -0.00518         &     -0.0137\sym{*}  &     -0.0164\sym{*}  &     -0.0196\sym{*}  &     -0.0233         &     -0.0297         \\
                    &   (0.00143)         &   (0.00277)         &   (0.00417)         &   (0.00758)         &   (0.00949)         &    (0.0113)         &    (0.0181)         &    (0.0196)         \\
[1em]
Energy.consump,log, lag             &     -0.0160\sym{***}&     -0.0317\sym{***}&     -0.0496\sym{***}&     -0.0652\sym{***}&     -0.0789\sym{***}&     -0.0914\sym{***}&      -0.135\sym{***}&      -0.151\sym{***}\\
                    &   (0.00217)         &   (0.00451)         &   (0.00744)         &    (0.0102)         &    (0.0128)         &    (0.0154)         &    (0.0254)         &    (0.0280)         \\
[1em]
Year                &    0.000538\sym{**} &    0.000702         &    0.000584         &    0.000115         &   -0.000622         &    -0.00165         &     -0.0100\sym{***}&     -0.0219\sym{***}\\
                    &  (0.000247)         &  (0.000509)         &  (0.000798)         &   (0.00111)         &   (0.00144)         &   (0.00179)         &   (0.00357)         &   (0.00491)         \\
[1em]
$\Delta$Neighboring fertility rates, lag&      0.0357\sym{***}&      0.0689\sym{***}&      0.0939\sym{***}&       0.118\sym{***}&       0.144\sym{***}&       0.166\sym{***}&       0.209\sym{***}&       0.189\sym{**} \\
                    &    (0.0117)         &    (0.0225)         &    (0.0309)         &    (0.0391)         &    (0.0484)         &    (0.0558)         &    (0.0752)         &    (0.0733)         \\
[1em]
Constant            &      -0.923\sym{*}  &      -1.073         &      -0.618         &       0.539         &       2.232         &       4.501         &       22.35\sym{***}&       46.93\sym{***}\\
                    &     (0.482)         &     (0.993)         &     (1.554)         &     (2.164)         &     (2.806)         &     (3.489)         &     (7.000)         &     (9.701)         \\
\hline
Observations        &        6677         &        6651         &        6550         &        6446         &        6316         &        6185         &        5530         &        4885         \\
\hline\hline
\multicolumn{9}{l}{\footnotesize Standard errors in parentheses}\\
\multicolumn{9}{l}{\footnotesize \sym{*} \(p<0.10\), \sym{**} \(p<0.05\), \sym{***} \(p<0.01\)}\\
\end{tabular}
\end{table}
 %label {fefertility} % index models  41-48
}

% Fig 6(b) based on Models 49-56
% fertility rates and existential war
{\renewcommand\normalsize{\tiny}% adjust table size
	\normalsize
\begin{table}[htbp]\centering
\def\sym#1{\ifmmode^{#1}\else\(^{#1}\)\fi}
\caption{Fixed-effects models of the effect of existential war on future changes in fertility rates\label{fefertilityexistential}}
\begin{tabular}{l*{8}{c}}
\hline\hline
                    &\multicolumn{1}{c}{\shortstack{Model 49\\(current)}}&\multicolumn{1}{c}{\shortstack{Model 50\\(1-year)}}&\multicolumn{1}{c}{\shortstack{Model 51\\(2-year)}}&\multicolumn{1}{c}{\shortstack{Model 52\\(3-year)}}&\multicolumn{1}{c}{\shortstack{Model 53\\(4-year)}}&\multicolumn{1}{c}{\shortstack{Model 54\\(5-year)}}&\multicolumn{1}{c}{\shortstack{Model 55\\(10-year)}}&\multicolumn{1}{c}{\shortstack{Model 56\\(15-year)}}\\
\hline
Existential war    &     -0.0233\sym{**} &     -0.0470\sym{**} &     -0.0687\sym{**} &     -0.0885\sym{**} &      -0.104\sym{**} &      -0.120\sym{*}  &      -0.168         &      -0.159         \\
                    &   (0.00962)         &    (0.0193)         &    (0.0290)         &    (0.0394)         &    (0.0491)         &    (0.0610)         &     (0.127)         &     (0.160)         \\
[1em]
Fertility rate, lag      &     -0.0120\sym{***}&     -0.0319\sym{***}&     -0.0581\sym{***}&     -0.0895\sym{***}&      -0.126\sym{***}&      -0.166\sym{***}&      -0.407\sym{***}&      -0.669\sym{***}\\
                    &   (0.00242)         &   (0.00490)         &   (0.00743)         &    (0.0100)         &    (0.0127)         &    (0.0155)         &    (0.0280)         &    (0.0355)         \\
[1em]
$\Delta$Polity scores          &    -0.00164\sym{***}&    -0.00347\sym{***}&    -0.00479\sym{***}&    -0.00628\sym{***}&    -0.00754\sym{***}&    -0.00919\sym{***}&     -0.0116\sym{***}&     -0.0109\sym{***}\\
                    &  (0.000509)         &  (0.000919)         &   (0.00130)         &   (0.00165)         &   (0.00204)         &   (0.00227)         &   (0.00327)         &   (0.00367)         \\
[1em]
Polity scores, lag          &    -0.00101\sym{**} &    -0.00209\sym{**} &    -0.00326\sym{**} &    -0.00437\sym{**} &    -0.00545\sym{**} &    -0.00638\sym{**} &    -0.00827\sym{*}  &    -0.00756         \\
                    &  (0.000480)         &  (0.000946)         &   (0.00140)         &   (0.00185)         &   (0.00226)         &   (0.00266)         &   (0.00444)         &   (0.00570)         \\
[1em]
$\Delta$Energy.consump,log            &    -0.00121         &    -0.00201         &    -0.00484         &     -0.0134\sym{*}  &     -0.0160\sym{*}  &     -0.0191\sym{*}  &     -0.0223         &     -0.0282         \\
                    &   (0.00145)         &   (0.00281)         &   (0.00423)         &   (0.00768)         &   (0.00960)         &    (0.0115)         &    (0.0183)         &    (0.0197)         \\
[1em]
Energy.consump,log, lag           &     -0.0160\sym{***}&     -0.0317\sym{***}&     -0.0495\sym{***}&     -0.0651\sym{***}&     -0.0788\sym{***}&     -0.0913\sym{***}&      -0.135\sym{***}&      -0.150\sym{***}\\
                    &   (0.00216)         &   (0.00449)         &   (0.00741)         &    (0.0102)         &    (0.0128)         &    (0.0153)         &    (0.0252)         &    (0.0279)         \\
[1em]
Year                &    0.000533\sym{**} &    0.000694         &    0.000571         &    0.000101         &   -0.000639         &    -0.00167         &     -0.0101\sym{***}&     -0.0220\sym{***}\\
                    &  (0.000246)         &  (0.000507)         &  (0.000795)         &   (0.00111)         &   (0.00143)         &   (0.00178)         &   (0.00356)         &   (0.00493)         \\
[1em]
$\Delta$Neighboring fertility rates, lag&      0.0355\sym{***}&      0.0685\sym{***}&      0.0932\sym{***}&       0.118\sym{***}&       0.143\sym{***}&       0.164\sym{***}&       0.207\sym{***}&       0.188\sym{**} \\
                    &    (0.0117)         &    (0.0224)         &    (0.0307)         &    (0.0388)         &    (0.0480)         &    (0.0555)         &    (0.0747)         &    (0.0731)         \\
[1em]
Constant            &      -0.916\sym{*}  &      -1.060         &      -0.596         &       0.561         &       2.258         &       4.533         &       22.41\sym{***}&       47.14\sym{***}\\
                    &     (0.480)         &     (0.989)         &     (1.547)         &     (2.155)         &     (2.795)         &     (3.477)         &     (6.987)         &     (9.726)         \\
\hline
Observations        &        6677         &        6651         &        6550         &        6446         &        6316         &        6185         &        5530         &        4885         \\
\hline\hline
\multicolumn{9}{l}{\footnotesize Standard errors in parentheses}\\
\multicolumn{9}{l}{\footnotesize \sym{*} \(p<0.10\), \sym{**} \(p<0.05\), \sym{***} \(p<0.01\)}\\
\end{tabular}
\end{table}
 %label {fefertilityexistential} % index models  49-56
}

%Fig 7(a) based on model 57-61
% war on intermediate variable
{\renewcommand\normalsize{\tiny}% adjust table size
	\normalsize
\begin{table}[htbp]\centering
\def\sym#1{\ifmmode^{#1}\else\(^{#1}\)\fi}
\caption{Fixed-effects models of the effect of war on intermediate variables\label{FEintermed}}
\begin{tabular}{l*{5}{c}}
\hline\hline
                    &\multicolumn{1}{c}{\shortstack{Model 57\\($\Delta$ mil. per)}}&\multicolumn{1}{c}{\shortstack{Model 58\\($\Delta$ population)}}&\multicolumn{1}{c}{\shortstack{Model 59\\($\Delta$ male population)}}&\multicolumn{1}{c}{\shortstack{Model 60\\($\Delta$ female population)}}&\multicolumn{1}{c}{\shortstack{Model 61\\(irregular leader change)}}\\
\hline
War           &      0.0618\sym{***}&     -0.0261         &    -0.00788         &    -0.00645         &      0.0723\sym{***}\\
                    &    (0.0122)         &    (0.0204)         &   (0.00928)         &   (0.00860)         &    (0.0108)         \\
[1em]
Mil.per.pc,log, lag      &      -0.172\sym{***}&                     &                     &                     &                     \\
                    &    (0.0241)         &                     &                     &                     &                     \\
[1em]
$\Delta$Polity scores          &    -0.00132         &     0.00106         &    0.000398         &    0.000452         &    -0.00767\sym{**} \\
                    &   (0.00173)         &   (0.00188)         &  (0.000703)         &  (0.000640)         &   (0.00333)         \\
[1em]
Polity scores, lag           &    -0.00210\sym{***}&    0.000623         &   -0.000228         &  0.00000559         &   -0.000923         \\
                    &  (0.000578)         &   (0.00166)         &  (0.000769)         &  (0.000711)         &  (0.000782)         \\
[1em]
$\Delta$Energy.consump,log           &     0.00702\sym{**} &      0.0378\sym{***}&     0.00770\sym{***}&     0.00745\sym{***}&     -0.0103\sym{***}\\
                    &   (0.00347)         &    (0.0138)         &   (0.00260)         &   (0.00232)         &   (0.00350)         \\
[1em]
Energy.consump,log, lag           &     0.00482\sym{***}&      0.0201\sym{***}&     0.00934\sym{***}&      0.0106\sym{***}&    -0.00228         \\
                    &   (0.00149)         &   (0.00427)         &   (0.00228)         &   (0.00250)         &   (0.00237)         \\
[1em]
Year                &   -0.000373\sym{***}&   -0.000135         &   -0.000790         &   -0.000830         &   -0.000336\sym{**} \\
                    &  (0.000110)         &  (0.000364)         &  (0.000576)         &  (0.000520)         &  (0.000167)         \\
[1em]
Population,log, lag             &                     &     -0.0103\sym{***}&    -0.00658\sym{**} &    -0.00748\sym{**} &                     \\
                    &                     &   (0.00285)         &   (0.00330)         &   (0.00324)         &                     \\
[1em]
Constant            &       0.770\sym{***}&       1.199\sym{**} &       2.273\sym{**} &       2.419\sym{***}&       0.717\sym{**} \\
                    &     (0.211)         &     (0.563)         &     (0.893)         &     (0.793)         &     (0.317)         \\
\hline
Observations        &       10557         &       10871         &        7314         &        7314         &       10845         \\
\hline\hline
\multicolumn{6}{l}{\footnotesize Standard errors in parentheses}\\
\multicolumn{6}{l}{\footnotesize \sym{*} \(p<0.10\), \sym{**} \(p<0.05\), \sym{***} \(p<0.01\)}\\
\end{tabular}
\end{table}
 %label {FEintermed} % index models  57-61
}

%Fig 7(b-e) based on model 62-69
%  intermediate variables on women's empowerment
{\renewcommand\normalsize{\tiny}% adjust table size
	\normalsize
\begin{table}[htbp]\centering
\def\sym#1{\ifmmode^{#1}\else\(^{#1}\)\fi}
\caption{Fixed-effects models of the effect of intermediate variables on future changes in women's empowerment\label{intermpolempowerment}}
\begin{tabular}{l*{8}{c}}
\hline\hline
                    &\multicolumn{1}{c}{\shortstack{Model 62\\(current)}}&\multicolumn{1}{c}{\shortstack{Model 63\\(1-year)}}&\multicolumn{1}{c}{\shortstack{Model 64\\(2-year)}}&\multicolumn{1}{c}{\shortstack{Model 65\\(3-year)}}&\multicolumn{1}{c}{\shortstack{Model 66\\(4-year)}}&\multicolumn{1}{c}{\shortstack{Model 67\\(5-year)}}&\multicolumn{1}{c}{\shortstack{Model 68\\(10-year)}}&\multicolumn{1}{c}{\shortstack{Model 69\\(15-year)}}\\
\hline
War            &    -0.00199         &   -0.000557         &     0.00161         &     0.00354         &     0.00432         &     0.00474         &     0.00343         &     0.00243         \\
                    &   (0.00133)         &   (0.00209)         &   (0.00292)         &   (0.00352)         &   (0.00415)         &   (0.00482)         &   (0.00758)         &   (0.00876)         \\
[1em]
Political empowerment, lag    &     -0.0527\sym{***}&     -0.0957\sym{***}&      -0.140\sym{***}&      -0.189\sym{***}&      -0.234\sym{***}&      -0.282\sym{***}&      -0.502\sym{***}&      -0.672\sym{***}\\
                    &   (0.00509)         &   (0.00879)         &    (0.0121)         &    (0.0157)         &    (0.0192)         &    (0.0228)         &    (0.0393)         &    (0.0464)         \\
[1em]
$\Delta$Mil.per.pc,log      &    -0.00298         &    -0.00117         & -0.00000996         &     0.00147         &     0.00614         &     0.00834         &      0.0110         &     0.00375         \\
                    &   (0.00264)         &   (0.00348)         &   (0.00395)         &   (0.00438)         &   (0.00479)         &   (0.00601)         &   (0.00802)         &    (0.0103)         \\
[1em]
Mil.per.pc,log, lag     &     0.00515\sym{***}&     0.00815\sym{***}&      0.0112\sym{**} &      0.0136\sym{**} &      0.0149\sym{**} &      0.0153\sym{*}  &      0.0103         &     -0.0107         \\
                    &   (0.00166)         &   (0.00293)         &   (0.00441)         &   (0.00588)         &   (0.00722)         &   (0.00837)         &    (0.0128)         &    (0.0137)         \\
[1em]
$\Delta$Population,log             &    -0.00125         &   -0.000608         &    0.000203         &    -0.00255         &    -0.00307         &    -0.00513\sym{*}  &     -0.0111\sym{***}&     -0.0103\sym{***}\\
                    &   (0.00102)         &   (0.00171)         &   (0.00189)         &   (0.00244)         &   (0.00257)         &   (0.00270)         &   (0.00346)         &   (0.00319)         \\
[1em]
Population,log, lag               &   0.0000380         &    0.000259         &    0.000465         &    0.000654         &    0.000879         &     0.00108         &     0.00228         &     0.00372         \\
                    &  (0.000174)         &  (0.000328)         &  (0.000494)         &  (0.000669)         &  (0.000837)         &   (0.00101)         &   (0.00178)         &   (0.00239)         \\
[1em]
Irregular.leader.change    &    0.000147         &     0.00968\sym{**} &     0.00932\sym{*}  &      0.0118\sym{**} &      0.0123\sym{**} &      0.0107\sym{**} &     0.00738         &    0.000962         \\
                    &   (0.00284)         &   (0.00446)         &   (0.00488)         &   (0.00484)         &   (0.00510)         &   (0.00534)         &   (0.00608)         &   (0.00563)         \\
[1em]
$\Delta$Polity scores            &     0.00323\sym{***}&     0.00570\sym{***}&     0.00594\sym{***}&     0.00552\sym{***}&     0.00524\sym{***}&     0.00500\sym{***}&     0.00368\sym{***}&     0.00334\sym{***}\\
                    &  (0.000459)         &  (0.000780)         &  (0.000890)         &  (0.000881)         &  (0.000957)         &  (0.000929)         &  (0.000807)         &  (0.000841)         \\
[1em]
Polity scores, lag            &    0.000776\sym{***}&     0.00103\sym{***}&     0.00119\sym{***}&     0.00136\sym{***}&     0.00148\sym{***}&     0.00166\sym{***}&     0.00195\sym{**} &    0.000904         \\
                    &  (0.000101)         &  (0.000174)         &  (0.000242)         &  (0.000314)         &  (0.000381)         &  (0.000455)         &  (0.000798)         &  (0.000968)         \\
[1em]
$\Delta$Energy.consump,log             &    0.000265         &   -0.000188         &   -0.000554         &    -0.00113         &    -0.00162         &    -0.00345         &    -0.00475         &    -0.00563         \\
                    &  (0.000862)         &  (0.000935)         &   (0.00113)         &   (0.00137)         &   (0.00163)         &   (0.00225)         &   (0.00314)         &   (0.00388)         \\
[1em]
Energy.consump,log, lag             &    0.000117         &    0.000184         &    0.000312         &    0.000568         &    0.000893         &     0.00119         &     0.00248         &     0.00479         \\
                    &  (0.000270)         &  (0.000510)         &  (0.000761)         &   (0.00102)         &   (0.00126)         &   (0.00149)         &   (0.00254)         &   (0.00323)         \\
[1em]
Year                &    0.000258\sym{***}&    0.000495\sym{***}&    0.000746\sym{***}&     0.00101\sym{***}&     0.00126\sym{***}&     0.00151\sym{***}&     0.00267\sym{***}&     0.00350\sym{***}\\
                    & (0.0000352)         & (0.0000655)         & (0.0000963)         &  (0.000129)         &  (0.000160)         &  (0.000192)         &  (0.000355)         &  (0.000496)         \\
[1em]
$\Delta$Neighboring empowerment, lag&     0.00576         &      0.0194\sym{***}&      0.0240\sym{***}&      0.0260\sym{***}&      0.0199\sym{**} &      0.0258\sym{***}&      0.0207         &      0.0289\sym{*}  \\
                    &   (0.00363)         &   (0.00584)         &   (0.00863)         &   (0.00857)         &   (0.00816)         &   (0.00888)         &    (0.0150)         &    (0.0159)         \\
[1em]
Constant            &      -0.483\sym{***}&      -0.947\sym{***}&      -1.436\sym{***}&      -1.953\sym{***}&      -2.432\sym{***}&      -2.918\sym{***}&      -5.176\sym{***}&      -6.833\sym{***}\\
                    &    (0.0610)         &     (0.113)         &     (0.165)         &     (0.220)         &     (0.271)         &     (0.324)         &     (0.595)         &     (0.831)         \\
\hline
Observations        &        7860         &        7662         &        7486         &        7309         &        7146         &        6996         &        6351         &        5857         \\
\hline\hline
\multicolumn{9}{l}{\footnotesize Standard errors in parentheses}\\
\multicolumn{9}{l}{\footnotesize \sym{*} \(p<0.10\), \sym{**} \(p<0.05\), \sym{***} \(p<0.01\)}\\
\end{tabular}
\end{table}
 %label {intermpolempowerment} % index models  62-69
}



\section{Supplementary Tables for Main Results and Figures}
\vspace*{.2in}
\setlength{\parskip}{-2em}
%\setlength{\parindent}{1cm}
\doublespacing
\setcounter{table}{0}
\setcounter{figure}{0}
\renewcommand{\thetable}{B\arabic{table}}	
\renewcommand{\thefigure}{B\arabic{figure}}	

Tables \ref{fyearonyearpolempower}- \ref{interaction} present additional model specifications to our core models as discussed in the main text.  \\


% Table B1
% Future year-on-year changes in empowerment
{\renewcommand\normalsize{\tiny}% adjust table size
	\normalsize
\begin{table}[htbp]\centering
\def\sym#1{\ifmmode^{#1}\else\(^{#1}\)\fi}
\caption{Fixed-effects models of the effect of war on future year-on-year changes in women's empowerment\label{fyearonyearpolempower}}
\begin{tabular}{l*{8}{c}}
\hline\hline
                    &\multicolumn{1}{c}{\shortstack{Model 70\\(current)}}&\multicolumn{1}{c}{\shortstack{Model 71\\(1-year)}}&\multicolumn{1}{c}{\shortstack{Model 72\\(2-year)}}&\multicolumn{1}{c}{\shortstack{Model 73\\(3-year)}}&\multicolumn{1}{c}{\shortstack{Model 74\\(4-year)}}&\multicolumn{1}{c}{\shortstack{Model 75\\(5-year)}}&\multicolumn{1}{c}{\shortstack{Model 76\\(10-year)}}&\multicolumn{1}{c}{\shortstack{Model 77\\(15-year)}}\\
\hline
War          &    -0.00155         &     0.00137         &     0.00280\sym{**} &     0.00233\sym{*}  &     0.00183         &     0.00130         &   -0.000986         &    0.000335         \\
                    &   (0.00124)         &   (0.00122)         &   (0.00119)         &   (0.00119)         &   (0.00120)         &   (0.00115)         &  (0.000955)         &   (0.00119)         \\
[1em]
Political empowerment, lag   &     -0.0519\sym{***}&     -0.0425\sym{***}&     -0.0387\sym{***}&     -0.0385\sym{***}&     -0.0392\sym{***}&     -0.0415\sym{***}&     -0.0353\sym{***}&     -0.0258\sym{***}\\
                    &   (0.00484)         &   (0.00458)         &   (0.00436)         &   (0.00411)         &   (0.00415)         &   (0.00425)         &   (0.00371)         &   (0.00423)         \\
[1em]
$\Delta$Polity scores        &     0.00318\sym{***}&     0.00262\sym{***}&    0.000363         &   -0.000130         &   -0.000323\sym{*}  &   -0.000213         &  -0.0000771         &   -0.000170         \\
                    &  (0.000434)         &  (0.000476)         &  (0.000263)         &  (0.000166)         &  (0.000169)         &  (0.000174)         &  (0.000188)         &  (0.000117)         \\
[1em]
Polity scores, lag           &    0.000708\sym{***}&    0.000244\sym{***}& -0.00000690         &  -0.0000459         &  -0.0000530         &   0.0000270         &   0.0000137         &   -0.000169\sym{*}  \\
                    & (0.0000937)         & (0.0000854)         & (0.0000844)         & (0.0000832)         & (0.0000840)         & (0.0000925)         & (0.0000825)         &  (0.000101)         \\
[1em]
$\Delta$Energy.consump,log            &    0.000444         &   -0.000490         &  -0.0000678         &   -0.000891\sym{*}  &   -0.000889         &    -0.00133         &    0.000543         &   -0.000652         \\
                    &  (0.000834)         &  (0.000449)         &  (0.000689)         &  (0.000455)         &  (0.000566)         &  (0.000994)         &  (0.000858)         &  (0.000644)         \\
[1em]
Energy.consump,log, lag            &    0.000286         &    0.000174         &    0.000390\sym{*}  &    0.000475\sym{**} &    0.000503\sym{**} &    0.000469\sym{**} &    0.000826\sym{***}&    0.000823\sym{***}\\
                    &  (0.000205)         &  (0.000209)         &  (0.000225)         &  (0.000223)         &  (0.000214)         &  (0.000207)         &  (0.000216)         &  (0.000276)         \\
[1em]
Year                &    0.000253\sym{***}&    0.000252\sym{***}&    0.000242\sym{***}&    0.000237\sym{***}&    0.000238\sym{***}&    0.000241\sym{***}&    0.000165\sym{***}&    0.000112\sym{***}\\
                    & (0.0000295)         & (0.0000295)         & (0.0000298)         & (0.0000285)         & (0.0000284)         & (0.0000279)         & (0.0000282)         & (0.0000301)         \\
[1em]
$\Delta$Neighboring empowerment, lag&     0.00606\sym{*}  &      0.0106\sym{**} &     0.00781         &     0.00320         &     0.00114         &     0.00505         &    -0.00151         &    -0.00120         \\
                    &   (0.00361)         &   (0.00436)         &   (0.00498)         &   (0.00526)         &   (0.00424)         &   (0.00512)         &   (0.00549)         &   (0.00505)         \\
[1em]
Constant            &      -0.471\sym{***}&      -0.473\sym{***}&      -0.457\sym{***}&      -0.448\sym{***}&      -0.447\sym{***}&      -0.454\sym{***}&      -0.309\sym{***}&      -0.208\sym{***}\\
                    &    (0.0552)         &    (0.0554)         &    (0.0561)         &    (0.0536)         &    (0.0535)         &    (0.0526)         &    (0.0529)         &    (0.0562)         \\
\hline
Observations        &        8090         &        7844         &        7628         &        7430         &        7252         &        7093         &        6406         &        5875         \\
\hline\hline
\multicolumn{9}{l}{\footnotesize Standard errors in parentheses}\\
\multicolumn{9}{l}{\footnotesize \sym{*} \(p<0.10\), \sym{**} \(p<0.05\), \sym{***} \(p<0.01\)}\\
\end{tabular}
\end{table}
 %label {fyearonyearpolempower} % index models  70-77
}

%Table B2
%cumulative death
{\renewcommand\normalsize{\tiny}% adjust table size
	\normalsize
\begin{table}[htbp]\centering
\def\sym#1{\ifmmode^{#1}\else\(^{#1}\)\fi}
\caption{Fixed-effects models of the effect of cumulative battle deaths on future changes in women's empowerment\label{culmudeaths}}
\begin{tabular}{l*{8}{c}}
\hline\hline
                    &\multicolumn{1}{c}{\shortstack{Model 78\\(current)}}&\multicolumn{1}{c}{\shortstack{Model 79\\(1-year)}}&\multicolumn{1}{c}{\shortstack{Model 80\\(2-year)}}&\multicolumn{1}{c}{\shortstack{Model 81\\(3-year)}}&\multicolumn{1}{c}{\shortstack{Model 82\\(4-year)}}&\multicolumn{1}{c}{\shortstack{Model 83\\(5-year)}}&\multicolumn{1}{c}{\shortstack{Model 84\\(10-year)}}&\multicolumn{1}{c}{\shortstack{Model 85\\(15-year)}}\\
\hline
Cumulative battle deaths, log    &   -0.000433\sym{**} &   -0.000517         &   -0.000534         &   -0.000496         &   -0.000366         &   -0.000567         &    0.000113         &     0.00173         \\
                    &  (0.000202)         &  (0.000323)         &  (0.000460)         &  (0.000622)         &  (0.000799)         &  (0.000976)         &   (0.00109)         &   (0.00121)         \\
[1em]
Political empowerment, lag  &     -0.0727\sym{***}&      -0.131\sym{***}&      -0.189\sym{***}&      -0.246\sym{***}&      -0.298\sym{***}&      -0.356\sym{***}&      -0.613\sym{***}&      -0.806\sym{***}\\
                    &   (0.00687)         &    (0.0110)         &    (0.0155)         &    (0.0197)         &    (0.0234)         &    (0.0274)         &    (0.0419)         &    (0.0450)         \\
[1em]
$\Delta$Polity scores           &     0.00372\sym{***}&     0.00649\sym{***}&     0.00656\sym{***}&     0.00596\sym{***}&     0.00548\sym{***}&     0.00490\sym{***}&     0.00375\sym{***}&     0.00357\sym{***}\\
                    &  (0.000596)         &  (0.000949)         &   (0.00105)         &  (0.000996)         &  (0.000983)         &  (0.000935)         &  (0.000777)         &  (0.000685)         \\
[1em]
Polity scores, lag            &     0.00106\sym{***}&     0.00145\sym{***}&     0.00167\sym{***}&     0.00184\sym{***}&     0.00195\sym{***}&     0.00217\sym{***}&     0.00324\sym{***}&     0.00289\sym{***}\\
                    &  (0.000133)         &  (0.000216)         &  (0.000292)         &  (0.000359)         &  (0.000419)         &  (0.000488)         &  (0.000841)         &  (0.000948)         \\
[1em]
$\Delta$Energy.consump,log            &    0.000259         &   -0.000275         &   -0.000662         &   -0.000913         &   -0.000833         &    -0.00129         &    -0.00172         &    -0.00268         \\
                    &  (0.000733)         &  (0.000900)         &   (0.00124)         &   (0.00148)         &   (0.00174)         &   (0.00247)         &   (0.00295)         &   (0.00371)         \\
[1em]
Energy.consump,log, lag              &    0.000280         &    0.000467         &    0.000805         &     0.00138         &     0.00215         &     0.00306\sym{*}  &     0.00539\sym{**} &     0.00735\sym{**} \\
                    &  (0.000292)         &  (0.000531)         &  (0.000805)         &   (0.00108)         &   (0.00132)         &   (0.00157)         &   (0.00268)         &   (0.00355)         \\
[1em]
Year                &    0.000363\sym{***}&    0.000736\sym{***}&     0.00111\sym{***}&     0.00146\sym{***}&     0.00177\sym{***}&     0.00211\sym{***}&     0.00376\sym{***}&     0.00511\sym{***}\\
                    & (0.0000517)         & (0.0000934)         &  (0.000142)         &  (0.000187)         &  (0.000232)         &  (0.000282)         &  (0.000486)         &  (0.000692)         \\
[1em]
$\Delta$Neighboring empowerment, lag&     0.00913\sym{**} &      0.0210\sym{***}&      0.0256\sym{***}&      0.0267\sym{***}&      0.0218\sym{**} &      0.0294\sym{***}&      0.0239\sym{*}  &      0.0365\sym{**} \\
                    &   (0.00421)         &   (0.00658)         &   (0.00950)         &   (0.00996)         &   (0.00999)         &    (0.0107)         &    (0.0137)         &    (0.0141)         \\
[1em]
Constant            &      -0.676\sym{***}&      -1.380\sym{***}&      -2.084\sym{***}&      -2.746\sym{***}&      -3.340\sym{***}&      -3.976\sym{***}&      -7.090\sym{***}&      -9.651\sym{***}\\
                    &    (0.0984)         &     (0.178)         &     (0.270)         &     (0.357)         &     (0.443)         &     (0.539)         &     (0.933)         &     (1.332)         \\
\hline
Observations        &        5928         &        5842         &        5787         &        5741         &        5717         &        5640         &        5090         &        4559         \\
\hline\hline
\multicolumn{9}{l}{\footnotesize Standard errors in parentheses}\\
\multicolumn{9}{l}{\footnotesize \sym{*} \(p<0.10\), \sym{**} \(p<0.05\), \sym{***} \(p<0.01\)}\\
\end{tabular}
\end{table}
 %label {culmudeaths} % index models  78-85
}

% Table B3
%New Intrastate war
{\renewcommand\normalsize{\tiny}% adjust table size
	\normalsize
\begin{table}[htbp]\centering
\def\sym#1{\ifmmode^{#1}\else\(^{#1}\)\fi}
\caption{Fixed-effects models of the effect of intrastate war on future changes in women's empowerment\label{intrawarpolempower}}
\begin{tabular}{l*{8}{c}}
\hline\hline
                    &\multicolumn{1}{c}{\shortstack{Model 86\\(current)}}&\multicolumn{1}{c}{\shortstack{Model 87\\(1-year)}}&\multicolumn{1}{c}{\shortstack{Model 88\\(2-year)}}&\multicolumn{1}{c}{\shortstack{Model 89\\(3-year)}}&\multicolumn{1}{c}{\shortstack{Model 90\\(4-year)}}&\multicolumn{1}{c}{\shortstack{Model 91\\(5-year)}}&\multicolumn{1}{c}{\shortstack{Model 92\\(10-year)}}&\multicolumn{1}{c}{\shortstack{Model 93\\(15-year)}}\\
\hline
New Intrastate war            &    -0.00164         &   -0.000175         &     0.00146         &     0.00227         &     0.00287         &     0.00221         &    -0.00210         &     0.00565         \\
                    &   (0.00285)         &   (0.00520)         &   (0.00704)         &   (0.00816)         &   (0.00809)         &   (0.00827)         &    (0.0116)         &    (0.0147)         \\
[1em]
Ongoing intrastate war         &    -0.00136         &     0.00164         &     0.00347         &     0.00698         &     0.00927         &     0.00924         &      0.0106         &      0.0125         \\
                    &   (0.00212)         &   (0.00352)         &   (0.00476)         &   (0.00590)         &   (0.00732)         &   (0.00867)         &    (0.0132)         &    (0.0156)         \\
[1em]
Recent Intrastate war           &     0.00864\sym{**} &      0.0103\sym{*}  &      0.0115\sym{*}  &     0.00987         &      0.0103         &      0.0103         &     0.00593         &     0.00722         \\
                    &   (0.00429)         &   (0.00609)         &   (0.00669)         &   (0.00716)         &   (0.00799)         &   (0.00834)         &    (0.0109)         &    (0.0130)         \\
[1em]
Political empowerment, lag  &     -0.0513\sym{***}&     -0.0943\sym{***}&      -0.140\sym{***}&      -0.190\sym{***}&      -0.237\sym{***}&      -0.288\sym{***}&      -0.510\sym{***}&      -0.674\sym{***}\\
                    &   (0.00488)         &   (0.00843)         &    (0.0118)         &    (0.0155)         &    (0.0190)         &    (0.0227)         &    (0.0374)         &    (0.0440)         \\
[1em]
$\Delta$Polity scores            &     0.00317\sym{***}&     0.00570\sym{***}&     0.00594\sym{***}&     0.00538\sym{***}&     0.00507\sym{***}&     0.00480\sym{***}&     0.00354\sym{***}&     0.00354\sym{***}\\
                    &  (0.000434)         &  (0.000758)         &  (0.000871)         &  (0.000873)         &  (0.000936)         &  (0.000902)         &  (0.000797)         &  (0.000842)         \\
[1em]
Polity scores, lag         &    0.000697\sym{***}&    0.000915\sym{***}&     0.00103\sym{***}&     0.00117\sym{***}&     0.00129\sym{***}&     0.00150\sym{***}&     0.00194\sym{**} &     0.00126         \\
                    & (0.0000946)         &  (0.000163)         &  (0.000230)         &  (0.000297)         &  (0.000359)         &  (0.000431)         &  (0.000770)         &  (0.000943)         \\
[1em]
$\Delta$Energy.consump,log             &    0.000459         &  -0.0000643         &   -0.000217         &   -0.000963         &    -0.00149         &    -0.00344         &    -0.00499         &    -0.00482         \\
                    &  (0.000841)         &  (0.000891)         &   (0.00108)         &   (0.00133)         &   (0.00159)         &   (0.00220)         &   (0.00314)         &   (0.00378)         \\
[1em]
Energy.consump,log, lag            &    0.000275         &    0.000540         &    0.000928         &     0.00137\sym{*}  &     0.00183\sym{*}  &     0.00228\sym{**} &     0.00477\sym{**} &     0.00790\sym{***}\\
                    &  (0.000204)         &  (0.000389)         &  (0.000584)         &  (0.000790)         &  (0.000974)         &   (0.00115)         &   (0.00203)         &   (0.00281)         \\
[1em]
Year                &    0.000253\sym{***}&    0.000505\sym{***}&    0.000776\sym{***}&     0.00106\sym{***}&     0.00134\sym{***}&     0.00163\sym{***}&     0.00290\sym{***}&     0.00383\sym{***}\\
                    & (0.0000296)         & (0.0000553)         & (0.0000823)         &  (0.000111)         &  (0.000136)         &  (0.000161)         &  (0.000283)         &  (0.000379)         \\
[1em]
$\Delta$Neighboring  empowerment, lag&     0.00606\sym{*}  &      0.0190\sym{***}&      0.0226\sym{***}&      0.0238\sym{***}&      0.0189\sym{**} &      0.0223\sym{**} &      0.0143         &      0.0248\sym{*}  \\
                    &   (0.00359)         &   (0.00573)         &   (0.00841)         &   (0.00851)         &   (0.00827)         &   (0.00870)         &    (0.0139)         &    (0.0132)         \\
[1em]
Constant            &      -0.471\sym{***}&      -0.945\sym{***}&      -1.454\sym{***}&      -1.994\sym{***}&      -2.509\sym{***}&      -3.048\sym{***}&      -5.440\sym{***}&      -7.185\sym{***}\\
                    &    (0.0554)         &     (0.104)         &     (0.155)         &     (0.208)         &     (0.256)         &     (0.304)         &     (0.533)         &     (0.716)         \\
\hline
Observations        &        8090         &        7889         &        7708         &        7527         &        7362         &        7209         &        6538         &        6009         \\
\hline\hline
\multicolumn{9}{l}{\footnotesize Standard errors in parentheses}\\
\multicolumn{9}{l}{\footnotesize \sym{*} \(p<0.10\), \sym{**} \(p<0.05\), \sym{***} \(p<0.01\)}\\
\end{tabular}
\end{table}
 %label {intrawarpolempower} % index models  86-93
}

% Table B4
%New Interstate war
{\renewcommand\normalsize{\tiny}% adjust table size
	\normalsize
\begin{table}[htbp]\centering
\def\sym#1{\ifmmode^{#1}\else\(^{#1}\)\fi}
\caption{Fixed-effects models of the effect of interstate war on future changes in women's empowerment\label{interwarpolempower}}
\begin{tabular}{l*{8}{c}}
\hline\hline
                    &\multicolumn{1}{c}{\shortstack{Model 94\\(current)}}&\multicolumn{1}{c}{\shortstack{Model 95\\(1-year)}}&\multicolumn{1}{c}{\shortstack{Model 96\\(2-year)}}&\multicolumn{1}{c}{\shortstack{Model 97\\(3-year)}}&\multicolumn{1}{c}{\shortstack{Model 98\\(4-year)}}&\multicolumn{1}{c}{\shortstack{Model 99\\(5-year)}}&\multicolumn{1}{c}{\shortstack{Model 100\\(10-year)}}&\multicolumn{1}{c}{\shortstack{Model 101\\(15-year)}}\\
\hline
New interstate war          &    -0.00179         &    -0.00185         &    -0.00164         &    -0.00389         &    -0.00267         &    0.000339         &     0.00324         &   -0.000947         \\
                    &   (0.00257)         &   (0.00279)         &   (0.00329)         &   (0.00434)         &   (0.00467)         &   (0.00504)         &   (0.00713)         &   (0.00770)         \\
[1em]
Ongoing interstate war          &    0.000672         &     0.00225         &     0.00679         &      0.0122\sym{**} &      0.0142\sym{*}  &      0.0136         &     0.00995         &    -0.00216         \\
                    &   (0.00154)         &   (0.00271)         &   (0.00417)         &   (0.00603)         &   (0.00783)         &   (0.00926)         &    (0.0142)         &    (0.0133)         \\
[1em]
Recent interstate war       &     0.00460\sym{***}&     0.00602\sym{**} &     0.00641\sym{*}  &     0.00289         &     0.00267         &     0.00329         &    -0.00158         &    -0.00343         \\
                    &   (0.00166)         &   (0.00234)         &   (0.00371)         &   (0.00409)         &   (0.00404)         &   (0.00462)         &   (0.00697)         &   (0.00719)         \\
[1em]
Political empowerment, lag  &     -0.0511\sym{***}&     -0.0949\sym{***}&      -0.140\sym{***}&      -0.190\sym{***}&      -0.238\sym{***}&      -0.288\sym{***}&      -0.511\sym{***}&      -0.679\sym{***}\\
                    &   (0.00483)         &   (0.00852)         &    (0.0120)         &    (0.0157)         &    (0.0193)         &    (0.0231)         &    (0.0391)         &    (0.0446)         \\
[1em]
$\Delta$Polity scores              &     0.00319\sym{***}&     0.00572\sym{***}&     0.00596\sym{***}&     0.00537\sym{***}&     0.00505\sym{***}&     0.00479\sym{***}&     0.00355\sym{***}&     0.00355\sym{***}\\
                    &  (0.000433)         &  (0.000757)         &  (0.000868)         &  (0.000866)         &  (0.000918)         &  (0.000887)         &  (0.000789)         &  (0.000833)         \\
[1em]
Polity scores, lag           &    0.000701\sym{***}&    0.000930\sym{***}&     0.00104\sym{***}&     0.00118\sym{***}&     0.00130\sym{***}&     0.00153\sym{***}&     0.00198\sym{**} &     0.00133         \\
                    & (0.0000936)         &  (0.000164)         &  (0.000232)         &  (0.000302)         &  (0.000367)         &  (0.000441)         &  (0.000793)         &  (0.000951)         \\
[1em]
$\Delta$Energy.consump,log              &    0.000420         &   -0.000124         &   -0.000307         &    -0.00101         &    -0.00154         &    -0.00331         &    -0.00479         &    -0.00494         \\
                    &  (0.000827)         &  (0.000893)         &   (0.00110)         &   (0.00135)         &   (0.00158)         &   (0.00218)         &   (0.00310)         &   (0.00378)         \\
[1em]
Energy.consump,log, lag            &    0.000272         &    0.000562         &    0.000961         &     0.00141\sym{*}  &     0.00189\sym{*}  &     0.00236\sym{**} &     0.00486\sym{**} &     0.00804\sym{***}\\
                    &  (0.000204)         &  (0.000388)         &  (0.000587)         &  (0.000798)         &  (0.000987)         &   (0.00117)         &   (0.00206)         &   (0.00288)         \\
[1em]
Year                &    0.000252\sym{***}&    0.000508\sym{***}&    0.000781\sym{***}&     0.00107\sym{***}&     0.00135\sym{***}&     0.00164\sym{***}&     0.00291\sym{***}&     0.00385\sym{***}\\
                    & (0.0000294)         & (0.0000554)         & (0.0000822)         &  (0.000110)         &  (0.000136)         &  (0.000161)         &  (0.000284)         &  (0.000377)         \\
[1em]
$\Delta$Neighboring empowerment, lag&     0.00602\sym{*}  &      0.0189\sym{***}&      0.0228\sym{***}&      0.0239\sym{***}&      0.0194\sym{**} &      0.0228\sym{***}&      0.0147         &      0.0251\sym{*}  \\
                    &   (0.00361)         &   (0.00574)         &   (0.00860)         &   (0.00868)         &   (0.00842)         &   (0.00872)         &    (0.0140)         &    (0.0133)         \\
[1em]
Constant            &      -0.469\sym{***}&      -0.951\sym{***}&      -1.463\sym{***}&      -2.011\sym{***}&      -2.530\sym{***}&      -3.068\sym{***}&      -5.459\sym{***}&      -7.213\sym{***}\\
                    &    (0.0551)         &     (0.104)         &     (0.154)         &     (0.208)         &     (0.255)         &     (0.303)         &     (0.535)         &     (0.712)         \\
\hline
Observations        &        8090         &        7889         &        7708         &        7527         &        7362         &        7209         &        6538         &        6009         \\
\hline\hline
\multicolumn{9}{l}{\footnotesize Standard errors in parentheses}\\
\multicolumn{9}{l}{\footnotesize \sym{*} \(p<0.10\), \sym{**} \(p<0.05\), \sym{***} \(p<0.01\)}\\
\end{tabular}
\end{table}
 %label {interwarpolempower} % index models  94-101
}

%Table B5
%new existential war
{\renewcommand\normalsize{\tiny}% adjust table size
	\normalsize
\begin{table}[htbp]\centering
\def\sym#1{\ifmmode^{#1}\else\(^{#1}\)\fi}
\caption{Fixed-effects models of the effect of existential war on future changes in women's empowerment\label{existentialwarpolempower}}
\begin{tabular}{l*{8}{c}}
\hline\hline
                    &\multicolumn{1}{c}{\shortstack{Model 102\\(current)}}&\multicolumn{1}{c}{\shortstack{Model 103\\(1-year)}}&\multicolumn{1}{c}{\shortstack{Model 104\\(2-year)}}&\multicolumn{1}{c}{\shortstack{Model 105\\(3-year)}}&\multicolumn{1}{c}{\shortstack{Model 106\\(4-year)}}&\multicolumn{1}{c}{\shortstack{Model 107\\(5-year)}}&\multicolumn{1}{c}{\shortstack{Model 108\\(10-year)}}&\multicolumn{1}{c}{\shortstack{Model 109\\(15-year)}}\\
\hline
New existential war            &   -0.000103         &    -0.00470         &    -0.00424         &    -0.00317         &    -0.00441         &    -0.00230         &     0.00552         &    -0.00343         \\
                    &   (0.00291)         &   (0.00406)         &   (0.00448)         &   (0.00485)         &   (0.00523)         &   (0.00565)         &   (0.00786)         &   (0.00840)         \\
[1em]
Ongoing existential war         &   -0.000254         &    0.000204         &     0.00402         &     0.00769         &     0.00970         &      0.0110         &     0.00741         &    -0.00378         \\
                    &   (0.00151)         &   (0.00254)         &   (0.00368)         &   (0.00527)         &   (0.00702)         &   (0.00845)         &    (0.0124)         &    (0.0112)         \\
[1em]
Recent existential war       &     0.00619\sym{**} &     0.00838\sym{**} &     0.00867\sym{**} &     0.00604         &     0.00744         &     0.00862         &     0.00246         &    -0.00286         \\
                    &   (0.00268)         &   (0.00325)         &   (0.00432)         &   (0.00487)         &   (0.00473)         &   (0.00552)         &   (0.00781)         &   (0.00729)         \\
[1em]
Political empowerment, lag   &     -0.0512\sym{***}&     -0.0952\sym{***}&      -0.141\sym{***}&      -0.191\sym{***}&      -0.239\sym{***}&      -0.289\sym{***}&      -0.511\sym{***}&      -0.679\sym{***}\\
                    &   (0.00480)         &   (0.00849)         &    (0.0120)         &    (0.0157)         &    (0.0193)         &    (0.0231)         &    (0.0390)         &    (0.0445)         \\
[1em]
$\Delta$Polity scores            &     0.00319\sym{***}&     0.00573\sym{***}&     0.00597\sym{***}&     0.00539\sym{***}&     0.00508\sym{***}&     0.00481\sym{***}&     0.00355\sym{***}&     0.00355\sym{***}\\
                    &  (0.000433)         &  (0.000758)         &  (0.000869)         &  (0.000869)         &  (0.000921)         &  (0.000888)         &  (0.000789)         &  (0.000833)         \\
[1em]
Polity scores, lag            &    0.000700\sym{***}&    0.000932\sym{***}&     0.00104\sym{***}&     0.00119\sym{***}&     0.00131\sym{***}&     0.00152\sym{***}&     0.00198\sym{**} &     0.00134         \\
                    & (0.0000933)         &  (0.000163)         &  (0.000231)         &  (0.000302)         &  (0.000367)         &  (0.000441)         &  (0.000794)         &  (0.000952)         \\
[1em]
$\Delta$Energy.consump,log            &    0.000409         &   -0.000195         &   -0.000383         &    -0.00108         &    -0.00167         &    -0.00347         &    -0.00480         &    -0.00499         \\
                    &  (0.000825)         &  (0.000892)         &   (0.00110)         &   (0.00134)         &   (0.00158)         &   (0.00217)         &   (0.00311)         &   (0.00378)         \\
[1em]
Energy.consump,log, lag            &    0.000270         &    0.000564         &    0.000964         &     0.00143\sym{*}  &     0.00191\sym{*}  &     0.00237\sym{**} &     0.00486\sym{**} &     0.00804\sym{***}\\
                    &  (0.000205)         &  (0.000389)         &  (0.000588)         &  (0.000799)         &  (0.000988)         &   (0.00116)         &   (0.00206)         &   (0.00288)         \\
[1em]
Year                &    0.000253\sym{***}&    0.000508\sym{***}&    0.000782\sym{***}&     0.00107\sym{***}&     0.00135\sym{***}&     0.00164\sym{***}&     0.00291\sym{***}&     0.00385\sym{***}\\
                    & (0.0000296)         & (0.0000556)         & (0.0000824)         &  (0.000111)         &  (0.000136)         &  (0.000161)         &  (0.000283)         &  (0.000377)         \\
[1em]
$\Delta$Neighboring empowerment, lag&     0.00605\sym{*}  &      0.0189\sym{***}&      0.0228\sym{***}&      0.0242\sym{***}&      0.0195\sym{**} &      0.0230\sym{***}&      0.0148         &      0.0251\sym{*}  \\
                    &   (0.00361)         &   (0.00575)         &   (0.00860)         &   (0.00871)         &   (0.00844)         &   (0.00874)         &    (0.0140)         &    (0.0133)         \\
[1em]
Constant            &      -0.470\sym{***}&      -0.951\sym{***}&      -1.465\sym{***}&      -2.014\sym{***}&      -2.534\sym{***}&      -3.073\sym{***}&      -5.464\sym{***}&      -7.213\sym{***}\\
                    &    (0.0555)         &     (0.104)         &     (0.155)         &     (0.208)         &     (0.256)         &     (0.304)         &     (0.534)         &     (0.711)         \\
\hline
Observations        &        8090         &        7889         &        7708         &        7527         &        7362         &        7209         &        6538         &        6009         \\
\hline\hline
\multicolumn{9}{l}{\footnotesize Standard errors in parentheses}\\
\multicolumn{9}{l}{\footnotesize \sym{*} \(p<0.10\), \sym{**} \(p<0.05\), \sym{***} \(p<0.01\)}\\
\end{tabular}
\end{table}
 %label {existentialwarpolempower} % index models  102-109
}

%Table B6
% civil society participation
{\renewcommand\normalsize{\tiny}% adjust table size
	\normalsize
\begin{table}[htbp]\centering
\def\sym#1{\ifmmode^{#1}\else\(^{#1}\)\fi}
\caption{Fixed-effects models of the effect of types of war on future changes in civil society participation \label{fecivilparticip}}
\begin{tabular}{l*{8}{c}}
\hline\hline
                    &\multicolumn{1}{c}{\shortstack{Model 110\\(current)}}&\multicolumn{1}{c}{\shortstack{Model 111\\(1-year)}}&\multicolumn{1}{c}{\shortstack{Model 112\\(2-year)}}&\multicolumn{1}{c}{\shortstack{Model 113\\(3-year)}}&\multicolumn{1}{c}{\shortstack{Model 114\\(4-year)}}&\multicolumn{1}{c}{\shortstack{Model 115\\(5-year)}}&\multicolumn{1}{c}{\shortstack{Model 116\\(10-year)}}&\multicolumn{1}{c}{\shortstack{Model 117\\(15-year)}}\\
\hline
New war                &     0.00228         &     0.00147         &     0.00289         &    -0.00186         &    -0.00370         &    -0.00115         &     0.00382         &    0.000888         \\
                    &   (0.00237)         &   (0.00322)         &   (0.00456)         &   (0.00482)         &   (0.00550)         &   (0.00588)         &   (0.00826)         &   (0.00888)         \\
[1em]
Ongoing war           &    -0.00129         &    0.000590         &     0.00336         &     0.00749         &      0.0116\sym{*}  &      0.0138\sym{*}  &      0.0190\sym{*}  &     0.00610         \\
                    &   (0.00165)         &   (0.00300)         &   (0.00422)         &   (0.00557)         &   (0.00694)         &   (0.00817)         &    (0.0108)         &    (0.0110)         \\
[1em]
Recent war          &     0.00952\sym{***}&      0.0108\sym{**} &     0.00980\sym{*}  &     0.00795         &     0.00602         &     0.00425         &     0.00214         &    -0.00457         \\
                    &   (0.00342)         &   (0.00461)         &   (0.00507)         &   (0.00563)         &   (0.00607)         &   (0.00716)         &   (0.00830)         &   (0.00834)         \\
[1em]
Civil participation, lag     &     -0.0593\sym{***}&      -0.114\sym{***}&      -0.165\sym{***}&      -0.215\sym{***}&      -0.265\sym{***}&      -0.315\sym{***}&      -0.545\sym{***}&      -0.756\sym{***}\\
                    &   (0.00538)         &   (0.00927)         &    (0.0128)         &    (0.0167)         &    (0.0204)         &    (0.0239)         &    (0.0401)         &    (0.0498)         \\
[1em]
$\Delta$Polity scores          &     0.00475\sym{***}&     0.00831\sym{***}&     0.00892\sym{***}&     0.00849\sym{***}&     0.00815\sym{***}&     0.00749\sym{***}&     0.00575\sym{***}&     0.00357\sym{***}\\
                    &  (0.000522)         &  (0.000907)         &   (0.00102)         &   (0.00104)         &   (0.00105)         &  (0.000977)         &  (0.000872)         &   (0.00101)         \\
[1em]
Polity scores, lag           &     0.00109\sym{***}&     0.00144\sym{***}&     0.00139\sym{***}&     0.00123\sym{***}&     0.00107\sym{**} &    0.000950         &    0.000464         &  0.00000834         \\
                    &  (0.000161)         &  (0.000260)         &  (0.000337)         &  (0.000428)         &  (0.000525)         &  (0.000622)         &   (0.00106)         &   (0.00130)         \\
[1em]
$\Delta$Energy.consump,log            &    0.000781         &   0.0000959         &    0.000724         &   -0.000927         &    -0.00173         &    -0.00386\sym{*}  &    -0.00566\sym{*}  &    -0.00900\sym{**} \\
                    &   (0.00120)         &   (0.00127)         &   (0.00157)         &   (0.00183)         &   (0.00200)         &   (0.00224)         &   (0.00320)         &   (0.00373)         \\
[1em]
Energy.consump,log, lag           &    0.000262         &    0.000482         &    0.000678         &    0.000852         &     0.00107         &     0.00133         &     0.00442         &     0.00747\sym{**} \\
                    &  (0.000248)         &  (0.000478)         &  (0.000743)         &   (0.00103)         &   (0.00131)         &   (0.00158)         &   (0.00280)         &   (0.00371)         \\
[1em]
year                &    0.000294\sym{***}&    0.000626\sym{***}&    0.000979\sym{***}&     0.00134\sym{***}&     0.00170\sym{***}&     0.00205\sym{***}&     0.00349\sym{***}&     0.00473\sym{***}\\
                    & (0.0000361)         & (0.0000687)         &  (0.000103)         &  (0.000138)         &  (0.000172)         &  (0.000206)         &  (0.000355)         &  (0.000463)         \\
[1em]
$\Delta$Neighboring empowerment, lag&      0.0167\sym{***}&      0.0303\sym{***}&      0.0357\sym{***}&      0.0360\sym{***}&      0.0289\sym{**} &      0.0323\sym{**} &      0.0410\sym{*}  &      0.0416\sym{***}\\
                    &   (0.00430)         &   (0.00736)         &    (0.0117)         &    (0.0113)         &    (0.0111)         &    (0.0131)         &    (0.0225)         &    (0.0156)         \\
[1em]
Constant            &      -0.550\sym{***}&      -1.173\sym{***}&      -1.842\sym{***}&      -2.520\sym{***}&      -3.200\sym{***}&      -3.863\sym{***}&      -6.587\sym{***}&      -8.917\sym{***}\\
                    &    (0.0682)         &     (0.130)         &     (0.194)         &     (0.261)         &     (0.325)         &     (0.390)         &     (0.672)         &     (0.877)         \\
\hline
Observations        &        9243         &        9163         &        9051         &        8935         &        8810         &        8684         &        8061         &        7465         \\
\hline\hline
\multicolumn{9}{l}{\footnotesize Standard errors in parentheses}\\
\multicolumn{9}{l}{\footnotesize \sym{*} \(p<0.10\), \sym{**} \(p<0.05\), \sym{***} \(p<0.01\)}\\
\end{tabular}
\end{table}
 %label {fecivilparticip} % index models  110-117
}

% Table B7
% infant mortality and population
{\renewcommand\normalsize{\tiny}% adjust table size
	\normalsize
\begin{table}[htbp]\centering
\def\sym#1{\ifmmode^{#1}\else\(^{#1}\)\fi}
\caption{Fixed-effects models of the effect of existential war on changes in fertility rates (controlling for infant mortality) \label{fefertilityinfant}}
\begin{tabular}{l*{8}{c}}
\hline\hline
                    &\multicolumn{1}{c}{\shortstack{Model 118\\(current)}}&\multicolumn{1}{c}{\shortstack{Model 119\\(1-year)}}&\multicolumn{1}{c}{\shortstack{Model 120\\(2-year)}}&\multicolumn{1}{c}{\shortstack{Model 121\\(3-year)}}&\multicolumn{1}{c}{\shortstack{Model 122\\(4-year)}}&\multicolumn{1}{c}{\shortstack{Model 123\\(5-year)}}&\multicolumn{1}{c}{\shortstack{Model 124\\(10-year)}}&\multicolumn{1}{c}{\shortstack{Model 125\\(15-year)}}\\
\hline
Existential war    &     -0.0183\sym{*}  &     -0.0370\sym{**} &     -0.0540\sym{**} &     -0.0684\sym{**} &     -0.0787\sym{**} &     -0.0903\sym{**} &      -0.135\sym{**} &      -0.121\sym{*}  \\
                    &   (0.00937)         &    (0.0177)         &    (0.0247)         &    (0.0311)         &    (0.0361)         &    (0.0409)         &    (0.0655)         &    (0.0628)         \\
[1em]
Fertility rate, lag      &     -0.0323\sym{***}&     -0.0727\sym{***}&      -0.120\sym{***}&      -0.173\sym{***}&      -0.230\sym{***}&      -0.289\sym{***}&      -0.604\sym{***}&      -0.877\sym{***}\\
                    &   (0.00318)         &   (0.00654)         &    (0.0103)         &    (0.0142)         &    (0.0178)         &    (0.0214)         &    (0.0354)         &    (0.0387)         \\
[1em]
$\Delta$Polity scores            &    -0.00109\sym{**} &    -0.00240\sym{**} &    -0.00328\sym{**} &    -0.00425\sym{**} &    -0.00466\sym{**} &    -0.00584\sym{**} &    -0.00612\sym{*}  &    -0.00492         \\
                    &  (0.000519)         &  (0.000946)         &   (0.00133)         &   (0.00170)         &   (0.00206)         &   (0.00229)         &   (0.00318)         &   (0.00305)         \\
[1em]
Polity scores, lag           &   -0.000422         &   -0.000905         &    -0.00150         &    -0.00208         &    -0.00255         &    -0.00295         &    -0.00305         &    -0.00243         \\
                    &  (0.000538)         &   (0.00106)         &   (0.00156)         &   (0.00203)         &   (0.00247)         &   (0.00288)         &   (0.00424)         &   (0.00400)         \\
[1em]
$\Delta$Energy.consump,log             &    0.000739         &     0.00220         &     0.00220         &     0.00127         &     0.00193         &     0.00124         &     0.00381         &    -0.00406         \\
                    &   (0.00153)         &   (0.00284)         &   (0.00405)         &   (0.00696)         &   (0.00841)         &   (0.00974)         &    (0.0139)         &    (0.0131)         \\
[1em]
Energy.consump,log, lag            &    -0.00619\sym{***}&     -0.0117\sym{***}&     -0.0181\sym{**} &     -0.0222\sym{**} &     -0.0254\sym{**} &     -0.0276\sym{**} &     -0.0329         &     -0.0370         \\
                    &   (0.00211)         &   (0.00428)         &   (0.00697)         &   (0.00933)         &    (0.0116)         &    (0.0136)         &    (0.0205)         &    (0.0225)         \\
[1em]
Year                &     0.00223\sym{***}&     0.00429\sym{***}&     0.00627\sym{***}&     0.00824\sym{***}&      0.0101\sym{***}&      0.0119\sym{***}&      0.0169\sym{***}&      0.0152\sym{***}\\
                    &  (0.000302)         &  (0.000607)         &  (0.000912)         &   (0.00121)         &   (0.00147)         &   (0.00172)         &   (0.00284)         &   (0.00336)         \\
[1em]
$\Delta$Infant mortality&   -0.000935         &    -0.00255         &    -0.00517         &    -0.00839         &     -0.0106         &     -0.0136         &    -0.00517         &      0.0190         \\
                    &   (0.00167)         &   (0.00305)         &   (0.00432)         &   (0.00579)         &   (0.00740)         &   (0.00956)         &    (0.0158)         &    (0.0149)         \\
[1em]
Infant mortality, lag &     0.00117\sym{***}&     0.00242\sym{***}&     0.00367\sym{***}&     0.00501\sym{***}&     0.00636\sym{***}&     0.00769\sym{***}&      0.0135\sym{***}&      0.0156\sym{***}\\
                    &  (0.000201)         &  (0.000403)         &  (0.000606)         &  (0.000810)         &   (0.00102)         &   (0.00121)         &   (0.00197)         &   (0.00207)         \\
[1em]
$\Delta$Population,log              &     0.00644         &     -0.0247         &     -0.0555         &      -0.122         &      -0.187         &      -0.258         &      -0.582\sym{**} &      -1.281\sym{***}\\
                    &    (0.0248)         &    (0.0482)         &    (0.0698)         &     (0.101)         &     (0.129)         &     (0.158)         &     (0.290)         &     (0.274)         \\
[1em]
Population,log, lag             &     -0.0886\sym{***}&      -0.183\sym{***}&      -0.289\sym{***}&      -0.405\sym{***}&      -0.525\sym{***}&      -0.652\sym{***}&      -1.228\sym{***}&      -1.607\sym{***}\\
                    &    (0.0199)         &    (0.0396)         &    (0.0599)         &    (0.0801)         &    (0.0999)         &     (0.119)         &     (0.178)         &     (0.154)         \\
[1em]
$\Delta$Neigh. fertility rate, lag&      0.0291\sym{***}&      0.0544\sym{***}&      0.0684\sym{**} &      0.0809\sym{**} &      0.0887\sym{**} &      0.0967\sym{**} &      0.0793         &      0.0363         \\
                    &    (0.0109)         &    (0.0203)         &    (0.0268)         &    (0.0325)         &    (0.0402)         &    (0.0454)         &    (0.0578)         &    (0.0559)         \\
[1em]
Constant            &      -3.553\sym{***}&      -6.706\sym{***}&      -9.548\sym{***}&      -12.29\sym{***}&      -14.86\sym{***}&      -17.11\sym{***}&      -20.95\sym{***}&      -13.22\sym{**} \\
                    &     (0.529)         &     (1.068)         &     (1.606)         &     (2.137)         &     (2.583)         &     (3.024)         &     (5.186)         &     (6.477)         \\
\hline
Observations        &        6339         &        6313         &        6212         &        6108         &        5978         &        5847         &        5192         &        4547         \\
\hline\hline
\multicolumn{9}{l}{\footnotesize Standard errors in parentheses}\\
\multicolumn{9}{l}{\footnotesize \sym{*} \(p<0.10\), \sym{**} \(p<0.05\), \sym{***} \(p<0.01\)}\\
\end{tabular}
\end{table}
 %label {fefertilityinfant} % index models  118-125
}

% Table B8
% Intrastate war and fertility 
{\renewcommand\normalsize{\tiny}% adjust table size
	\normalsize
\begin{table}[htbp]\centering
\def\sym#1{\ifmmode^{#1}\else\(^{#1}\)\fi}
\caption{Fixed-effects models of the effect of intrastate war on future changes in fertility rates \label{intrawarfertility}}
\begin{tabular}{l*{8}{c}}
\hline\hline
                    &\multicolumn{1}{c}{\shortstack{Model 126\\(current)}}&\multicolumn{1}{c}{\shortstack{Model 127\\(1-year)}}&\multicolumn{1}{c}{\shortstack{Model 128\\(2-year)}}&\multicolumn{1}{c}{\shortstack{Model 129\\(3-year)}}&\multicolumn{1}{c}{\shortstack{Model 130\\(4-year)}}&\multicolumn{1}{c}{\shortstack{Model 131\\(5-year)}}&\multicolumn{1}{c}{\shortstack{Model 132\\(10-year)}}&\multicolumn{1}{c}{\shortstack{Model 133\\(15-year)}}\\
\hline
Intrastate war      &     0.00443         &     0.00753         &     0.00996         &      0.0108         &      0.0124         &      0.0127         &    -0.00129         &     -0.0388         \\
                    &   (0.00636)         &    (0.0126)         &    (0.0187)         &    (0.0246)         &    (0.0293)         &    (0.0337)         &    (0.0512)         &    (0.0677)         \\
[1em]
Fertility rate, lag        &     -0.0124\sym{***}&     -0.0329\sym{***}&     -0.0595\sym{***}&     -0.0914\sym{***}&      -0.128\sym{***}&      -0.169\sym{***}&      -0.411\sym{***}&      -0.672\sym{***}\\
                    &   (0.00242)         &   (0.00488)         &   (0.00741)         &   (0.01000)         &    (0.0127)         &    (0.0154)         &    (0.0277)         &    (0.0355)         \\
[1em]
$\Delta$Polity scores           &    -0.00163\sym{***}&    -0.00344\sym{***}&    -0.00475\sym{***}&    -0.00622\sym{***}&    -0.00747\sym{***}&    -0.00909\sym{***}&     -0.0114\sym{***}&     -0.0107\sym{***}\\
                    &  (0.000523)         &  (0.000944)         &   (0.00133)         &   (0.00170)         &   (0.00209)         &   (0.00233)         &   (0.00331)         &   (0.00365)         \\
[1em]
Polity scores, lag             &    -0.00103\sym{**} &    -0.00213\sym{**} &    -0.00332\sym{**} &    -0.00444\sym{**} &    -0.00554\sym{**} &    -0.00647\sym{**} &    -0.00833\sym{*}  &    -0.00752         \\
                    &  (0.000475)         &  (0.000936)         &   (0.00139)         &   (0.00183)         &   (0.00224)         &   (0.00263)         &   (0.00442)         &   (0.00567)         \\
[1em]
$\Delta$Energy.consump,log            &    -0.00121         &    -0.00203         &    -0.00491         &     -0.0134\sym{*}  &     -0.0160\sym{*}  &     -0.0192\sym{*}  &     -0.0228         &     -0.0293         \\
                    &   (0.00143)         &   (0.00277)         &   (0.00417)         &   (0.00757)         &   (0.00948)         &    (0.0113)         &    (0.0180)         &    (0.0193)         \\
[1em]
Energy.consump,log, lag            &     -0.0160\sym{***}&     -0.0317\sym{***}&     -0.0496\sym{***}&     -0.0652\sym{***}&     -0.0789\sym{***}&     -0.0914\sym{***}&      -0.135\sym{***}&      -0.151\sym{***}\\
                    &   (0.00217)         &   (0.00451)         &   (0.00745)         &    (0.0102)         &    (0.0129)         &    (0.0154)         &    (0.0253)         &    (0.0280)         \\
[1em]
year                &    0.000533\sym{**} &    0.000694         &    0.000575         &    0.000107         &   -0.000634         &    -0.00166         &     -0.0101\sym{***}&     -0.0219\sym{***}\\
                    &  (0.000247)         &  (0.000509)         &  (0.000798)         &   (0.00111)         &   (0.00144)         &   (0.00179)         &   (0.00357)         &   (0.00493)         \\
[1em]
$\Delta$ Neigh. Fertility rate, lag&      0.0358\sym{***}&      0.0692\sym{***}&      0.0942\sym{***}&       0.119\sym{***}&       0.145\sym{***}&       0.166\sym{***}&       0.209\sym{***}&       0.190\sym{**} \\
                    &    (0.0118)         &    (0.0226)         &    (0.0311)         &    (0.0392)         &    (0.0486)         &    (0.0561)         &    (0.0756)         &    (0.0737)         \\
[1em]
Constant            &      -0.915\sym{*}  &      -1.058         &      -0.601         &       0.554         &       2.256         &       4.534         &       22.43\sym{***}&       46.98\sym{***}\\
                    &     (0.482)         &     (0.993)         &     (1.554)         &     (2.163)         &     (2.805)         &     (3.488)         &     (7.000)         &     (9.729)         \\
\hline
Observations        &        6677         &        6651         &        6550         &        6446         &        6316         &        6185         &        5530         &        4885         \\
\hline\hline
\multicolumn{9}{l}{\footnotesize Standard errors in parentheses}\\
\multicolumn{9}{l}{\footnotesize \sym{*} \(p<0.10\), \sym{**} \(p<0.05\), \sym{***} \(p<0.01\)}\\
\end{tabular}
\end{table}
 %label {intrawarfertility} % index models  126-133
}

% Table B9
% Interstate war and fertility
{\renewcommand\normalsize{\tiny}% adjust table size
	\normalsize
\begin{table}[htbp]\centering
\def\sym#1{\ifmmode^{#1}\else\(^{#1}\)\fi}
\caption{Fixed-effects models of the effect of interstate war on future changes in fertility rates \label{interwarfertility}}
\begin{tabular}{l*{8}{c}}
\hline\hline
                    &\multicolumn{1}{c}{\shortstack{Model 134\\(current)}}&\multicolumn{1}{c}{\shortstack{Model 135\\(1-year)}}&\multicolumn{1}{c}{\shortstack{Model 136\\(2-year)}}&\multicolumn{1}{c}{\shortstack{Model 137\\(3-year)}}&\multicolumn{1}{c}{\shortstack{Model 138\\(4-year)}}&\multicolumn{1}{c}{\shortstack{Model 139\\(5-year)}}&\multicolumn{1}{c}{\shortstack{Model 140\\(10-year)}}&\multicolumn{1}{c}{\shortstack{Model 141\\(15-year)}}\\
\hline
Interstate war     &     -0.0167\sym{*}  &     -0.0303\sym{*}  &     -0.0442\sym{*}  &     -0.0557         &     -0.0643         &     -0.0750         &     -0.0915         &      -0.131         \\
                    &   (0.00896)         &    (0.0173)         &    (0.0263)         &    (0.0362)         &    (0.0470)         &    (0.0613)         &     (0.130)         &     (0.153)         \\
[1em]
Fertility rate, lag       &     -0.0122\sym{***}&     -0.0324\sym{***}&     -0.0588\sym{***}&     -0.0904\sym{***}&      -0.127\sym{***}&      -0.167\sym{***}&      -0.409\sym{***}&      -0.670\sym{***}\\
                    &   (0.00243)         &   (0.00490)         &   (0.00744)         &    (0.0100)         &    (0.0127)         &    (0.0154)         &    (0.0280)         &    (0.0356)         \\
[1em]
$\Delta$Polity scores          &    -0.00164\sym{***}&    -0.00346\sym{***}&    -0.00478\sym{***}&    -0.00626\sym{***}&    -0.00751\sym{***}&    -0.00915\sym{***}&     -0.0115\sym{***}&     -0.0108\sym{***}\\
                    &  (0.000513)         &  (0.000926)         &   (0.00131)         &   (0.00167)         &   (0.00206)         &   (0.00229)         &   (0.00329)         &   (0.00367)         \\
[1em]
Polity scores, lag             &    -0.00101\sym{**} &    -0.00209\sym{**} &    -0.00325\sym{**} &    -0.00436\sym{**} &    -0.00544\sym{**} &    -0.00636\sym{**} &    -0.00823\sym{*}  &    -0.00743         \\
                    &  (0.000480)         &  (0.000945)         &   (0.00140)         &   (0.00185)         &   (0.00226)         &   (0.00266)         &   (0.00444)         &   (0.00569)         \\
[1em]
$\Delta$Energy.consump,log           &    -0.00123         &    -0.00207         &    -0.00493         &     -0.0135\sym{*}  &     -0.0161\sym{*}  &     -0.0193\sym{*}  &     -0.0227         &     -0.0284         \\
                    &   (0.00144)         &   (0.00279)         &   (0.00419)         &   (0.00764)         &   (0.00956)         &    (0.0114)         &    (0.0182)         &    (0.0196)         \\
[1em]
Energy.consump,log, lag           &     -0.0159\sym{***}&     -0.0316\sym{***}&     -0.0494\sym{***}&     -0.0650\sym{***}&     -0.0787\sym{***}&     -0.0911\sym{***}&      -0.135\sym{***}&      -0.150\sym{***}\\
                    &   (0.00215)         &   (0.00448)         &   (0.00740)         &    (0.0102)         &    (0.0128)         &    (0.0154)         &    (0.0253)         &    (0.0280)         \\
[1em]
Year                &    0.000527\sym{**} &    0.000685         &    0.000559         &   0.0000858         &   -0.000655         &    -0.00168         &     -0.0101\sym{***}&     -0.0221\sym{***}\\
                    &  (0.000245)         &  (0.000506)         &  (0.000795)         &   (0.00111)         &   (0.00143)         &   (0.00178)         &   (0.00357)         &   (0.00494)         \\
[1em]
$\Delta$Neigh. Fertility rate, lag&      0.0356\sym{***}&      0.0688\sym{***}&      0.0937\sym{***}&       0.118\sym{***}&       0.144\sym{***}&       0.165\sym{***}&       0.209\sym{***}&       0.189\sym{**} \\
                    &    (0.0117)         &    (0.0225)         &    (0.0309)         &    (0.0390)         &    (0.0483)         &    (0.0558)         &    (0.0752)         &    (0.0735)         \\
[1em]
Constant            &      -0.904\sym{*}  &      -1.041         &      -0.571         &       0.592         &       2.294         &       4.573         &       22.45\sym{***}&       47.27\sym{***}\\
                    &     (0.479)         &     (0.988)         &     (1.547)         &     (2.156)         &     (2.797)         &     (3.481)         &     (7.004)         &     (9.750)         \\
\hline
Observations        &        6677         &        6651         &        6550         &        6446         &        6316         &        6185         &        5530         &        4885         \\
\hline\hline
\multicolumn{9}{l}{\footnotesize Standard errors in parentheses}\\
\multicolumn{9}{l}{\footnotesize \sym{*} \(p<0.10\), \sym{**} \(p<0.05\), \sym{***} \(p<0.01\)}\\
\end{tabular}
\end{table}
 %label {interwarfertility} % index models  134-141
}

% Table B10
% battle deaths and fertility
{\renewcommand\normalsize{\tiny}% adjust table size
	\normalsize
\begin{table}[htbp]\centering
\def\sym#1{\ifmmode^{#1}\else\(^{#1}\)\fi}
\caption{Fixed-effects models of the effect of battle deaths on future changes in fertility rates \label{fertilitybdeath}}
\begin{tabular}{l*{8}{c}}
\hline\hline
                    &\multicolumn{1}{c}{\shortstack{Model 142\\(current)}}&\multicolumn{1}{c}{\shortstack{Model 143\\(1-year)}}&\multicolumn{1}{c}{\shortstack{Model 144\\(2-year)}}&\multicolumn{1}{c}{\shortstack{Model 145\\(3-year)}}&\multicolumn{1}{c}{\shortstack{Model 146\\(4-year)}}&\multicolumn{1}{c}{\shortstack{Model 147\\(5-year)}}&\multicolumn{1}{c}{\shortstack{Model 148\\(10-year)}}&\multicolumn{1}{c}{\shortstack{Model 149\\(15-year)}}\\
\hline
Battle deaths, log           &   -0.000491         &   -0.000969         &    -0.00136         &    -0.00172         &    -0.00208         &    -0.00238         &    -0.00601         &     -0.0112         \\
                    &  (0.000752)         &   (0.00150)         &   (0.00224)         &   (0.00298)         &   (0.00372)         &   (0.00449)         &   (0.00855)         &    (0.0108)         \\
[1em]
Fertility rate, lag      &    -0.00861\sym{***}&     -0.0263\sym{***}&     -0.0516\sym{***}&     -0.0838\sym{***}&      -0.122\sym{***}&      -0.164\sym{***}&      -0.410\sym{***}&      -0.672\sym{***}\\
                    &   (0.00305)         &   (0.00604)         &   (0.00888)         &    (0.0116)         &    (0.0142)         &    (0.0166)         &    (0.0279)         &    (0.0356)         \\
[1em]
$\Delta$Polity scores            &    -0.00160\sym{***}&    -0.00340\sym{***}&    -0.00493\sym{***}&    -0.00649\sym{***}&    -0.00763\sym{***}&    -0.00902\sym{***}&     -0.0115\sym{***}&     -0.0109\sym{***}\\
                    &  (0.000539)         &  (0.000968)         &   (0.00137)         &   (0.00176)         &   (0.00216)         &   (0.00238)         &   (0.00331)         &   (0.00368)         \\
[1em]
Polity scores, lag          &   -0.000887\sym{*}  &    -0.00182\sym{*}  &    -0.00283\sym{**} &    -0.00387\sym{**} &    -0.00492\sym{**} &    -0.00592\sym{**} &    -0.00830\sym{*}  &    -0.00745         \\
                    &  (0.000482)         &  (0.000947)         &   (0.00141)         &   (0.00185)         &   (0.00226)         &   (0.00265)         &   (0.00444)         &   (0.00569)         \\
[1em]
$\Delta$Energy.consump,log             &    -0.00418\sym{**} &    -0.00768\sym{**} &     -0.0117\sym{**} &     -0.0149\sym{*}  &     -0.0174\sym{*}  &     -0.0199\sym{*}  &     -0.0237         &     -0.0304         \\
                    &   (0.00190)         &   (0.00375)         &   (0.00563)         &   (0.00753)         &   (0.00946)         &    (0.0113)         &    (0.0180)         &    (0.0197)         \\
[1em]
Energy.consump,log, lag             &     -0.0168\sym{***}&     -0.0329\sym{***}&     -0.0485\sym{***}&     -0.0631\sym{***}&     -0.0767\sym{***}&     -0.0892\sym{***}&      -0.135\sym{***}&      -0.150\sym{***}\\
                    &   (0.00267)         &   (0.00542)         &   (0.00819)         &    (0.0109)         &    (0.0135)         &    (0.0159)         &    (0.0254)         &    (0.0280)         \\
[1em]
Year                &    0.000536         &    0.000663         &    0.000493         & -0.00000774         &   -0.000783         &    -0.00188         &    -0.01000\sym{***}&     -0.0218\sym{***}\\
                    &  (0.000346)         &  (0.000687)         &   (0.00103)         &   (0.00136)         &   (0.00167)         &   (0.00197)         &   (0.00355)         &   (0.00489)         \\
[1em]
$\Delta$Neigh. fertility rate, lag&      0.0316\sym{***}&      0.0614\sym{***}&      0.0866\sym{***}&       0.112\sym{***}&       0.135\sym{***}&       0.158\sym{***}&       0.209\sym{***}&       0.190\sym{**} \\
                    &    (0.0108)         &    (0.0208)         &    (0.0294)         &    (0.0378)         &    (0.0458)         &    (0.0538)         &    (0.0750)         &    (0.0734)         \\
[1em]
Constant            &      -0.932         &      -1.016         &      -0.481         &       0.730         &       2.504         &       4.922         &       22.28\sym{***}&       46.73\sym{***}\\
                    &     (0.677)         &     (1.344)         &     (2.008)         &     (2.657)         &     (3.275)         &     (3.857)         &     (6.979)         &     (9.658)         \\
\hline
Observations        &        5922         &        5923         &        5923         &        5923         &        5923         &        5923         &        5530         &        4885         \\
\hline\hline
\multicolumn{9}{l}{\footnotesize Standard errors in parentheses}\\
\multicolumn{9}{l}{\footnotesize \sym{*} \(p<0.10\), \sym{**} \(p<0.05\), \sym{***} \(p<0.01\)}\\
\end{tabular}
\end{table}
 %label {fertilitybdeath} % index models  142-149
}

%Table B11
\begin{landscape}
% Military expenditure
{\renewcommand\normalsize{\tiny}% adjust table size
	\normalsize
\begin{table}[htbp]\centering
\def\sym#1{\ifmmode^{#1}\else\(^{#1}\)\fi}
\caption{Fixed-effects models of the effect of war on future changes in women’s empowerment using military expenditures \label{fepolemmilex}}
\begin{tabular}{l*{9}{c}}
\hline\hline
                    &\multicolumn{1}{c}{\shortstack{Model 150\\(military expenditure)}}&\multicolumn{1}{c}{\shortstack{Model 151\\(current)}}&\multicolumn{1}{c}{\shortstack{Model 152\\(1-year)}}&\multicolumn{1}{c}{\shortstack{Model 153\\(2-year)}}&\multicolumn{1}{c}{\shortstack{Model 154\\(3-year)}}&\multicolumn{1}{c}{\shortstack{Model 155\\(4-year)}}&\multicolumn{1}{c}{\shortstack{Model 156\\(5-year)}}&\multicolumn{1}{c}{\shortstack{Model 157\\(10-year)}}&\multicolumn{1}{c}{\shortstack{Model 158\\(15-year)}}\\
\hline
War             &       0.160\sym{***}&    -0.00181         &   -0.000587         &     0.00131         &     0.00310         &     0.00399         &     0.00443         &    0.000183         &    -0.00563         \\
                    &    (0.0251)         &   (0.00136)         &   (0.00206)         &   (0.00299)         &   (0.00376)         &   (0.00442)         &   (0.00530)         &   (0.00862)         &   (0.00972)         \\
[1em]
Mili ex.pc, log, lag        &      -0.101\sym{***}&     0.00133\sym{***}&     0.00224\sym{***}&     0.00398\sym{***}&     0.00542\sym{***}&     0.00640\sym{**} &     0.00715\sym{**} &      0.0105\sym{**} &     0.00723         \\
                    &   (0.00763)         &  (0.000371)         &  (0.000839)         &   (0.00142)         &   (0.00198)         &   (0.00245)         &   (0.00290)         &   (0.00461)         &   (0.00553)         \\
[1em]
$\Delta$Mil.ex.pc,log        &                     &    -0.00178\sym{*}  &    -0.00199\sym{*}  &    -0.00114         &   -0.000673         &     0.00170         &     0.00256         &     0.00818\sym{**} &      0.0105\sym{**} \\
                    &                     &  (0.000908)         &   (0.00105)         &   (0.00145)         &   (0.00238)         &   (0.00250)         &   (0.00302)         &   (0.00401)         &   (0.00501)         \\
[1em]
$\Delta$Polity scores           &    -0.00522         &     0.00345\sym{***}&     0.00601\sym{***}&     0.00632\sym{***}&     0.00578\sym{***}&     0.00561\sym{***}&     0.00535\sym{***}&     0.00403\sym{***}&     0.00397\sym{***}\\
                    &   (0.00322)         &  (0.000471)         &  (0.000797)         &  (0.000933)         &  (0.000937)         &   (0.00102)         &  (0.000982)         &  (0.000884)         &  (0.000922)         \\
[1em]
Polity scores, lag            &   -0.000947         &    0.000799\sym{***}&     0.00107\sym{***}&     0.00129\sym{***}&     0.00150\sym{***}&     0.00167\sym{***}&     0.00190\sym{***}&     0.00248\sym{***}&     0.00149         \\
                    &   (0.00133)         &  (0.000102)         &  (0.000181)         &  (0.000262)         &  (0.000337)         &  (0.000406)         &  (0.000485)         &  (0.000840)         &  (0.000988)         \\
[1em]
$\Delta$Energy.consump,log            &       0.337\sym{***}&    0.000762         &    0.000718         &    0.000276         &   -0.000759         &    -0.00108         &    -0.00284         &    -0.00417         &    -0.00537         \\
                    &    (0.0531)         &  (0.000971)         &   (0.00101)         &   (0.00110)         &   (0.00127)         &   (0.00159)         &   (0.00227)         &   (0.00318)         &   (0.00377)         \\
[1em]
Energy.consump,log, lag            &    -0.00470         &    0.000209         &    0.000278         &    0.000325         &    0.000584         &    0.000866         &     0.00116         &     0.00231         &     0.00471         \\
                    &   (0.00447)         &  (0.000285)         &  (0.000538)         &  (0.000803)         &   (0.00109)         &   (0.00136)         &   (0.00161)         &   (0.00269)         &   (0.00343)         \\
[1em]
Year                &     0.00847\sym{***}&    0.000174\sym{***}&    0.000341\sym{***}&    0.000483\sym{***}&    0.000654\sym{***}&    0.000845\sym{***}&     0.00105\sym{***}&     0.00205\sym{***}&     0.00314\sym{***}\\
                    &  (0.000590)         & (0.0000373)         & (0.0000716)         &  (0.000115)         &  (0.000158)         &  (0.000202)         &  (0.000242)         &  (0.000429)         &  (0.000560)         \\
[1em]
Political empowerment, lag   &                     &     -0.0545\sym{***}&     -0.0963\sym{***}&      -0.144\sym{***}&      -0.194\sym{***}&      -0.243\sym{***}&      -0.292\sym{***}&      -0.525\sym{***}&      -0.699\sym{***}\\
                    &                     &   (0.00515)         &   (0.00880)         &    (0.0123)         &    (0.0163)         &    (0.0198)         &    (0.0234)         &    (0.0399)         &    (0.0465)         \\
[1em]
$\Delta$Population,log               &                     &    -0.00727         &    -0.00201         &      0.0127         &    -0.00877         &    -0.00894         &    -0.00668         &     -0.0812\sym{*}  &     -0.0938\sym{***}\\
                    &                     &   (0.00880)         &    (0.0139)         &    (0.0161)         &    (0.0213)         &    (0.0226)         &    (0.0307)         &    (0.0443)         &    (0.0325)         \\
[1em]
Population,log, lag              &                     &     0.00209         &     0.00542         &      0.0103\sym{*}  &      0.0143\sym{*}  &      0.0182\sym{*}  &      0.0215\sym{*}  &      0.0380\sym{*}  &      0.0460\sym{*}  \\
                    &                     &   (0.00179)         &   (0.00352)         &   (0.00532)         &   (0.00736)         &   (0.00933)         &    (0.0112)         &    (0.0199)         &    (0.0271)         \\
[1em]
Irre.leade.change        &                     &     0.00169         &     0.00983\sym{*}  &     0.00954\sym{*}  &      0.0137\sym{**} &      0.0150\sym{**} &      0.0142\sym{**} &     0.00892         &     0.00276         \\
                    &                     &   (0.00341)         &   (0.00522)         &   (0.00568)         &   (0.00553)         &   (0.00588)         &   (0.00620)         &   (0.00691)         &   (0.00657)         \\
[1em]
$\Delta$Neigh. empowerment, lag&                     &     0.00669         &      0.0203\sym{***}&      0.0217\sym{**} &      0.0228\sym{**} &      0.0172\sym{*}  &      0.0187\sym{**} &      0.0101         &      0.0142         \\
                    &                     &   (0.00405)         &   (0.00666)         &   (0.00943)         &   (0.00938)         &   (0.00877)         &   (0.00917)         &    (0.0151)         &    (0.0145)         \\
[1em]
Constant            &      -16.30\sym{***}&      -0.336\sym{***}&      -0.673\sym{***}&      -0.974\sym{***}&      -1.323\sym{***}&      -1.710\sym{***}&      -2.126\sym{***}&      -4.115\sym{***}&      -6.220\sym{***}\\
                    &     (1.125)         &    (0.0638)         &     (0.122)         &     (0.196)         &     (0.269)         &     (0.344)         &     (0.410)         &     (0.715)         &     (0.916)         \\
\hline
Observations        &        9810         &        7402         &        7223         &        7061         &        6895         &        6742         &        6604         &        5986         &        5472         \\
\hline\hline
\multicolumn{10}{l}{\footnotesize Standard errors in parentheses}\\
\multicolumn{10}{l}{\footnotesize \sym{*} \(p<0.10\), \sym{**} \(p<0.05\), \sym{***} \(p<0.01\)}\\
\end{tabular}
\end{table}
 %label {fepolemmilex} % index models  150-158
}
\end{landscape}


% Table B12
% territorial threats w/o war
{\renewcommand\normalsize{\tiny}% adjust table size
	\normalsize
\begin{table}[htbp]\centering
\def\sym#1{\ifmmode^{#1}\else\(^{#1}\)\fi}
\caption{Fixed-effects models of the effect of territorial threat on future changes in women's empowerment (without controlling for war) \label{fepolempnowar}}
\begin{tabular}{l*{8}{c}}
\hline\hline
                    &\multicolumn{1}{c}{\shortstack{Model 159\\(current)}}&\multicolumn{1}{c}{\shortstack{Model 160\\(1-year)}}&\multicolumn{1}{c}{\shortstack{Model 161\\(2-year)}}&\multicolumn{1}{c}{\shortstack{Model 162\\(3-year)}}&\multicolumn{1}{c}{\shortstack{Model 163\\(4-year)}}&\multicolumn{1}{c}{\shortstack{Model 164\\(5-year)}}&\multicolumn{1}{c}{\shortstack{Model 165\\(10-year)}}&\multicolumn{1}{c}{\shortstack{Model 166\\(15-year)}}\\
\hline
$\Delta$Territorial threat       &    -0.00482         &    0.000955         &     0.00447         &     0.00958         &      0.0107         &      0.0162         &      0.0315\sym{***}&     -0.0115         \\
                    &   (0.00404)         &   (0.00499)         &   (0.00647)         &   (0.00942)         &    (0.0107)         &    (0.0109)         &    (0.0115)         &    (0.0116)         \\
[1em]
Territorial threat, lag       &     0.00187         &     0.00878         &      0.0177         &      0.0268\sym{*}  &      0.0362\sym{**} &      0.0504\sym{***}&      0.0420\sym{*}  &     -0.0281         \\
                    &   (0.00423)         &   (0.00849)         &    (0.0129)         &    (0.0162)         &    (0.0178)         &    (0.0191)         &    (0.0214)         &    (0.0244)         \\
[1em]
Political empowerment, lag    &     -0.0552\sym{***}&      -0.101\sym{***}&      -0.151\sym{***}&      -0.205\sym{***}&      -0.255\sym{***}&      -0.306\sym{***}&      -0.513\sym{***}&      -0.672\sym{***}\\
                    &   (0.00648)         &    (0.0110)         &    (0.0152)         &    (0.0196)         &    (0.0239)         &    (0.0278)         &    (0.0416)         &    (0.0441)         \\
[1em]
$\Delta$Polity scores            &     0.00326\sym{***}&     0.00587\sym{***}&     0.00619\sym{***}&     0.00551\sym{***}&     0.00530\sym{***}&     0.00499\sym{***}&     0.00374\sym{***}&     0.00361\sym{***}\\
                    &  (0.000477)         &  (0.000819)         &  (0.000939)         &  (0.000941)         &  (0.000989)         &  (0.000939)         &  (0.000819)         &  (0.000838)         \\
[1em]
Polity scores, lag           &    0.000816\sym{***}&     0.00107\sym{***}&     0.00119\sym{***}&     0.00136\sym{***}&     0.00151\sym{***}&     0.00174\sym{***}&     0.00216\sym{**} &     0.00148         \\
                    &  (0.000125)         &  (0.000209)         &  (0.000286)         &  (0.000365)         &  (0.000434)         &  (0.000507)         &  (0.000842)         &  (0.000969)         \\
[1em]
$\Delta$Energy.consump,log            &    0.000307         &   -0.000265         &   -0.000671         &    -0.00142         &    -0.00225         &    -0.00394\sym{*}  &    -0.00496         &    -0.00523         \\
                    &  (0.000643)         &  (0.000807)         &   (0.00117)         &   (0.00134)         &   (0.00160)         &   (0.00220)         &   (0.00310)         &   (0.00384)         \\
[1em]
Energy.consump,log, lag            &  -0.0000737         &   -0.000189         &   -0.000133         &   0.0000489         &    0.000375         &    0.000755         &     0.00396\sym{*}  &     0.00805\sym{***}\\
                    &  (0.000230)         &  (0.000440)         &  (0.000676)         &  (0.000911)         &   (0.00111)         &   (0.00129)         &   (0.00219)         &   (0.00298)         \\
[1em]
Year                &    0.000319\sym{***}&    0.000635\sym{***}&    0.000967\sym{***}&     0.00130\sym{***}&     0.00161\sym{***}&     0.00191\sym{***}&     0.00303\sym{***}&     0.00381\sym{***}\\
                    & (0.0000366)         & (0.0000670)         & (0.0000975)         &  (0.000131)         &  (0.000160)         &  (0.000185)         &  (0.000301)         &  (0.000389)         \\
[1em]
$\Delta$Neighboring  empowerment, lag&     0.00893\sym{**} &      0.0157\sym{***}&      0.0177\sym{**} &      0.0167\sym{**} &      0.0126         &      0.0195\sym{*}  &      0.0105         &      0.0256\sym{*}  \\
                    &   (0.00408)         &   (0.00591)         &   (0.00693)         &   (0.00772)         &   (0.00903)         &    (0.0107)         &    (0.0148)         &    (0.0134)         \\
[1em]
Constant            &      -0.596\sym{***}&      -1.190\sym{***}&      -1.814\sym{***}&      -2.446\sym{***}&      -3.024\sym{***}&      -3.580\sym{***}&      -5.682\sym{***}&      -7.135\sym{***}\\
                    &    (0.0685)         &     (0.125)         &     (0.183)         &     (0.245)         &     (0.300)         &     (0.347)         &     (0.566)         &     (0.734)         \\
\hline
Observations        &        6514         &        6405         &        6332         &        6267         &        6221         &        6191         &        6121         &        5870         \\
\hline\hline
\multicolumn{9}{l}{\footnotesize Standard errors in parentheses}\\
\multicolumn{9}{l}{\footnotesize \sym{*} \(p<0.10\), \sym{**} \(p<0.05\), \sym{***} \(p<0.01\)}\\
\end{tabular}
\end{table}
 %label {fepolempnowar} % index models  159-166
}

% Table B13
% territorial threats with war
{\renewcommand\normalsize{\tiny}% adjust table size
	\normalsize
\begin{table}[htbp]\centering
\def\sym#1{\ifmmode^{#1}\else\(^{#1}\)\fi}
\caption{Fixed-effects models of the effect of territorial threat on future changes in women's empowerment (controlling for war) \label{fepolempwar}}
\begin{tabular}{l*{8}{c}}
\hline\hline
                    &\multicolumn{1}{c}{\shortstack{Model 167\\(current)}}&\multicolumn{1}{c}{\shortstack{Model 168\\(1-year)}}&\multicolumn{1}{c}{\shortstack{Model 169\\(2-year)}}&\multicolumn{1}{c}{\shortstack{Model 170\\(3-year)}}&\multicolumn{1}{c}{\shortstack{Model 171\\(4-year)}}&\multicolumn{1}{c}{\shortstack{Model 172\\(5-year)}}&\multicolumn{1}{c}{\shortstack{Model 173\\(10-year)}}&\multicolumn{1}{c}{\shortstack{Model 174\\(15-year)}}\\
\hline
$\Delta$Territorial threat      &    -0.00374         &     0.00139         &     0.00393         &     0.00777         &     0.00829         &      0.0143         &      0.0290\sym{**} &     -0.0135         \\
                    &   (0.00403)         &   (0.00509)         &   (0.00649)         &   (0.00945)         &    (0.0107)         &    (0.0106)         &    (0.0118)         &    (0.0132)         \\
[1em]
Territorial threat, lag    &     0.00349         &     0.00946         &      0.0169         &      0.0240         &      0.0324\sym{*}  &      0.0474\sym{**} &      0.0381\sym{*}  &     -0.0311         \\
                    &   (0.00438)         &   (0.00855)         &    (0.0128)         &    (0.0160)         &    (0.0172)         &    (0.0184)         &    (0.0221)         &    (0.0270)         \\
[1em]
War                 &    -0.00214         &   -0.000893         &     0.00111         &     0.00372         &     0.00494         &     0.00390         &     0.00512         &     0.00417         \\
                    &   (0.00159)         &   (0.00261)         &   (0.00353)         &   (0.00431)         &   (0.00516)         &   (0.00579)         &   (0.00821)         &   (0.00969)         \\
[1em]
Political empowerment, lag   &     -0.0563\sym{***}&      -0.102\sym{***}&      -0.150\sym{***}&      -0.203\sym{***}&      -0.252\sym{***}&      -0.304\sym{***}&      -0.510\sym{***}&      -0.670\sym{***}\\
                    &   (0.00650)         &    (0.0110)         &    (0.0152)         &    (0.0196)         &    (0.0238)         &    (0.0277)         &    (0.0418)         &    (0.0453)         \\
[1em]
$\Delta$Polity scores          &     0.00326\sym{***}&     0.00587\sym{***}&     0.00619\sym{***}&     0.00551\sym{***}&     0.00531\sym{***}&     0.00499\sym{***}&     0.00375\sym{***}&     0.00361\sym{***}\\
                    &  (0.000477)         &  (0.000820)         &  (0.000939)         &  (0.000939)         &  (0.000991)         &  (0.000940)         &  (0.000815)         &  (0.000838)         \\
[1em]
Polity scores, lag            &    0.000830\sym{***}&     0.00108\sym{***}&     0.00118\sym{***}&     0.00133\sym{***}&     0.00148\sym{***}&     0.00171\sym{***}&     0.00213\sym{**} &     0.00145         \\
                    &  (0.000125)         &  (0.000209)         &  (0.000287)         &  (0.000366)         &  (0.000435)         &  (0.000508)         &  (0.000840)         &  (0.000975)         \\
[1em]
$\Delta$Energy.consump,log            &    0.000277         &   -0.000277         &   -0.000652         &    -0.00135         &    -0.00216         &    -0.00385\sym{*}  &    -0.00488         &    -0.00514         \\
                    &  (0.000642)         &  (0.000799)         &   (0.00115)         &   (0.00133)         &   (0.00159)         &   (0.00219)         &   (0.00307)         &   (0.00380)         \\
[1em]
Energy.consump,log, lag           &  -0.0000647         &   -0.000186         &   -0.000136         &   0.0000359         &    0.000359         &    0.000745         &     0.00392\sym{*}  &     0.00801\sym{***}\\
                    &  (0.000230)         &  (0.000440)         &  (0.000675)         &  (0.000910)         &   (0.00111)         &   (0.00129)         &   (0.00217)         &   (0.00295)         \\
[1em]
year                &    0.000323\sym{***}&    0.000637\sym{***}&    0.000965\sym{***}&     0.00130\sym{***}&     0.00160\sym{***}&     0.00190\sym{***}&     0.00302\sym{***}&     0.00380\sym{***}\\
                    & (0.0000367)         & (0.0000670)         & (0.0000976)         &  (0.000131)         &  (0.000160)         &  (0.000185)         &  (0.000302)         &  (0.000394)         \\
[1em]
$\Delta$Neighboring  empowerment, lag&     0.00901\sym{**} &      0.0158\sym{***}&      0.0177\sym{**} &      0.0165\sym{**} &      0.0125         &      0.0194\sym{*}  &      0.0106         &      0.0255\sym{*}  \\
                    &   (0.00408)         &   (0.00592)         &   (0.00694)         &   (0.00773)         &   (0.00902)         &    (0.0107)         &    (0.0148)         &    (0.0134)         \\
[1em]
Constant            &      -0.603\sym{***}&      -1.193\sym{***}&      -1.811\sym{***}&      -2.434\sym{***}&      -3.006\sym{***}&      -3.566\sym{***}&      -5.666\sym{***}&      -7.123\sym{***}\\
                    &    (0.0686)         &     (0.125)         &     (0.183)         &     (0.245)         &     (0.299)         &     (0.347)         &     (0.569)         &     (0.743)         \\
\hline
Observations        &        6514         &        6405         &        6332         &        6267         &        6221         &        6191         &        6121         &        5870         \\
\hline\hline
\multicolumn{9}{l}{\footnotesize Standard errors in parentheses}\\
\multicolumn{9}{l}{\footnotesize \sym{*} \(p<0.10\), \sym{**} \(p<0.05\), \sym{***} \(p<0.01\)}\\
\end{tabular}
\end{table}
 %label {fepolempnowar} % index models  167-174
}

%Table B14
% Interaction,  *** women's empowerment on the interaction of  the change in power consumption variable and the war variable
{\renewcommand\normalsize{\tiny}% adjust table size
	\normalsize
\begin{table}[htbp]\centering
\def\sym#1{\ifmmode^{#1}\else\(^{#1}\)\fi}
\caption{Interaction effects between change in energy consumption and war on changes in women's empowerment \label{interaction}}
\begin{tabular}{l*{8}{c}}
\hline\hline
                    &\multicolumn{1}{c}{\shortstack{Model 175-1\\(current)}}&\multicolumn{1}{c}{\shortstack{Model 175-2\\(1-year)}}&\multicolumn{1}{c}{\shortstack{Model 175-3\\(2-year)}}&\multicolumn{1}{c}{\shortstack{Model 175-4\\(3-year)}}&\multicolumn{1}{c}{\shortstack{Model 175-5\\(4-year)}}&\multicolumn{1}{c}{\shortstack{Model 175-6\\(5-year)}}&\multicolumn{1}{c}{\shortstack{Model 175-7\\(10-year)}}&\multicolumn{1}{c}{\shortstack{Model 175-8\\(15-year)}}\\
\hline
War            &    -0.00161         &    0.000615         &     0.00339         &     0.00686\sym{*}  &     0.00927\sym{**} &      0.0103\sym{*}  &      0.0108         &     0.00555         \\
                    &   (0.00124)         &   (0.00211)         &   (0.00297)         &   (0.00364)         &   (0.00448)         &   (0.00531)         &   (0.00811)         &   (0.00923)         \\
[1em]
$\Delta$Energy.consump,log             &    0.000275         &  -0.0000910         &   -0.000352         &    -0.00113         &    -0.00137         &    -0.00268         &    -0.00292         &    -0.00394         \\
                    &  (0.000883)         &  (0.000992)         &   (0.00110)         &   (0.00135)         &   (0.00163)         &   (0.00217)         &   (0.00282)         &   (0.00382)         \\
[1em]
War $\times$ $\Delta$Energy.consump,log&     0.00132         &    0.000281         &     0.00147         &     0.00238         &    0.000577         &    -0.00437         &     -0.0180         &     -0.0122         \\
                    &   (0.00204)         &   (0.00363)         &   (0.00453)         &   (0.00637)         &   (0.00648)         &    (0.0111)         &    (0.0117)         &    (0.0138)         \\
[1em]
Political empowerment, lag   &     -0.0520\sym{***}&     -0.0949\sym{***}&      -0.140\sym{***}&      -0.189\sym{***}&      -0.236\sym{***}&      -0.286\sym{***}&      -0.508\sym{***}&      -0.675\sym{***}\\
                    &   (0.00484)         &   (0.00847)         &    (0.0118)         &    (0.0155)         &    (0.0190)         &    (0.0226)         &    (0.0380)         &    (0.0447)         \\
[1em]
$\Delta$Polity scores            &     0.00318\sym{***}&     0.00572\sym{***}&     0.00595\sym{***}&     0.00539\sym{***}&     0.00509\sym{***}&     0.00482\sym{***}&     0.00358\sym{***}&     0.00356\sym{***}\\
                    &  (0.000434)         &  (0.000759)         &  (0.000870)         &  (0.000869)         &  (0.000929)         &  (0.000893)         &  (0.000780)         &  (0.000832)         \\
[1em]
Polity scores, lag            &    0.000707\sym{***}&    0.000928\sym{***}&     0.00103\sym{***}&     0.00117\sym{***}&     0.00128\sym{***}&     0.00149\sym{***}&     0.00194\sym{**} &     0.00129         \\
                    & (0.0000940)         &  (0.000164)         &  (0.000231)         &  (0.000300)         &  (0.000362)         &  (0.000434)         &  (0.000780)         &  (0.000955)         \\
[1em]
Energy.consump,log, lag            &    0.000286         &    0.000555         &    0.000945         &     0.00138\sym{*}  &     0.00184\sym{*}  &     0.00231\sym{**} &     0.00480\sym{**} &     0.00799\sym{***}\\
                    &  (0.000205)         &  (0.000388)         &  (0.000587)         &  (0.000796)         &  (0.000982)         &   (0.00116)         &   (0.00203)         &   (0.00284)         \\
[1em]
Year                &    0.000253\sym{***}&    0.000507\sym{***}&    0.000777\sym{***}&     0.00107\sym{***}&     0.00134\sym{***}&     0.00162\sym{***}&     0.00290\sym{***}&     0.00384\sym{***}\\
                    & (0.0000295)         & (0.0000552)         & (0.0000819)         &  (0.000110)         &  (0.000135)         &  (0.000160)         &  (0.000283)         &  (0.000380)         \\
[1em]
$\Delta$Neigh. empowerment, lag&     0.00605\sym{*}  &      0.0189\sym{***}&      0.0227\sym{***}&      0.0238\sym{***}&      0.0189\sym{**} &      0.0223\sym{**} &      0.0142         &      0.0249\sym{*}  \\
                    &   (0.00360)         &   (0.00574)         &   (0.00853)         &   (0.00861)         &   (0.00834)         &   (0.00875)         &    (0.0139)         &    (0.0132)         \\
[1em]
Constant            &      -0.471\sym{***}&      -0.949\sym{***}&      -1.456\sym{***}&      -1.997\sym{***}&      -2.510\sym{***}&      -3.045\sym{***}&      -5.440\sym{***}&      -7.197\sym{***}\\
                    &    (0.0552)         &     (0.104)         &     (0.154)         &     (0.207)         &     (0.253)         &     (0.301)         &     (0.534)         &     (0.716)         \\
\hline
Observations        &        8090         &        7889         &        7708         &        7527         &        7362         &        7209         &        6538         &        6009         \\
\hline\hline
\multicolumn{9}{l}{\footnotesize Standard errors in parentheses}\\
\multicolumn{9}{l}{\footnotesize \sym{*} \(p<0.10\), \sym{**} \(p<0.05\), \sym{***} \(p<0.01\)}\\
\end{tabular}
\end{table}
 %label {interaction} % index models  175
}



\section{Robustness Checks and Additional Mechanisms}
\vspace*{.2in}
\setlength{\parskip}{-2em}
%\setlength{\parindent}{1cm}
\doublespacing
\setcounter{table}{0}
\setcounter{figure}{0}
\renewcommand{\thetable}{C\arabic{table}}	
\renewcommand{\thefigure}{C\arabic{figure}}	

We run several sets of robustness checks to demonstrate that our findings are generally consistent across alternative measures of some key variables and alternative sets of controls. We present the following findings in the Tables \ref{intermedrbst}-\ref{polemprobustsimpyear}. \\

First, we consider the potential for the irregular regime change variable to be highly correlated with the levels and changes in the Polity score included as a control variable. We thus run the models that treat regime change as an intermediate variable without the Polity variables included.\footnote{See Model 176 in Table \ref{intermedrbst}.}  Even when the Polity variables are excluded, we consistently find that war increases the propensity for irregular regime change, which is associated with increases in women's empowerment.\footnote{See Models 178-185 in Table \ref{fepolempnopolity}.} \\

Second, we consider the potential for changes in the Polity score to serve as an intermediate variable.\footnote{See Model 177 in Table \ref{intermedrbst}; and Models 62-69 in Table \ref{intermpolempowerment}.} We find, however, that expected war is not a statistically significant variable that explains changes in Polity scores. Combined with the first robustness check above, it does appear that irregular regime change is capturing a political process that responds to war in ways that are distinct from shifts in the Polity score. Our results also show, consistent with common expectations about the relationship between democratization and gender equality, that Polity-score increases are associated with women's empowerment increases.\\


Polity scores might not adequately control for broader changes and differences in political participation that confound the relationship between war and women's empowerment. Specifically, we consider the possibility that war increases the political engagement of all members of society, not just women, which would undermine the inference that war has gendered social consequences. We thus replace the Polity score variables with two measures from the V-Dem data that capture aspects of democratic governance separate from women's empowerment. The first variable is the {\it civil society index}, which captures civil society robustness---how much individuals enjoy autonomy from the state and are free to pursue their political and civic goals. The second is the {\it clean election index}, defined as the extent to which elections are free and fair. Like the Polity scores, we include both the temporal lags and the first differences.\footnote{See Models 186-193 in Table \ref{fepolempvdem}.} The results confirm that war still is associated with women's empowerment increases even when these alternative controls---which are also associated with gains in women's empowerment---are included. These tests further support the conclusion that war is shaping women's empowerment and not just general political shifts that correlate with women's empowerment.\\


We also consider the possibility that other forms of instability create similar opportunities for women's empowerment. In separate fixed-effects regression models that replace war with {\it national strikes}, {\it government crises}, {\it riots} and {\it mass demonstrations}, we find that both national strikes and mass demonstrations are associated with gains in women's empowerment in the short, medium and long runs.\footnote{See Models 194-201, 202-209, 210-217 and 218-225, in Tables \ref{fepolemstriksnowar}, \ref{polemgovnowar}, \ref{fepolemriotnowar}, and \ref{polemdemonstrationnowar}, respectively. } When war is added in addition to these variables, the medium-term effect of war is robust.\footnote{See Models 26-233 and 234-241 in Tables \ref{fepolemstrikswar} and \ref{polemdemonstrationwar}, respectively. } War shapes gender power dynamics differently than other sources of social upheaval that also appear to open up opportunities for women's empowerment. Further analysis should investigate the mechanisms by which other types of social upheaval contribute to gender power dynamics.\\

We additionally consider whether WWII is a breakpoint in the relationship between war and women's empowerment. We thus run our main  model with an interaction between war and a dummy variable indicator of the post-1946 period.\footnote{See Models 242-249 in Table \ref{fepolemwwII}.} We do not find that the relationship between war and women's empowerment is significantly different in the pre- and post-WWII periods.\\

Finally, we consider a number of additional robustness checks of our main model of types of war on future changes in women's empowerment. In one set of models, we use lagged changes of polity scores and energy consumption to avoid post-treatment bias.\footnote{See Models 250-257 in Table \ref{polemprobustls}.} In another set, we add year fixed effects to further control for cross-panel autocorrelation.\footnote{See Models 258-265 in Table \ref{polemprobustyear}.} In two other sets of models, we check if our results hold in simple models without any control variables but with country fixed effects and, separately, with both country and year fixed effects.\footnote{See Models 266-273, and 274-281 in Tables \ref{polemprobustwocontros} and \ref{polemprobustsimpyear}.} Our main findings hold in each set of models.\\


%Table C1
% irregular without polity, polity as intermediate
{\renewcommand\normalsize{\tiny}% adjust table size
	\normalsize
\begin{table}[htbp]\centering
\def\sym#1{\ifmmode^{#1}\else\(^{#1}\)\fi}
\caption{Robustness Check: Irregular regime change and polity as intermediate variables \label{intermedrbst}}
\begin{tabular}{l*{2}{c}}
\hline\hline
                    &\multicolumn{1}{c}{\shortstack{Model 176\\(DV: Irregular leader change w/o polity)}}&\multicolumn{1}{c}{\shortstack{Model 177\\(DV: $\Delta$ polity2)}}\\
\hline
War                &      0.0883\sym{***}&     -0.0261         \\
                    &    (0.0118)         &    (0.0754)         \\
[1em]
$\Delta$Energy.consump,log             &     -0.0132\sym{***}&     -0.0834         \\
                    &   (0.00381)         &    (0.0660)         \\
[1em]
Energy.consump,log, lag             &    -0.00290         &      0.0257\sym{*}  \\
                    &   (0.00227)         &    (0.0143)         \\
[1em]
Year                &   -0.000424\sym{***}&     0.00601\sym{***}\\
                    &  (0.000149)         &   (0.00125)         \\
[1em]
Polity scores, lag         &                     &     -0.0735\sym{***}\\
                    &                     &   (0.00608)         \\
[1em]
Constant            &       0.896\sym{***}&      -11.95\sym{***}\\
                    &     (0.280)         &     (2.380)         \\
\hline
Observations        &       11507         &       10871         \\
\hline\hline
\multicolumn{3}{l}{\footnotesize Standard errors in parentheses}\\
\multicolumn{3}{l}{\footnotesize \sym{*} \(p<0.10\), \sym{**} \(p<0.05\), \sym{***} \(p<0.01\)}\\
\end{tabular}
\end{table}
 %label {intermedrbst} % index models  176-177
}

%Table C2
% irregular on empowerment without polity
{\renewcommand\normalsize{\tiny}% adjust table size
	\normalsize
\begin{table}[htbp]\centering
\def\sym#1{\ifmmode^{#1}\else\(^{#1}\)\fi}
\caption{Robustness Check: Fixed-effects models of the effect of irregular leadership change on future changes in women's empowerment(without polity) \label{fepolempnopolity}}
\begin{tabular}{l*{8}{c}}
\hline\hline
                    &\multicolumn{1}{c}{\shortstack{Model 178\\(current)}}&\multicolumn{1}{c}{\shortstack{Model 179\\(1-year)}}&\multicolumn{1}{c}{\shortstack{Model 180\\(2-year)}}&\multicolumn{1}{c}{\shortstack{Model 181\\(3-year)}}&\multicolumn{1}{c}{\shortstack{Model 182\\(4-year)}}&\multicolumn{1}{c}{\shortstack{Model 183\\(5-year)}}&\multicolumn{1}{c}{\shortstack{Model 184\\(10-year)}}&\multicolumn{1}{c}{\shortstack{Model 185\\(15-year)}}\\
\hline
War           &    -0.00162         &   -0.000227         &     0.00257         &     0.00446         &     0.00548         &     0.00635         &     0.00608         &     0.00413         \\
                    &   (0.00123)         &   (0.00202)         &   (0.00282)         &   (0.00335)         &   (0.00395)         &   (0.00457)         &   (0.00752)         &   (0.00844)         \\
[1em]
Political empowerment, lag   &     -0.0360\sym{***}&     -0.0741\sym{***}&      -0.115\sym{***}&      -0.159\sym{***}&      -0.201\sym{***}&      -0.243\sym{***}&      -0.454\sym{***}&      -0.656\sym{***}\\
                    &   (0.00423)         &   (0.00740)         &    (0.0102)         &    (0.0132)         &    (0.0160)         &    (0.0187)         &    (0.0312)         &    (0.0389)         \\
[1em]
$\Delta$Mil.per.pc,log         &    -0.00471         &    -0.00104         &   -0.000115         &    0.000809         &     0.00451         &     0.00584         &     0.00691         &     0.00105         \\
                    &   (0.00288)         &   (0.00329)         &   (0.00370)         &   (0.00412)         &   (0.00444)         &   (0.00566)         &   (0.00753)         &   (0.00976)         \\
[1em]
Mil.per.pc,log, lag     &     0.00332\sym{**} &     0.00743\sym{**} &     0.00978\sym{**} &      0.0118\sym{*}  &      0.0129\sym{*}  &      0.0132         &     0.00664         &     -0.0120         \\
                    &   (0.00152)         &   (0.00296)         &   (0.00453)         &   (0.00599)         &   (0.00732)         &   (0.00843)         &    (0.0129)         &    (0.0142)         \\
[1em]
$\Delta$Population,log             &    -0.00358         &     -0.0256         &     -0.0200         &     -0.0523         &     -0.0546\sym{*}  &     -0.0693\sym{**} &      -0.105\sym{***}&      -0.101\sym{***}\\
                    &   (0.00948)         &    (0.0256)         &    (0.0309)         &    (0.0319)         &    (0.0322)         &    (0.0311)         &    (0.0329)         &    (0.0327)         \\
[1em]
Population,log, lag             &     0.00174         &     0.00464         &     0.00749         &     0.00977         &      0.0121         &      0.0142         &      0.0263         &      0.0353         \\
                    &   (0.00160)         &   (0.00321)         &   (0.00479)         &   (0.00634)         &   (0.00787)         &   (0.00942)         &    (0.0167)         &    (0.0226)         \\
[1em]
Irre.leade.change    &    -0.00278         &      0.0124\sym{**} &      0.0124\sym{**} &      0.0131\sym{**} &      0.0106\sym{**} &     0.00786         &     0.00777         &    0.000133         \\
                    &   (0.00326)         &   (0.00484)         &   (0.00512)         &   (0.00516)         &   (0.00519)         &   (0.00539)         &   (0.00598)         &   (0.00534)         \\
[1em]
$\Delta$Energy.consump,log            &   0.0000585         &   -0.000446         &    -0.00117         &    -0.00205         &    -0.00285\sym{*}  &    -0.00449\sym{**} &    -0.00498         &    -0.00567         \\
                    &  (0.000896)         &  (0.000968)         &   (0.00121)         &   (0.00143)         &   (0.00166)         &   (0.00227)         &   (0.00314)         &   (0.00392)         \\
[1em]
Energy.consump,log, lag           &    0.000182         &    0.000213         &    0.000290         &    0.000515         &    0.000819         &     0.00114         &     0.00277         &     0.00535\sym{*}  \\
                    &  (0.000264)         &  (0.000537)         &  (0.000783)         &   (0.00103)         &   (0.00127)         &   (0.00151)         &   (0.00256)         &   (0.00321)         \\
[1em]
Year                &    0.000212\sym{***}&    0.000435\sym{***}&    0.000673\sym{***}&    0.000924\sym{***}&     0.00116\sym{***}&     0.00139\sym{***}&     0.00249\sym{***}&     0.00348\sym{***}\\
                    & (0.0000324)         & (0.0000609)         & (0.0000900)         &  (0.000118)         &  (0.000148)         &  (0.000177)         &  (0.000327)         &  (0.000469)         \\
[1em]
$\Delta$Neighboring empowerment, lag&     0.00685\sym{*}  &      0.0206\sym{***}&      0.0263\sym{***}&      0.0293\sym{***}&      0.0226\sym{***}&      0.0284\sym{***}&      0.0242         &      0.0307\sym{*}  \\
                    &   (0.00360)         &   (0.00588)         &   (0.00896)         &   (0.00889)         &   (0.00851)         &   (0.00909)         &    (0.0151)         &    (0.0159)         \\
[1em]
Constant            &      -0.413\sym{***}&      -0.858\sym{***}&      -1.327\sym{***}&      -1.818\sym{***}&      -2.273\sym{***}&      -2.736\sym{***}&      -4.883\sym{***}&      -6.784\sym{***}\\
                    &    (0.0571)         &     (0.107)         &     (0.157)         &     (0.206)         &     (0.255)         &     (0.303)         &     (0.553)         &     (0.792)         \\
\hline
Observations        &        8135         &        7924         &        7743         &        7562         &        7390         &        7238         &        6556         &        6041         \\
\hline\hline
\multicolumn{9}{l}{\footnotesize Standard errors in parentheses}\\
\multicolumn{9}{l}{\footnotesize \sym{*} \(p<0.10\), \sym{**} \(p<0.05\), \sym{***} \(p<0.01\)}\\
\end{tabular}
\end{table}
 %label {fepolempnopolity} % index models  178-185
}

% Table C3
% replace polity with V-Dem index
{\renewcommand\normalsize{\tiny}% adjust table size
	\normalsize
\begin{table}[htbp]\centering
\def\sym#1{\ifmmode^{#1}\else\(^{#1}\)\fi}
\caption{Robustness Check: Fixed-effects models of the effect of civil society index and clean election index on future changes in women's empowerment \label{fepolempvdem}}
\begin{tabular}{l*{8}{c}}
\hline\hline
                    &\multicolumn{1}{c}{\shortstack{Model 186\\(current)}}&\multicolumn{1}{c}{\shortstack{Model 187\\(1-year)}}&\multicolumn{1}{c}{\shortstack{Model 188\\(2-year)}}&\multicolumn{1}{c}{\shortstack{Model 189\\(3-year)}}&\multicolumn{1}{c}{\shortstack{Model 190\\(4-year)}}&\multicolumn{1}{c}{\shortstack{Model 191\\(5-year)}}&\multicolumn{1}{c}{\shortstack{Model 192\\(10-year)}}&\multicolumn{1}{c}{\shortstack{Model 193\\(15-year)}}\\
\hline
War            &   -0.000990         &     0.00121         &     0.00448\sym{*}  &     0.00723\sym{**} &     0.00946\sym{**} &      0.0106\sym{**} &     0.00951         &     0.00369         \\
                    &   (0.00100)         &   (0.00176)         &   (0.00269)         &   (0.00335)         &   (0.00414)         &   (0.00481)         &   (0.00741)         &   (0.00864)         \\
[1em]
Political empowerment, lag &     -0.0488\sym{***}&     -0.0946\sym{***}&      -0.145\sym{***}&      -0.196\sym{***}&      -0.245\sym{***}&      -0.294\sym{***}&      -0.531\sym{***}&      -0.718\sym{***}\\
                    &   (0.00486)         &   (0.00966)         &    (0.0138)         &    (0.0177)         &    (0.0214)         &    (0.0251)         &    (0.0443)         &    (0.0599)         \\
[1em]
$\Delta$Civil society index        &       0.193\sym{***}&       0.257\sym{***}&       0.255\sym{***}&       0.247\sym{***}&       0.244\sym{***}&       0.237\sym{***}&       0.161\sym{***}&       0.128\sym{***}\\
                    &    (0.0180)         &    (0.0265)         &    (0.0292)         &    (0.0304)         &    (0.0317)         &    (0.0307)         &    (0.0297)         &    (0.0268)         \\
[1em]
Civil society index, lag      &      0.0133\sym{***}&      0.0193\sym{***}&      0.0235\sym{***}&      0.0271\sym{***}&      0.0292\sym{***}&      0.0307\sym{***}&      0.0345\sym{*}  &      0.0337         \\
                    &   (0.00228)         &   (0.00434)         &   (0.00644)         &   (0.00853)         &    (0.0102)         &    (0.0117)         &    (0.0205)         &    (0.0274)         \\
[1em]
$\Delta$Clean election index      &      0.0337\sym{***}&      0.0301\sym{***}&      0.0277\sym{***}&      0.0282\sym{**} &      0.0217\sym{*}  &      0.0299\sym{**} &      0.0247         &      0.0155         \\
                    &   (0.00875)         &    (0.0114)         &    (0.0106)         &    (0.0119)         &    (0.0129)         &    (0.0134)         &    (0.0192)         &    (0.0216)         \\
[1em]
Clean election index, lag        &     0.00359         &     0.00398         &     0.00476         &     0.00521         &     0.00806         &      0.0118         &      0.0298         &      0.0307         \\
                    &   (0.00248)         &   (0.00478)         &   (0.00728)         &    (0.0101)         &    (0.0126)         &    (0.0149)         &    (0.0255)         &    (0.0342)         \\
[1em]
$\Delta$Energy.consump,log             &  -0.0000925         &   -0.000955         &    -0.00110         &    -0.00207         &    -0.00248         &    -0.00408\sym{*}  &    -0.00492         &    -0.00491         \\
                    &  (0.000733)         &  (0.000888)         &   (0.00118)         &   (0.00133)         &   (0.00157)         &   (0.00213)         &   (0.00310)         &   (0.00375)         \\
[1em]
Energy.consump,log, lag             &    0.000149         &    0.000373         &    0.000776         &     0.00124         &     0.00178\sym{*}  &     0.00224\sym{**} &     0.00497\sym{**} &     0.00815\sym{***}\\
                    &  (0.000174)         &  (0.000346)         &  (0.000552)         &  (0.000752)         &  (0.000935)         &   (0.00110)         &   (0.00197)         &   (0.00276)         \\
[1em]
Year                &    0.000228\sym{***}&    0.000490\sym{***}&    0.000779\sym{***}&     0.00107\sym{***}&     0.00134\sym{***}&     0.00162\sym{***}&     0.00291\sym{***}&     0.00392\sym{***}\\
                    & (0.0000276)         & (0.0000565)         & (0.0000862)         &  (0.000114)         &  (0.000140)         &  (0.000166)         &  (0.000300)         &  (0.000420)         \\
[1em]
$\Delta$Neighboring empowerment, lag&    0.000678         &      0.0128\sym{**} &      0.0180\sym{**} &      0.0184\sym{**} &      0.0129         &      0.0165\sym{*}  &      0.0109         &      0.0192         \\
                    &   (0.00335)         &   (0.00549)         &   (0.00831)         &   (0.00843)         &   (0.00818)         &   (0.00873)         &    (0.0139)         &    (0.0132)         \\
[1em]
Constant            &      -0.432\sym{***}&      -0.926\sym{***}&      -1.471\sym{***}&      -2.021\sym{***}&      -2.529\sym{***}&      -3.052\sym{***}&      -5.468\sym{***}&      -7.363\sym{***}\\
                    &    (0.0523)         &     (0.107)         &     (0.163)         &     (0.216)         &     (0.266)         &     (0.315)         &     (0.570)         &     (0.798)         \\
\hline
Observations        &        8408         &        8193         &        8004         &        7816         &        7639         &        7480         &        6756         &        6200         \\
\hline\hline
\multicolumn{9}{l}{\footnotesize Standard errors in parentheses}\\
\multicolumn{9}{l}{\footnotesize \sym{*} \(p<0.10\), \sym{**} \(p<0.05\), \sym{***} \(p<0.01\)}\\
\end{tabular}
\end{table}
 %label {fepolempvdem} % index models  186-193
}

% Table C4
%{\it national strikes}, {\it government crises}, {\it riots} and {\it mass demonstrations}
{\renewcommand\normalsize{\tiny}% adjust table size
	\normalsize
\begin{table}[htbp]\centering
\def\sym#1{\ifmmode^{#1}\else\(^{#1}\)\fi}
\caption{Robustness Check: Fixed-effects models of the effect of strikes on future changes in women's empowerment \label{fepolemstriksnowar}}
\begin{tabular}{l*{8}{c}}
\hline\hline
                    &\multicolumn{1}{c}{\shortstack{Model 194\\(current)}}&\multicolumn{1}{c}{\shortstack{Model 195\\(1-year)}}&\multicolumn{1}{c}{\shortstack{Model 196\\(2-year)}}&\multicolumn{1}{c}{\shortstack{Model 197\\(3-year)}}&\multicolumn{1}{c}{\shortstack{Model 198\\(4-year)}}&\multicolumn{1}{c}{\shortstack{Model 199\\(5-year)}}&\multicolumn{1}{c}{\shortstack{Model 200\\(10-year)}}&\multicolumn{1}{c}{\shortstack{Model 201\\(15-year)}}\\
\hline
Strike, dummy          &     0.00249\sym{*}  &     0.00389\sym{*}  &     0.00628\sym{**} &     0.00907\sym{***}&      0.0104\sym{***}&      0.0121\sym{***}&      0.0223\sym{***}&      0.0191\sym{**} \\
                    &   (0.00132)         &   (0.00204)         &   (0.00246)         &   (0.00280)         &   (0.00369)         &   (0.00396)         &   (0.00714)         &   (0.00782)         \\
[1em]
Political empowerment, lag  &     -0.0512\sym{***}&     -0.0952\sym{***}&      -0.141\sym{***}&      -0.192\sym{***}&      -0.240\sym{***}&      -0.291\sym{***}&      -0.513\sym{***}&      -0.679\sym{***}\\
                    &   (0.00485)         &   (0.00851)         &    (0.0120)         &    (0.0158)         &    (0.0194)         &    (0.0232)         &    (0.0383)         &    (0.0433)         \\
[1em]
$\Delta$Polity scores         &     0.00318\sym{***}&     0.00569\sym{***}&     0.00591\sym{***}&     0.00532\sym{***}&     0.00500\sym{***}&     0.00475\sym{***}&     0.00347\sym{***}&     0.00344\sym{***}\\
                    &  (0.000433)         &  (0.000756)         &  (0.000862)         &  (0.000858)         &  (0.000909)         &  (0.000881)         &  (0.000781)         &  (0.000824)         \\
[1em]
Polity scores, lag          &    0.000691\sym{***}&    0.000916\sym{***}&     0.00102\sym{***}&     0.00116\sym{***}&     0.00128\sym{***}&     0.00148\sym{***}&     0.00186\sym{**} &     0.00119         \\
                    & (0.0000931)         &  (0.000162)         &  (0.000228)         &  (0.000297)         &  (0.000361)         &  (0.000435)         &  (0.000784)         &  (0.000928)         \\
[1em]
$\Delta$Energy.consump,log          &    0.000498         &  -0.0000248         &   -0.000230         &   -0.000942         &    -0.00147         &    -0.00334         &    -0.00469         &    -0.00481         \\
                    &  (0.000839)         &  (0.000905)         &   (0.00111)         &   (0.00136)         &   (0.00160)         &   (0.00222)         &   (0.00313)         &   (0.00380)         \\
[1em]
Energy.consump,log, lag             &    0.000251         &    0.000530         &    0.000918         &     0.00135\sym{*}  &     0.00181\sym{*}  &     0.00226\sym{*}  &     0.00467\sym{**} &     0.00786\sym{***}\\
                    &  (0.000203)         &  (0.000382)         &  (0.000579)         &  (0.000790)         &  (0.000980)         &   (0.00115)         &   (0.00203)         &   (0.00284)         \\
[1em]
Year                &    0.000250\sym{***}&    0.000506\sym{***}&    0.000777\sym{***}&     0.00107\sym{***}&     0.00134\sym{***}&     0.00163\sym{***}&     0.00290\sym{***}&     0.00383\sym{***}\\
                    & (0.0000294)         & (0.0000552)         & (0.0000818)         &  (0.000110)         &  (0.000135)         &  (0.000160)         &  (0.000278)         &  (0.000370)         \\
[1em]
$\Delta$Neighboring empowerment, lag&     0.00553         &      0.0182\sym{***}&      0.0219\sym{**} &      0.0229\sym{***}&      0.0179\sym{**} &      0.0209\sym{**} &      0.0106         &      0.0227\sym{*}  \\
                    &   (0.00355)         &   (0.00570)         &   (0.00853)         &   (0.00864)         &   (0.00842)         &   (0.00884)         &    (0.0142)         &    (0.0135)         \\
[1em]
Constant            &      -0.466\sym{***}&      -0.945\sym{***}&      -1.455\sym{***}&      -2.000\sym{***}&      -2.519\sym{***}&      -3.054\sym{***}&      -5.428\sym{***}&      -7.177\sym{***}\\
                    &    (0.0550)         &     (0.103)         &     (0.154)         &     (0.206)         &     (0.253)         &     (0.301)         &     (0.525)         &     (0.699)         \\
\hline
Observations        &        8090         &        7889         &        7708         &        7527         &        7362         &        7209         &        6538         &        6009         \\
\hline\hline
\multicolumn{9}{l}{\footnotesize Standard errors in parentheses}\\
\multicolumn{9}{l}{\footnotesize \sym{*} \(p<0.10\), \sym{**} \(p<0.05\), \sym{***} \(p<0.01\)}\\
\end{tabular}
\end{table}
 %label {fepolemstriksnowar} % index models  194-201
}

%Table C5
{\renewcommand\normalsize{\tiny}% adjust table size
	\normalsize
\begin{table}[htbp]\centering
\def\sym#1{\ifmmode^{#1}\else\(^{#1}\)\fi}
\caption{Robustness Check: Fixed-effects models of the effect of government crises on future changes in women's empowerment \label{polemgovnowar}}
\begin{tabular}{l*{8}{c}}
\hline\hline
                    &\multicolumn{1}{c}{\shortstack{Model 202\\(current)}}&\multicolumn{1}{c}{\shortstack{Model 203\\(1-year)}}&\multicolumn{1}{c}{\shortstack{Model 204\\(2-year)}}&\multicolumn{1}{c}{\shortstack{Model 205\\(3-year)}}&\multicolumn{1}{c}{\shortstack{Model 206\\(4-year)}}&\multicolumn{1}{c}{\shortstack{Model 207\\(5-year)}}&\multicolumn{1}{c}{\shortstack{Model 208\\(10-year)}}&\multicolumn{1}{c}{\shortstack{Model 209\\(15-year)}}\\
\hline
Gov.crises, dummy    &    0.000808         &     0.00187         &     0.00260         &     0.00280         &     0.00297         &     0.00152         &    -0.00239         &    -0.00349         \\
                    &  (0.000920)         &   (0.00137)         &   (0.00180)         &   (0.00227)         &   (0.00259)         &   (0.00279)         &   (0.00403)         &   (0.00475)         \\
[1em]
Political empowerment, lag   &     -0.0510\sym{***}&     -0.0948\sym{***}&      -0.141\sym{***}&      -0.192\sym{***}&      -0.240\sym{***}&      -0.291\sym{***}&      -0.514\sym{***}&      -0.679\sym{***}\\
                    &   (0.00487)         &   (0.00849)         &    (0.0119)         &    (0.0157)         &    (0.0193)         &    (0.0231)         &    (0.0382)         &    (0.0433)         \\
[1em]
$\Delta$Polity score            &     0.00318\sym{***}&     0.00570\sym{***}&     0.00594\sym{***}&     0.00538\sym{***}&     0.00506\sym{***}&     0.00480\sym{***}&     0.00355\sym{***}&     0.00357\sym{***}\\
                    &  (0.000434)         &  (0.000758)         &  (0.000869)         &  (0.000869)         &  (0.000925)         &  (0.000894)         &  (0.000796)         &  (0.000836)         \\
[1em]
Polity score, lag           &    0.000695\sym{***}&    0.000917\sym{***}&     0.00103\sym{***}&     0.00118\sym{***}&     0.00130\sym{***}&     0.00153\sym{***}&     0.00202\sym{**} &     0.00137         \\
                    & (0.0000952)         &  (0.000164)         &  (0.000230)         &  (0.000299)         &  (0.000363)         &  (0.000436)         &  (0.000785)         &  (0.000926)         \\
[1em]
$\Delta$Energy.consump,log            &    0.000452         &  -0.0000941         &   -0.000281         &    -0.00102         &    -0.00157         &    -0.00345         &    -0.00489         &    -0.00489         \\
                    &  (0.000835)         &  (0.000899)         &   (0.00110)         &   (0.00135)         &   (0.00159)         &   (0.00220)         &   (0.00307)         &   (0.00378)         \\
[1em]
Energy.consump,log, lag           &    0.000274         &    0.000565         &    0.000975\sym{*}  &     0.00144\sym{*}  &     0.00192\sym{*}  &     0.00237\sym{**} &     0.00487\sym{**} &     0.00803\sym{***}\\
                    &  (0.000206)         &  (0.000389)         &  (0.000588)         &  (0.000799)         &  (0.000988)         &   (0.00117)         &   (0.00207)         &   (0.00287)         \\
[1em]
Year                &    0.000250\sym{***}&    0.000506\sym{***}&    0.000777\sym{***}&     0.00107\sym{***}&     0.00135\sym{***}&     0.00164\sym{***}&     0.00292\sym{***}&     0.00386\sym{***}\\
                    & (0.0000295)         & (0.0000553)         & (0.0000820)         &  (0.000110)         &  (0.000136)         &  (0.000161)         &  (0.000283)         &  (0.000375)         \\
[1em]
$\Delta$Neighboring empowerment, lag&     0.00601\sym{*}  &      0.0189\sym{***}&      0.0229\sym{***}&      0.0243\sym{***}&      0.0194\sym{**} &      0.0227\sym{**} &      0.0145         &      0.0250\sym{*}  \\
                    &   (0.00361)         &   (0.00574)         &   (0.00858)         &   (0.00870)         &   (0.00844)         &   (0.00878)         &    (0.0140)         &    (0.0133)         \\
[1em]
Constant            &      -0.466\sym{***}&      -0.945\sym{***}&      -1.456\sym{***}&      -2.004\sym{***}&      -2.524\sym{***}&      -3.067\sym{***}&      -5.473\sym{***}&      -7.227\sym{***}\\
                    &    (0.0553)         &     (0.104)         &     (0.154)         &     (0.207)         &     (0.255)         &     (0.303)         &     (0.533)         &     (0.709)         \\
\hline
Observations        &        8090         &        7889         &        7708         &        7527         &        7362         &        7209         &        6538         &        6009         \\
\hline\hline
\multicolumn{9}{l}{\footnotesize Standard errors in parentheses}\\
\multicolumn{9}{l}{\footnotesize \sym{*} \(p<0.10\), \sym{**} \(p<0.05\), \sym{***} \(p<0.01\)}\\
\end{tabular}
\end{table}
 %label {fepolemgovnowar} % index models  202-209
}

%Table C6
{\renewcommand\normalsize{\tiny}% adjust table size
	\normalsize
\begin{table}[htbp]\centering
\def\sym#1{\ifmmode^{#1}\else\(^{#1}\)\fi}
\caption{Robustness Check: Fixed-effects models of the effect of riots on future changes in women's empowerment \label{fepolemriotnowar}}
\begin{tabular}{l*{8}{c}}
\hline\hline
                    &\multicolumn{1}{c}{\shortstack{Model 210\\(current)}}&\multicolumn{1}{c}{\shortstack{Model 211\\(1-year)}}&\multicolumn{1}{c}{\shortstack{Model 212\\(2-year)}}&\multicolumn{1}{c}{\shortstack{Model 213\\(3-year)}}&\multicolumn{1}{c}{\shortstack{Model 214\\(4-year)}}&\multicolumn{1}{c}{\shortstack{Model 215\\(5-year)}}&\multicolumn{1}{c}{\shortstack{Model 216\\(10-year)}}&\multicolumn{1}{c}{\shortstack{Model 217\\(15-year)}}\\
\hline
Riots,dummy         &     0.00180\sym{*}  &     0.00248\sym{*}  &     0.00209         &     0.00295         &     0.00177         &    0.000935         &    0.000793         &     0.00106         \\
                    &  (0.000924)         &   (0.00137)         &   (0.00193)         &   (0.00234)         &   (0.00277)         &   (0.00321)         &   (0.00480)         &   (0.00505)         \\
[1em]
Political empowerment, lag &     -0.0505\sym{***}&     -0.0943\sym{***}&      -0.141\sym{***}&      -0.191\sym{***}&      -0.240\sym{***}&      -0.291\sym{***}&      -0.513\sym{***}&      -0.678\sym{***}\\
                    &   (0.00485)         &   (0.00843)         &    (0.0119)         &    (0.0157)         &    (0.0193)         &    (0.0232)         &    (0.0384)         &    (0.0432)         \\
[1em]
$\Delta$Polity scores            &     0.00318\sym{***}&     0.00570\sym{***}&     0.00595\sym{***}&     0.00537\sym{***}&     0.00506\sym{***}&     0.00480\sym{***}&     0.00355\sym{***}&     0.00356\sym{***}\\
                    &  (0.000434)         &  (0.000755)         &  (0.000867)         &  (0.000867)         &  (0.000921)         &  (0.000892)         &  (0.000792)         &  (0.000836)         \\
[1em]
Polity scores, lag          &    0.000691\sym{***}&    0.000917\sym{***}&     0.00104\sym{***}&     0.00119\sym{***}&     0.00132\sym{***}&     0.00154\sym{***}&     0.00199\sym{**} &     0.00132         \\
                    & (0.0000923)         &  (0.000161)         &  (0.000228)         &  (0.000297)         &  (0.000361)         &  (0.000435)         &  (0.000786)         &  (0.000935)         \\
[1em]
$\Delta$Energy.consump,log            &    0.000487         &  -0.0000381         &   -0.000235         &   -0.000966         &    -0.00152         &    -0.00344         &    -0.00493         &    -0.00490         \\
                    &  (0.000834)         &  (0.000899)         &   (0.00110)         &   (0.00136)         &   (0.00160)         &   (0.00220)         &   (0.00315)         &   (0.00382)         \\
[1em]
Energy.consump,log, lag             &    0.000265         &    0.000554         &    0.000962         &     0.00142\sym{*}  &     0.00190\sym{*}  &     0.00236\sym{**} &     0.00487\sym{**} &     0.00805\sym{***}\\
                    &  (0.000202)         &  (0.000384)         &  (0.000583)         &  (0.000793)         &  (0.000985)         &   (0.00116)         &   (0.00206)         &   (0.00287)         \\
[1em]
Year                &    0.000247\sym{***}&    0.000502\sym{***}&    0.000776\sym{***}&     0.00107\sym{***}&     0.00135\sym{***}&     0.00164\sym{***}&     0.00291\sym{***}&     0.00384\sym{***}\\
                    & (0.0000291)         & (0.0000544)         & (0.0000811)         &  (0.000109)         &  (0.000135)         &  (0.000161)         &  (0.000281)         &  (0.000370)         \\
[1em]
$\Delta$Neighboring empowerment, lag&     0.00571         &      0.0186\sym{***}&      0.0227\sym{***}&      0.0239\sym{***}&      0.0192\sym{**} &      0.0226\sym{**} &      0.0145         &      0.0251\sym{*}  \\
                    &   (0.00357)         &   (0.00568)         &   (0.00853)         &   (0.00867)         &   (0.00843)         &   (0.00881)         &    (0.0141)         &    (0.0133)         \\
[1em]
Constant            &      -0.460\sym{***}&      -0.939\sym{***}&      -1.453\sym{***}&      -1.998\sym{***}&      -2.525\sym{***}&      -3.067\sym{***}&      -5.460\sym{***}&      -7.205\sym{***}\\
                    &    (0.0545)         &     (0.102)         &     (0.152)         &     (0.205)         &     (0.254)         &     (0.302)         &     (0.530)         &     (0.700)         \\
\hline
Observations        &        8090         &        7889         &        7708         &        7527         &        7362         &        7209         &        6538         &        6009         \\
\hline\hline
\multicolumn{9}{l}{\footnotesize Standard errors in parentheses}\\
\multicolumn{9}{l}{\footnotesize \sym{*} \(p<0.10\), \sym{**} \(p<0.05\), \sym{***} \(p<0.01\)}\\
\end{tabular}
\end{table}
 %label {fepolemriotnowa} % index models  210-217
}

%Table C7
{\renewcommand\normalsize{\tiny}% adjust table size
	\normalsize
\begin{table}[htbp]\centering
\def\sym#1{\ifmmode^{#1}\else\(^{#1}\)\fi}
\caption{Robustness Check: Fixed-effects models of the effect of demonstrations on future changes in women's empowerment \label{polemdemonstrationnowar}}
\begin{tabular}{l*{8}{c}}
\hline\hline
                    &\multicolumn{1}{c}{\shortstack{Model 218\\(current)}}&\multicolumn{1}{c}{\shortstack{Model 219\\(1-year)}}&\multicolumn{1}{c}{\shortstack{Model 220\\(2-year)}}&\multicolumn{1}{c}{\shortstack{Model 221\\(3-year)}}&\multicolumn{1}{c}{\shortstack{Model 222\\(4-year)}}&\multicolumn{1}{c}{\shortstack{Model 223\\(5-year)}}&\multicolumn{1}{c}{\shortstack{Model 224\\(10-year)}}&\multicolumn{1}{c}{\shortstack{Model 225\\(15-year)}}\\
\hline
Demonstration,dummy &     0.00269\sym{***}&     0.00419\sym{***}&     0.00564\sym{***}&     0.00770\sym{***}&     0.00790\sym{***}&     0.00905\sym{***}&     0.00926\sym{**} &      0.0138\sym{***}\\
                    &  (0.000864)         &   (0.00123)         &   (0.00171)         &   (0.00212)         &   (0.00270)         &   (0.00328)         &   (0.00418)         &   (0.00464)         \\
[1em]
Political empowerment, lag    &     -0.0509\sym{***}&     -0.0949\sym{***}&      -0.141\sym{***}&      -0.192\sym{***}&      -0.239\sym{***}&      -0.290\sym{***}&      -0.513\sym{***}&      -0.677\sym{***}\\
                    &   (0.00488)         &   (0.00853)         &    (0.0120)         &    (0.0158)         &    (0.0195)         &    (0.0233)         &    (0.0384)         &    (0.0434)         \\
[1em]
$\Delta$Polity scores            &     0.00317\sym{***}&     0.00567\sym{***}&     0.00589\sym{***}&     0.00531\sym{***}&     0.00499\sym{***}&     0.00472\sym{***}&     0.00347\sym{***}&     0.00341\sym{***}\\
                    &  (0.000435)         &  (0.000754)         &  (0.000862)         &  (0.000860)         &  (0.000911)         &  (0.000877)         &  (0.000777)         &  (0.000816)         \\
[1em]
Polity scores, lag           &    0.000689\sym{***}&    0.000912\sym{***}&     0.00102\sym{***}&     0.00116\sym{***}&     0.00128\sym{***}&     0.00150\sym{***}&     0.00194\sym{**} &     0.00124         \\
                    & (0.0000938)         &  (0.000163)         &  (0.000230)         &  (0.000300)         &  (0.000365)         &  (0.000438)         &  (0.000787)         &  (0.000936)         \\
[1em]
$\Delta$Energy.consump,log            &    0.000496         &   0.0000120         &   -0.000190         &   -0.000867         &    -0.00142         &    -0.00334         &    -0.00484         &    -0.00475         \\
                    &  (0.000835)         &  (0.000901)         &   (0.00111)         &   (0.00135)         &   (0.00159)         &   (0.00221)         &   (0.00316)         &   (0.00387)         \\
[1em]
Energy.consump,log, lag            &    0.000241         &    0.000518         &    0.000903         &     0.00133\sym{*}  &     0.00180\sym{*}  &     0.00223\sym{*}  &     0.00476\sym{**} &     0.00788\sym{***}\\
                    &  (0.000201)         &  (0.000381)         &  (0.000575)         &  (0.000782)         &  (0.000967)         &   (0.00114)         &   (0.00203)         &   (0.00284)         \\
[1em]
Year                &    0.000243\sym{***}&    0.000495\sym{***}&    0.000763\sym{***}&     0.00105\sym{***}&     0.00133\sym{***}&     0.00161\sym{***}&     0.00288\sym{***}&     0.00380\sym{***}\\
                    & (0.0000294)         & (0.0000546)         & (0.0000808)         &  (0.000109)         &  (0.000135)         &  (0.000160)         &  (0.000280)         &  (0.000369)         \\
[1em]
$\Delta$Neighboring empowerment, lag&     0.00584         &      0.0189\sym{***}&      0.0233\sym{***}&      0.0248\sym{***}&      0.0199\sym{**} &      0.0233\sym{***}&      0.0154         &      0.0266\sym{**} \\
                    &   (0.00362)         &   (0.00571)         &   (0.00855)         &   (0.00864)         &   (0.00828)         &   (0.00876)         &    (0.0140)         &    (0.0134)         \\
[1em]
Constant            &      -0.452\sym{***}&      -0.925\sym{***}&      -1.429\sym{***}&      -1.966\sym{***}&      -2.487\sym{***}&      -3.020\sym{***}&      -5.405\sym{***}&      -7.122\sym{***}\\
                    &    (0.0551)         &     (0.102)         &     (0.152)         &     (0.204)         &     (0.253)         &     (0.300)         &     (0.528)         &     (0.697)         \\
\hline
Observations        &        8090         &        7889         &        7708         &        7527         &        7362         &        7209         &        6538         &        6009         \\
\hline\hline
\multicolumn{9}{l}{\footnotesize Standard errors in parentheses}\\
\multicolumn{9}{l}{\footnotesize \sym{*} \(p<0.10\), \sym{**} \(p<0.05\), \sym{***} \(p<0.01\)}\\
\end{tabular}
\end{table}
 %label {fepolemdemonstrationnowar} % index models  218-225
}

%Table C8
% add war
{\renewcommand\normalsize{\tiny}% adjust table size
	\normalsize
\begin{table}[htbp]\centering
\def\sym#1{\ifmmode^{#1}\else\(^{#1}\)\fi}
\caption{Robustness Check: Fixed-effects models of the effect of strikes on future changes in women's empowerment (controlling for war) \label{fepolemstrikswar}}
\begin{tabular}{l*{8}{c}}
\hline\hline
                    &\multicolumn{1}{c}{\shortstack{Model 226\\(current)}}&\multicolumn{1}{c}{\shortstack{Model 227\\(1-year)}}&\multicolumn{1}{c}{\shortstack{Model 228\\(2-year)}}&\multicolumn{1}{c}{\shortstack{Model 229\\(3-year)}}&\multicolumn{1}{c}{\shortstack{Model 230\\(4-year)}}&\multicolumn{1}{c}{\shortstack{Model 231\\(5-year)}}&\multicolumn{1}{c}{\shortstack{Model 232\\(10-year)}}&\multicolumn{1}{c}{\shortstack{Model 233\\(15-year)}}\\
\hline
Strike, dummy          &     0.00249\sym{*}  &     0.00390\sym{*}  &     0.00628\sym{**} &     0.00905\sym{***}&      0.0104\sym{***}&      0.0121\sym{***}&      0.0223\sym{***}&      0.0191\sym{**} \\
                    &   (0.00132)         &   (0.00204)         &   (0.00246)         &   (0.00280)         &   (0.00370)         &   (0.00396)         &   (0.00714)         &   (0.00781)         \\
[1em]
War          &    -0.00156         &    0.000663         &     0.00346         &     0.00694\sym{*}  &     0.00928\sym{**} &      0.0101\sym{*}  &      0.0100         &     0.00502         \\
                    &   (0.00125)         &   (0.00209)         &   (0.00294)         &   (0.00363)         &   (0.00445)         &   (0.00525)         &   (0.00800)         &   (0.00911)         \\
[1em]
Political empowerment, lag  &     -0.0519\sym{***}&     -0.0949\sym{***}&      -0.140\sym{***}&      -0.189\sym{***}&      -0.235\sym{***}&      -0.285\sym{***}&      -0.508\sym{***}&      -0.676\sym{***}\\
                    &   (0.00485)         &   (0.00848)         &    (0.0118)         &    (0.0155)         &    (0.0190)         &    (0.0227)         &    (0.0381)         &    (0.0445)         \\
[1em]
$\Delta$Polity scores              &     0.00317\sym{***}&     0.00569\sym{***}&     0.00590\sym{***}&     0.00533\sym{***}&     0.00503\sym{***}&     0.00477\sym{***}&     0.00349\sym{***}&     0.00344\sym{***}\\
                    &  (0.000433)         &  (0.000756)         &  (0.000860)         &  (0.000854)         &  (0.000910)         &  (0.000879)         &  (0.000774)         &  (0.000824)         \\
[1em]
Polity scores         &    0.000698\sym{***}&    0.000913\sym{***}&     0.00100\sym{***}&     0.00113\sym{***}&     0.00123\sym{***}&     0.00143\sym{***}&     0.00179\sym{**} &     0.00115         \\
                    & (0.0000931)         &  (0.000162)         &  (0.000227)         &  (0.000295)         &  (0.000358)         &  (0.000431)         &  (0.000780)         &  (0.000939)         \\
[1em]
$\Delta$Energy.consump,log           &    0.000473         &  -0.0000137         &   -0.000158         &   -0.000774         &    -0.00124         &    -0.00302         &    -0.00442         &    -0.00466         \\
                    &  (0.000835)         &  (0.000896)         &   (0.00110)         &   (0.00134)         &   (0.00158)         &   (0.00220)         &   (0.00308)         &   (0.00374)         \\
[1em]
Energy.consump,log, lag            &    0.000264         &    0.000525         &    0.000896         &     0.00130         &     0.00175\sym{*}  &     0.00220\sym{*}  &     0.00458\sym{**} &     0.00780\sym{***}\\
                    &  (0.000202)         &  (0.000382)         &  (0.000578)         &  (0.000788)         &  (0.000975)         &   (0.00115)         &   (0.00201)         &   (0.00281)         \\
[1em]
Year                &    0.000252\sym{***}&    0.000505\sym{***}&    0.000774\sym{***}&     0.00106\sym{***}&     0.00133\sym{***}&     0.00161\sym{***}&     0.00288\sym{***}&     0.00382\sym{***}\\
                    & (0.0000294)         & (0.0000550)         & (0.0000813)         &  (0.000109)         &  (0.000133)         &  (0.000158)         &  (0.000279)         &  (0.000374)         \\
[1em]
$\Delta$Neighboring empowerment, lag&     0.00556         &      0.0182\sym{***}&      0.0217\sym{**} &      0.0224\sym{***}&      0.0175\sym{**} &      0.0206\sym{**} &      0.0105         &      0.0226\sym{*}  \\
                    &   (0.00355)         &   (0.00570)         &   (0.00848)         &   (0.00856)         &   (0.00832)         &   (0.00879)         &    (0.0142)         &    (0.0134)         \\
[1em]
Constant            &      -0.468\sym{***}&      -0.944\sym{***}&      -1.449\sym{***}&      -1.987\sym{***}&      -2.497\sym{***}&      -3.028\sym{***}&      -5.402\sym{***}&      -7.164\sym{***}\\
                    &    (0.0551)         &     (0.103)         &     (0.153)         &     (0.204)         &     (0.251)         &     (0.298)         &     (0.525)         &     (0.706)         \\
\hline
Observations        &        8090         &        7889         &        7708         &        7527         &        7362         &        7209         &        6538         &        6009         \\
\hline\hline
\multicolumn{9}{l}{\footnotesize Standard errors in parentheses}\\
\multicolumn{9}{l}{\footnotesize \sym{*} \(p<0.10\), \sym{**} \(p<0.05\), \sym{***} \(p<0.01\)}\\
\end{tabular}
\end{table}
 %label {fepolemstrikswar} % index models  226-233
}

%Table C9
{\renewcommand\normalsize{\tiny}% adjust table size
	\normalsize
\begin{table}[htbp]\centering
\def\sym#1{\ifmmode^{#1}\else\(^{#1}\)\fi}
\caption{Robustness Check: Fixed-effects models of the effect of demonstrations on future changes in women's empowerment(controlling for war) \label{polemdemonstrationwar}}
\begin{tabular}{l*{8}{c}}
\hline\hline
                    &\multicolumn{1}{c}{\shortstack{Model 234\\(current)}}&\multicolumn{1}{c}{\shortstack{Model 235\\(1-year)}}&\multicolumn{1}{c}{\shortstack{Model 236\\(2-year)}}&\multicolumn{1}{c}{\shortstack{Model 237\\(3-year)}}&\multicolumn{1}{c}{\shortstack{Model 238\\(4-year)}}&\multicolumn{1}{c}{\shortstack{Model 239\\(5-year)}}&\multicolumn{1}{c}{\shortstack{Model 240\\(10-year)}}&\multicolumn{1}{c}{\shortstack{Model 241\\(15-year)}}\\
\hline
Demonstration,dummy&     0.00269\sym{***}&     0.00419\sym{***}&     0.00562\sym{***}&     0.00766\sym{***}&     0.00789\sym{***}&     0.00901\sym{***}&     0.00926\sym{**} &      0.0138\sym{***}\\
                    &  (0.000863)         &   (0.00123)         &   (0.00170)         &   (0.00209)         &   (0.00267)         &   (0.00325)         &   (0.00416)         &   (0.00460)         \\
[1em]
War           &    -0.00158         &    0.000626         &     0.00340         &     0.00691\sym{*}  &     0.00928\sym{**} &      0.0100\sym{*}  &      0.0101         &     0.00519         \\
                    &   (0.00124)         &   (0.00208)         &   (0.00292)         &   (0.00360)         &   (0.00443)         &   (0.00523)         &   (0.00798)         &   (0.00910)         \\
[1em]
Political empowerment, lag   &     -0.0517\sym{***}&     -0.0946\sym{***}&      -0.139\sym{***}&      -0.189\sym{***}&      -0.235\sym{***}&      -0.285\sym{***}&      -0.507\sym{***}&      -0.674\sym{***}\\
                    &   (0.00488)         &   (0.00849)         &    (0.0118)         &    (0.0156)         &    (0.0191)         &    (0.0228)         &    (0.0382)         &    (0.0445)         \\
[1em]
$\Delta$Polity scores            &     0.00316\sym{***}&     0.00567\sym{***}&     0.00589\sym{***}&     0.00531\sym{***}&     0.00502\sym{***}&     0.00474\sym{***}&     0.00349\sym{***}&     0.00341\sym{***}\\
                    &  (0.000435)         &  (0.000754)         &  (0.000860)         &  (0.000856)         &  (0.000912)         &  (0.000876)         &  (0.000769)         &  (0.000816)         \\
[1em]
Polity scores, lag           &    0.000696\sym{***}&    0.000908\sym{***}&     0.00100\sym{***}&     0.00113\sym{***}&     0.00124\sym{***}&     0.00144\sym{***}&     0.00187\sym{**} &     0.00120         \\
                    & (0.0000938)         &  (0.000163)         &  (0.000229)         &  (0.000298)         &  (0.000361)         &  (0.000434)         &  (0.000783)         &  (0.000946)         \\
[1em]
$\Delta$Energy.consump,log             &    0.000470         &   0.0000225         &   -0.000119         &   -0.000702         &    -0.00119         &    -0.00301         &    -0.00457         &    -0.00460         \\
                    &  (0.000831)         &  (0.000892)         &   (0.00110)         &   (0.00133)         &   (0.00157)         &   (0.00220)         &   (0.00312)         &   (0.00381)         \\
[1em]
Energy.consump,log, lag             &    0.000254         &    0.000513         &    0.000882         &     0.00128         &     0.00173\sym{*}  &     0.00217\sym{*}  &     0.00466\sym{**} &     0.00783\sym{***}\\
                    &  (0.000200)         &  (0.000381)         &  (0.000574)         &  (0.000781)         &  (0.000965)         &   (0.00114)         &   (0.00201)         &   (0.00280)         \\
[1em]
Year                &    0.000245\sym{***}&    0.000495\sym{***}&    0.000760\sym{***}&     0.00104\sym{***}&     0.00132\sym{***}&     0.00160\sym{***}&     0.00287\sym{***}&     0.00379\sym{***}\\
                    & (0.0000295)         & (0.0000544)         & (0.0000803)         &  (0.000108)         &  (0.000133)         &  (0.000158)         &  (0.000280)         &  (0.000373)         \\
[1em]
$\Delta$Neighboring empowerment, lag&     0.00587         &      0.0189\sym{***}&      0.0232\sym{***}&      0.0244\sym{***}&      0.0194\sym{**} &      0.0229\sym{***}&      0.0153         &      0.0265\sym{**} \\
                    &   (0.00362)         &   (0.00571)         &   (0.00851)         &   (0.00857)         &   (0.00819)         &   (0.00873)         &    (0.0139)         &    (0.0134)         \\
[1em]
Constant            &      -0.455\sym{***}&      -0.924\sym{***}&      -1.423\sym{***}&      -1.952\sym{***}&      -2.466\sym{***}&      -2.994\sym{***}&      -5.379\sym{***}&      -7.108\sym{***}\\
                    &    (0.0553)         &     (0.102)         &     (0.151)         &     (0.202)         &     (0.250)         &     (0.297)         &     (0.528)         &     (0.703)         \\
\hline
Observations        &        8090         &        7889         &        7708         &        7527         &        7362         &        7209         &        6538         &        6009         \\
\hline\hline
\multicolumn{9}{l}{\footnotesize Standard errors in parentheses}\\
\multicolumn{9}{l}{\footnotesize \sym{*} \(p<0.10\), \sym{**} \(p<0.05\), \sym{***} \(p<0.01\)}\\
\end{tabular}
\end{table}
 %label {fepolemdemonstrationwar} % index models  234-241
}

% Table C10
% WWII 
{\renewcommand\normalsize{\tiny}% adjust table size
	\normalsize
\begin{table}[htbp]\centering
\def\sym#1{\ifmmode^{#1}\else\(^{#1}\)\fi}
\caption{Robustness Check: Fixed-effects models of the effect of WWII on future changes in women's empowerment \label{fepolemwwII}}
\begin{tabular}{l*{8}{c}}
\hline\hline
                    &\multicolumn{1}{c}{\shortstack{Model 242\\(current)}}&\multicolumn{1}{c}{\shortstack{Model 243\\(1-year)}}&\multicolumn{1}{c}{\shortstack{Model 244\\(2-year)}}&\multicolumn{1}{c}{\shortstack{Model 245\\(3-year)}}&\multicolumn{1}{c}{\shortstack{Model 246\\(4-year)}}&\multicolumn{1}{c}{\shortstack{Model 247\\(5-year)}}&\multicolumn{1}{c}{\shortstack{Model 248\\(10-year)}}&\multicolumn{1}{c}{\shortstack{Model 249\\(15-year)}}\\
\hline
War            &    -0.00151         &    0.000133         &     0.00617         &      0.0129\sym{**} &      0.0196\sym{**} &      0.0210\sym{**} &      0.0226         &      0.0108         \\
                    &   (0.00138)         &   (0.00297)         &   (0.00414)         &   (0.00551)         &   (0.00755)         &   (0.00899)         &    (0.0147)         &    (0.0153)         \\
[1em]
War $\times$ Post-1945 period&   0.0000270         &    0.000737         &    -0.00355         &    -0.00787         &     -0.0138         &     -0.0149         &     -0.0183         &    -0.00947         \\
                    &   (0.00204)         &   (0.00383)         &   (0.00529)         &   (0.00680)         &   (0.00892)         &    (0.0106)         &    (0.0162)         &    (0.0179)         \\
[1em]
Post-1945 period            &     0.00319\sym{**} &     0.00394         &     0.00355         &     0.00255         &    0.000772         &    -0.00292         &     -0.0190\sym{*}  &     -0.0257\sym{*}  \\
                    &   (0.00142)         &   (0.00266)         &   (0.00387)         &   (0.00513)         &   (0.00632)         &   (0.00741)         &    (0.0112)         &    (0.0148)         \\
[1em]
Political empowerment, lag  &     -0.0510\sym{***}&     -0.0937\sym{***}&      -0.139\sym{***}&      -0.189\sym{***}&      -0.236\sym{***}&      -0.287\sym{***}&      -0.514\sym{***}&      -0.682\sym{***}\\
                    &   (0.00485)         &   (0.00849)         &    (0.0118)         &    (0.0155)         &    (0.0189)         &    (0.0225)         &    (0.0376)         &    (0.0444)         \\
[1em]
$\Delta$Polity scores           &     0.00321\sym{***}&     0.00575\sym{***}&     0.00597\sym{***}&     0.00538\sym{***}&     0.00505\sym{***}&     0.00475\sym{***}&     0.00340\sym{***}&     0.00335\sym{***}\\
                    &  (0.000437)         &  (0.000766)         &  (0.000876)         &  (0.000872)         &  (0.000932)         &  (0.000899)         &  (0.000800)         &  (0.000863)         \\
[1em]
Polity scores, lag           &    0.000731\sym{***}&    0.000957\sym{***}&     0.00106\sym{***}&     0.00119\sym{***}&     0.00128\sym{***}&     0.00146\sym{***}&     0.00182\sym{**} &     0.00118         \\
                    & (0.0000965)         &  (0.000167)         &  (0.000236)         &  (0.000309)         &  (0.000373)         &  (0.000446)         &  (0.000788)         &  (0.000950)         \\
[1em]
$\Delta$Energy.consump,log             &    0.000261         &   -0.000292         &   -0.000353         &   -0.000917         &    -0.00117         &    -0.00269         &    -0.00317         &    -0.00298         \\
                    &  (0.000854)         &  (0.000910)         &   (0.00106)         &   (0.00125)         &   (0.00147)         &   (0.00200)         &   (0.00267)         &   (0.00338)         \\
[1em]
Energy.consump,log, lag             &    0.000310         &    0.000591         &    0.000977         &     0.00139\sym{*}  &     0.00182\sym{*}  &     0.00224\sym{*}  &     0.00445\sym{**} &     0.00751\sym{***}\\
                    &  (0.000205)         &  (0.000390)         &  (0.000592)         &  (0.000806)         &  (0.000999)         &   (0.00118)         &   (0.00209)         &   (0.00287)         \\
[1em]
Year                &    0.000211\sym{***}&    0.000454\sym{***}&    0.000734\sym{***}&     0.00104\sym{***}&     0.00135\sym{***}&     0.00169\sym{***}&     0.00322\sym{***}&     0.00427\sym{***}\\
                    & (0.0000323)         & (0.0000640)         & (0.0000967)         &  (0.000130)         &  (0.000162)         &  (0.000194)         &  (0.000340)         &  (0.000468)         \\
[1em]
$\Delta$Neighboring empowerment, lag&     0.00505         &      0.0177\sym{***}&      0.0219\sym{***}&      0.0236\sym{***}&      0.0195\sym{**} &      0.0239\sym{***}&      0.0207         &      0.0328\sym{**} \\
                    &   (0.00359)         &   (0.00555)         &   (0.00837)         &   (0.00877)         &   (0.00850)         &   (0.00897)         &    (0.0137)         &    (0.0138)         \\
[1em]
Constant            &      -0.391\sym{***}&      -0.847\sym{***}&      -1.374\sym{***}&      -1.952\sym{***}&      -2.529\sym{***}&      -3.168\sym{***}&      -6.046\sym{***}&      -8.017\sym{***}\\
                    &    (0.0607)         &     (0.120)         &     (0.182)         &     (0.246)         &     (0.306)         &     (0.365)         &     (0.642)         &     (0.885)         \\
\hline
Observations        &        8090         &        7889         &        7708         &        7527         &        7362         &        7209         &        6538         &        6009         \\
\hline\hline
\multicolumn{9}{l}{\footnotesize Standard errors in parentheses}\\
\multicolumn{9}{l}{\footnotesize \sym{*} \(p<0.10\), \sym{**} \(p<0.05\), \sym{***} \(p<0.01\)}\\
\end{tabular}
\end{table}
 %label {fepolemwwII} % index models  242-249
}

%%%%%%%%%%%%%%%% simple models
%Table C11
% lag of changes in controls
{\renewcommand\normalsize{\tiny}% adjust table size
	\normalsize
\begin{table}[htbp]\centering
\def\sym#1{\ifmmode^{#1}\else\(^{#1}\)\fi}
\caption{Robustness Check: Fixed-effects models of the effect of war on future changes in women's empowerment using lags of changes in control variables\label{polemprobustls}}
\begin{tabular}{l*{8}{c}}
\hline\hline
                    &\multicolumn{1}{c}{\shortstack{Model 250\\(current)}}&\multicolumn{1}{c}{\shortstack{Model 251\\(1-year)}}&\multicolumn{1}{c}{\shortstack{Model 252\\(2-year)}}&\multicolumn{1}{c}{\shortstack{Model 253\\(3-year)}}&\multicolumn{1}{c}{\shortstack{Model 254\\(4-year)}}&\multicolumn{1}{c}{\shortstack{Model 255\\(5-year)}}&\multicolumn{1}{c}{\shortstack{Model 256\\(10-year)}}&\multicolumn{1}{c}{\shortstack{Model 257\\(15-year)}}\\
\hline
New war              &    -0.00299         &    -0.00204         &    0.000685         &   0.0000649         &    0.000762         &     0.00279         &     0.00587         &     0.00162         \\
                    &   (0.00189)         &   (0.00291)         &   (0.00376)         &   (0.00443)         &   (0.00445)         &   (0.00467)         &   (0.00731)         &   (0.00873)         \\
[1em]
Ongoing war          &   -0.000489         &     0.00328         &     0.00739\sym{**} &      0.0123\sym{***}&      0.0146\sym{**} &      0.0148\sym{**} &      0.0131         &     0.00661         \\
                    &   (0.00146)         &   (0.00243)         &   (0.00352)         &   (0.00465)         &   (0.00581)         &   (0.00678)         &    (0.0100)         &    (0.0112)         \\
[1em]
Recent war            &     0.00591\sym{**} &     0.00803\sym{**} &     0.00912\sym{**} &     0.00617         &     0.00715         &     0.00794         &     0.00454         &    0.000330         \\
                    &   (0.00228)         &   (0.00339)         &   (0.00419)         &   (0.00448)         &   (0.00485)         &   (0.00519)         &   (0.00733)         &   (0.00835)         \\
[1em]
Political empowerment, lag   &     -0.0377\sym{***}&     -0.0804\sym{***}&      -0.128\sym{***}&      -0.176\sym{***}&      -0.226\sym{***}&      -0.275\sym{***}&      -0.496\sym{***}&      -0.661\sym{***}\\
                    &   (0.00465)         &   (0.00874)         &    (0.0123)         &    (0.0162)         &    (0.0200)         &    (0.0241)         &    (0.0384)         &    (0.0458)         \\
[1em]
$\Delta$Polity scores, lag          &     0.00279\sym{***}&     0.00310\sym{***}&     0.00256\sym{***}&     0.00263\sym{***}&     0.00220\sym{***}&     0.00197\sym{***}&     0.00105         &     0.00148\sym{*}  \\
                    &  (0.000416)         &  (0.000516)         &  (0.000597)         &  (0.000624)         &  (0.000643)         &  (0.000666)         &  (0.000760)         &  (0.000856)         \\
[1em]
Polity scores, lag          &    0.000136         &    0.000231         &    0.000358         &    0.000475         &    0.000681\sym{*}  &    0.000875\sym{*}  &     0.00141\sym{*}  &    0.000687         \\
                    & (0.0000899)         &  (0.000162)         &  (0.000233)         &  (0.000301)         &  (0.000378)         &  (0.000463)         &  (0.000813)         &   (0.00100)         \\
[1em]
$\Delta$Energy.consump,log, lag           &   -0.000700         &    -0.00118         &    -0.00178         &    -0.00284\sym{**} &    -0.00457\sym{***}&    -0.00457\sym{**} &    -0.00827\sym{**} &     -0.0120\sym{***}\\
                    &  (0.000477)         &  (0.000900)         &   (0.00108)         &   (0.00126)         &   (0.00175)         &   (0.00179)         &   (0.00320)         &   (0.00401)         \\
[1em]
Energy.consump,log, lag            &    0.000364\sym{*}  &    0.000714\sym{*}  &     0.00112\sym{*}  &     0.00160\sym{**} &     0.00210\sym{**} &     0.00257\sym{**} &     0.00531\sym{**} &     0.00863\sym{***}\\
                    &  (0.000184)         &  (0.000390)         &  (0.000598)         &  (0.000799)         &  (0.000982)         &   (0.00115)         &   (0.00204)         &   (0.00287)         \\
[1em]
Year                &    0.000224\sym{***}&    0.000481\sym{***}&    0.000762\sym{***}&     0.00104\sym{***}&     0.00132\sym{***}&     0.00160\sym{***}&     0.00285\sym{***}&     0.00376\sym{***}\\
                    & (0.0000278)         & (0.0000568)         & (0.0000838)         &  (0.000111)         &  (0.000136)         &  (0.000162)         &  (0.000284)         &  (0.000382)         \\
[1em]
Neighboring empowerment, lag&     0.00692\sym{*}  &      0.0198\sym{***}&      0.0240\sym{***}&      0.0254\sym{***}&      0.0196\sym{**} &      0.0248\sym{***}&      0.0162         &      0.0255\sym{*}  \\
                    &   (0.00366)         &   (0.00579)         &   (0.00872)         &   (0.00882)         &   (0.00844)         &   (0.00884)         &    (0.0140)         &    (0.0132)         \\
[1em]
Constant            &      -0.421\sym{***}&      -0.905\sym{***}&      -1.433\sym{***}&      -1.955\sym{***}&      -2.483\sym{***}&      -3.013\sym{***}&      -5.346\sym{***}&      -7.061\sym{***}\\
                    &    (0.0521)         &     (0.107)         &     (0.157)         &     (0.208)         &     (0.256)         &     (0.305)         &     (0.536)         &     (0.721)         \\
\hline
Observations        &        8099         &        7873         &        7691         &        7505         &        7340         &        7193         &        6524         &        5996         \\
\hline\hline
\multicolumn{9}{l}{\footnotesize Standard errors in parentheses}\\
\multicolumn{9}{l}{\footnotesize \sym{*} \(p<0.10\), \sym{**} \(p<0.05\), \sym{***} \(p<0.01\)}\\
\end{tabular}
\end{table}
 %label {polemprobustls} % index models  250-257
}

% Table C12
%year fixed effects
{\renewcommand\normalsize{\tiny}% adjust table size
	\normalsize
\begin{table}[htbp]\centering
\def\sym#1{\ifmmode^{#1}\else\(^{#1}\)\fi}
\caption{Robustness Check: Fixed-effects models of the effect of war on future changes in women's empowerment with year fixed effects\label{polemprobustyear}}
\begin{tabular}{l*{8}{c}}
\hline\hline
                    &\multicolumn{1}{c}{\shortstack{Model 258\\(current)}}&\multicolumn{1}{c}{\shortstack{Model 259\\(1-year)}}&\multicolumn{1}{c}{\shortstack{Model 260\\(2-year)}}&\multicolumn{1}{c}{\shortstack{Model 261\\(3-year)}}&\multicolumn{1}{c}{\shortstack{Model 262\\(4-year)}}&\multicolumn{1}{c}{\shortstack{Model 263\\(5-year)}}&\multicolumn{1}{c}{\shortstack{Model 264\\(10-year)}}&\multicolumn{1}{c}{\shortstack{Model 265\\(15-year)}}\\
\hline
New war             &    -0.00276         &    -0.00182         &   -0.000685         &    -0.00182         &    -0.00115         &     0.00104         &     0.00290         &   -0.000143         \\
                    &   (0.00187)         &   (0.00264)         &   (0.00360)         &   (0.00444)         &   (0.00445)         &   (0.00484)         &   (0.00757)         &   (0.00890)         \\
[1em]
Ongoing war          &    -0.00119         &    0.000950         &     0.00376         &     0.00790\sym{*}  &     0.00977\sym{*}  &     0.00934         &     0.00696         &     0.00546         \\
                    &   (0.00153)         &   (0.00252)         &   (0.00348)         &   (0.00447)         &   (0.00554)         &   (0.00659)         &    (0.0102)         &    (0.0116)         \\
[1em]
Recent war             &     0.00496\sym{**} &     0.00747\sym{**} &     0.00867\sym{**} &     0.00552         &     0.00586         &     0.00590         &     0.00484         &     0.00249         \\
                    &   (0.00232)         &   (0.00348)         &   (0.00409)         &   (0.00442)         &   (0.00479)         &   (0.00516)         &   (0.00751)         &   (0.00841)         \\
[1em]
Political empowerment, lag  &     -0.0401\sym{***}&     -0.0961\sym{***}&      -0.142\sym{***}&      -0.192\sym{***}&      -0.239\sym{***}&      -0.290\sym{***}&      -0.529\sym{***}&      -0.718\sym{***}\\
                    &   (0.00483)         &   (0.00882)         &    (0.0126)         &    (0.0166)         &    (0.0201)         &    (0.0240)         &    (0.0406)         &    (0.0481)         \\
[1em]
$\Delta$Polity scores, lag          &     0.00264\sym{***}&                     &                     &                     &                     &                     &                     &                     \\
                    &  (0.000404)         &                     &                     &                     &                     &                     &                     &                     \\
[1em]
Polity scores, lag            &    0.000178\sym{*}  &    0.000981\sym{***}&     0.00113\sym{***}&     0.00130\sym{***}&     0.00143\sym{***}&     0.00166\sym{***}&     0.00231\sym{***}&     0.00186\sym{*}  \\
                    & (0.0000984)         &  (0.000186)         &  (0.000263)         &  (0.000342)         &  (0.000414)         &  (0.000490)         &  (0.000828)         &  (0.000996)         \\
[1em]
$\Delta$Polity scores            &                     &     0.00544\sym{***}&     0.00564\sym{***}&     0.00507\sym{***}&     0.00469\sym{***}&     0.00439\sym{***}&     0.00322\sym{***}&     0.00323\sym{***}\\
                    &                     &  (0.000740)         &  (0.000842)         &  (0.000836)         &  (0.000887)         &  (0.000868)         &  (0.000817)         &  (0.000879)         \\
[1em]
$\Delta$Energy.consump,log, lag             &   -0.000200         &                     &                     &                     &                     &                     &                     &                     \\
                    &  (0.000594)         &                     &                     &                     &                     &                     &                     &                     \\
[1em]
Energy.consump,log, lag             &   0.0000706         &  -0.0000109         &   0.0000427         &    0.000109         &    0.000301         &    0.000480         &     0.00187         &     0.00408         \\
                    &  (0.000235)         &  (0.000519)         &  (0.000775)         &   (0.00105)         &   (0.00130)         &   (0.00154)         &   (0.00267)         &   (0.00347)         \\
[1em]
$\Delta$Energy.consump,log             &                     &    0.000607         &    0.000334         &   -0.000468         &    -0.00103         &    -0.00209         &    -0.00361         &    -0.00240         \\
                    &                     &   (0.00111)         &   (0.00122)         &   (0.00142)         &   (0.00161)         &   (0.00197)         &   (0.00261)         &   (0.00330)         \\
[1em]
$\Delta$Neighboring empowerment, lag&     0.00449         &      0.0125\sym{**} &      0.0156\sym{*}  &      0.0168\sym{**} &      0.0144\sym{*}  &      0.0170\sym{**} &      0.0151         &      0.0250\sym{*}  \\
                    &   (0.00353)         &   (0.00540)         &   (0.00795)         &   (0.00792)         &   (0.00763)         &   (0.00782)         &    (0.0135)         &    (0.0131)         \\
[1em]
Constant            &     0.00843\sym{***}&      0.0226\sym{***}&      0.0324\sym{***}&      0.0433\sym{***}&      0.0535\sym{***}&      0.0650\sym{***}&       0.118\sym{***}&       0.157\sym{***}\\
                    &   (0.00185)         &   (0.00391)         &   (0.00562)         &   (0.00741)         &   (0.00896)         &    (0.0106)         &    (0.0198)         &    (0.0261)         \\
\hline
Observations        &        8099         &        7889         &        7708         &        7527         &        7362         &        7209         &        6538         &        6009         \\
Year fixed effects           &           Yes        &     Yes          &    Yes          &   Yes         &     Yes          &     Yes           &    Yes          &    Yes         \\
\hline\hline
\multicolumn{9}{l}{\footnotesize Standard errors in parentheses}\\
\multicolumn{9}{l}{\footnotesize \sym{*} \(p<0.10\), \sym{**} \(p<0.05\), \sym{***} \(p<0.01\)}\\
\end{tabular}
\end{table}
 %label {polemprobustyear} % index models  258-265
}

% Table C13
%no control variables
{\renewcommand\normalsize{\tiny}% adjust table size
	\normalsize
\begin{table}[htbp]\centering
\def\sym#1{\ifmmode^{#1}\else\(^{#1}\)\fi}
\caption{Robustness Check: Fixed-effects models of the effect of war on future changes in women's empowerment with control variables\label{polemprobustwocontros}}
\begin{tabular}{l*{8}{c}}
\hline\hline
                    &\multicolumn{1}{c}{\shortstack{Model 266\\(current)}}&\multicolumn{1}{c}{\shortstack{Model 267\\(1-year)}}&\multicolumn{1}{c}{\shortstack{Model 268\\(2-year)}}&\multicolumn{1}{c}{\shortstack{Model 269\\(3-year)}}&\multicolumn{1}{c}{\shortstack{Model 270\\(4-year)}}&\multicolumn{1}{c}{\shortstack{Model 271\\(5-year)}}&\multicolumn{1}{c}{\shortstack{Model 272\\(10-year)}}&\multicolumn{1}{c}{\shortstack{Model 273\\(15-year)}}\\
\hline
New war               &    -0.00332\sym{*}  &    -0.00239         &     0.00112         &   -0.000122         &     0.00226         &     0.00543         &     0.00913         &     0.00653         \\
                    &   (0.00186)         &   (0.00277)         &   (0.00345)         &   (0.00423)         &   (0.00430)         &   (0.00467)         &   (0.00734)         &   (0.00961)         \\
[1em]
Ongoing war           &    0.000320         &     0.00540\sym{**} &      0.0109\sym{***}&      0.0177\sym{***}&      0.0220\sym{***}&      0.0242\sym{***}&      0.0318\sym{***}&      0.0335\sym{**} \\
                    &   (0.00132)         &   (0.00234)         &   (0.00350)         &   (0.00465)         &   (0.00586)         &   (0.00689)         &    (0.0107)         &    (0.0135)         \\
[1em]
Recent war           &     0.00671\sym{***}&     0.00886\sym{***}&      0.0112\sym{***}&     0.00853\sym{*}  &      0.0107\sym{**} &      0.0136\sym{**} &      0.0113         &     0.00616         \\
                    &   (0.00238)         &   (0.00336)         &   (0.00405)         &   (0.00443)         &   (0.00489)         &   (0.00525)         &   (0.00695)         &   (0.00953)         \\
[1em]
Constant            &     0.00467\sym{***}&     0.00889\sym{***}&      0.0133\sym{***}&      0.0180\sym{***}&      0.0228\sym{***}&      0.0277\sym{***}&      0.0546\sym{***}&      0.0833\sym{***}\\
                    &  (0.000126)         &  (0.000246)         &  (0.000359)         &  (0.000455)         &  (0.000557)         &  (0.000658)         &   (0.00109)         &   (0.00143)         \\
\hline
Observations        &        9628         &        9347         &        9123         &        8891         &        8680         &        8487         &        7603         &        6848         \\
\hline\hline
\multicolumn{9}{l}{\footnotesize Standard errors in parentheses}\\
\multicolumn{9}{l}{\footnotesize \sym{*} \(p<0.10\), \sym{**} \(p<0.05\), \sym{***} \(p<0.01\)}\\
\end{tabular}
\end{table}
 %label {polemprobustwocontros} % index models  266-273
}
% Table C14
% no controls with year fixed
{\renewcommand\normalsize{\tiny}% adjust table size
	\normalsize
\begin{table}[htbp]\centering
\def\sym#1{\ifmmode^{#1}\else\(^{#1}\)\fi}
\caption{Robustness Check: Fixed-effects models of the effect of war on future changes in women's empowerment with control variables and year fixed effects\label{polemprobustsimpyear}}
\begin{tabular}{l*{8}{c}}
\hline\hline
                    &\multicolumn{1}{c}{\shortstack{Model 274\\(current)}}&\multicolumn{1}{c}{\shortstack{Model 275\\(1-year)}}&\multicolumn{1}{c}{\shortstack{Model 276\\(2-year)}}&\multicolumn{1}{c}{\shortstack{Model 277\\(3-year)}}&\multicolumn{1}{c}{\shortstack{Model 278\\(4-year)}}&\multicolumn{1}{c}{\shortstack{Model 279\\(5-year)}}&\multicolumn{1}{c}{\shortstack{Model 280\\(10-year)}}&\multicolumn{1}{c}{\shortstack{Model 281\\(15-year)}}\\
\hline
New war              &    -0.00267         &    -0.00263         &    0.000899         &   -0.000797         &     0.00165         &     0.00487         &     0.00973         &     0.00995         \\
                    &   (0.00181)         &   (0.00264)         &   (0.00325)         &   (0.00397)         &   (0.00395)         &   (0.00445)         &   (0.00697)         &   (0.00894)         \\
[1em]
Ongoing war          &   -0.000376         &     0.00367         &     0.00820\sym{**} &      0.0140\sym{***}&      0.0173\sym{***}&      0.0190\sym{***}&      0.0254\sym{**} &      0.0312\sym{**} \\
                    &   (0.00135)         &   (0.00234)         &   (0.00345)         &   (0.00454)         &   (0.00563)         &   (0.00660)         &    (0.0103)         &    (0.0132)         \\
[1em]
Recent war             &     0.00542\sym{**} &     0.00741\sym{**} &     0.00951\sym{**} &     0.00706         &     0.00945\sym{**} &      0.0117\sym{**} &      0.0125\sym{*}  &      0.0116         \\
                    &   (0.00238)         &   (0.00340)         &   (0.00398)         &   (0.00430)         &   (0.00476)         &   (0.00507)         &   (0.00669)         &   (0.00917)         \\
[1em]
Constant            &    -0.00159\sym{***}&    -0.00276\sym{***}&    -0.00397\sym{***}&    -0.00433\sym{***}&    -0.00552\sym{***}&    -0.00626\sym{***}&    -0.00854\sym{**} &     -0.0102         \\
                    &  (0.000563)         &  (0.000886)         &   (0.00116)         &   (0.00142)         &   (0.00174)         &   (0.00223)         &   (0.00391)         &    (0.0111)         \\
\hline
Observations        &        9628         &        9347         &        9123         &        8891         &        8680         &        8487         &        7603         &        6848         \\
Year fixed effects           &           Yes        &     Yes          &    Yes          &   Yes         &     Yes          &     Yes           &    Yes          &    Yes         \\
\hline\hline
\multicolumn{9}{l}{\footnotesize Standard errors in parentheses}\\
\multicolumn{9}{l}{\footnotesize \sym{*} \(p<0.10\), \sym{**} \(p<0.05\), \sym{***} \(p<0.01\)}\\
\end{tabular}
\end{table}
 %label {polemprobustsimpyear} % index models  274-281
}



\section{Specification of the Endogenous Treatment Effects Regression Model}
\vspace*{.2in}
\setlength{\parskip}{-2em}
%\setlength{\parindent}{1cm}
\doublespacing
\setcounter{table}{0}
\setcounter{figure}{0}


The endogenous treatment-regression model is defined as the simultaneous estimation of the following two equations:
\begin{align}
& \Delta y_{i,t} = \mathbf{z_{i,t}} \beta + \delta x_{i,t} + \epsilon_{i,t} \\
& x_{i,t} = 
 \begin{cases}
 1, & \text{if}\ \mathbf{w_{i,t}} \gamma + \mu_{i,t} >0\\ %``\'' is break
 0, & otherwise
 \end{cases}
\end{align}
where $\Delta y_{i,t}$ is the outcome of interest (here, change in women's empowerment), and $x_{i,t}$ is the endogenous treatment (here, war and instability) for country $i$ at time $t$. Note that $\mathbf{z_{i,t}}$ are the covariates used to model the outcome and $\mathbf{w_{i,t}}$ are the covariates used to model the treatment, and both include time-lagged and change variables as discussed above. The error terms $\epsilon_{i,t}$ and $\mu_{i,t}$ are from the bivariate normal distribution with mean zero and covariance matrix  
\[\begin{bmatrix}
\sigma^2 & \rho \sigma \\
\rho \sigma & 1
\end{bmatrix}.\] \\

